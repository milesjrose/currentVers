%% Generated by Sphinx.
\def\sphinxdocclass{report}
\documentclass[letterpaper,10pt,english]{sphinxmanual}
\ifdefined\pdfpxdimen
   \let\sphinxpxdimen\pdfpxdimen\else\newdimen\sphinxpxdimen
\fi \sphinxpxdimen=.75bp\relax
\ifdefined\pdfimageresolution
    \pdfimageresolution= \numexpr \dimexpr1in\relax/\sphinxpxdimen\relax
\fi
%% let collapsible pdf bookmarks panel have high depth per default
\PassOptionsToPackage{bookmarksdepth=5}{hyperref}

\PassOptionsToPackage{booktabs}{sphinx}
\PassOptionsToPackage{colorrows}{sphinx}

\PassOptionsToPackage{warn}{textcomp}
\usepackage[utf8]{inputenc}
\ifdefined\DeclareUnicodeCharacter
% support both utf8 and utf8x syntaxes
  \ifdefined\DeclareUnicodeCharacterAsOptional
    \def\sphinxDUC#1{\DeclareUnicodeCharacter{"#1}}
  \else
    \let\sphinxDUC\DeclareUnicodeCharacter
  \fi
  \sphinxDUC{00A0}{\nobreakspace}
  \sphinxDUC{2500}{\sphinxunichar{2500}}
  \sphinxDUC{2502}{\sphinxunichar{2502}}
  \sphinxDUC{2514}{\sphinxunichar{2514}}
  \sphinxDUC{251C}{\sphinxunichar{251C}}
  \sphinxDUC{2572}{\textbackslash}
\fi
\usepackage{cmap}
\usepackage[T1]{fontenc}
\usepackage{amsmath,amssymb,amstext}
\usepackage{babel}



\usepackage{tgtermes}
\usepackage{tgheros}
\renewcommand{\ttdefault}{txtt}



\usepackage[Bjarne]{fncychap}
\usepackage{sphinx}

\fvset{fontsize=auto}
\usepackage{geometry}


% Include hyperref last.
\usepackage{hyperref}
% Fix anchor placement for figures with captions.
\usepackage{hypcap}% it must be loaded after hyperref.
% Set up styles of URL: it should be placed after hyperref.
\urlstyle{same}

\addto\captionsenglish{\renewcommand{\contentsname}{Contents:}}

\usepackage{sphinxmessages}
\setcounter{tocdepth}{1}



\title{DORA}
\date{Apr 01, 2025}
\release{0.2.3}
\author{Alex Doumas, Ivan Vegner, Miles Rose}
\newcommand{\sphinxlogo}{\vbox{}}
\renewcommand{\releasename}{Release}
\makeindex
\begin{document}

\ifdefined\shorthandoff
  \ifnum\catcode`\=\string=\active\shorthandoff{=}\fi
  \ifnum\catcode`\"=\active\shorthandoff{"}\fi
\fi

\pagestyle{empty}
\sphinxmaketitle
\pagestyle{plain}
\sphinxtableofcontents
\pagestyle{normal}
\phantomsection\label{\detokenize{index::doc}}


\sphinxAtStartPar
Add your content using \sphinxcode{\sphinxupquote{reStructuredText}} syntax. See the
\sphinxhref{https://www.sphinx-doc.org/en/master/usage/restructuredtext/index.html}{reStructuredText}
documentation for details.

\sphinxstepscope


\chapter{nodes package}
\label{\detokenize{nodes:nodes-package}}\label{\detokenize{nodes::doc}}

\section{Subpackages}
\label{\detokenize{nodes:subpackages}}
\sphinxstepscope


\subsection{nodes.nodeTests package}
\label{\detokenize{nodes.nodeTests:nodes-nodetests-package}}\label{\detokenize{nodes.nodeTests::doc}}

\subsubsection{Submodules}
\label{\detokenize{nodes.nodeTests:submodules}}

\subsubsection{nodes.nodeTests.test\_1 module}
\label{\detokenize{nodes.nodeTests:module-nodes.nodeTests.test_1}}\label{\detokenize{nodes.nodeTests:nodes-nodetests-test-1-module}}\index{module@\spxentry{module}!nodes.nodeTests.test\_1@\spxentry{nodes.nodeTests.test\_1}}\index{nodes.nodeTests.test\_1@\spxentry{nodes.nodeTests.test\_1}!module@\spxentry{module}}\index{test\_nodes\_from\_file() (in module nodes.nodeTests.test\_1)@\spxentry{test\_nodes\_from\_file()}\spxextra{in module nodes.nodeTests.test\_1}}

\begin{fulllineitems}
\phantomsection\label{\detokenize{nodes.nodeTests:nodes.nodeTests.test_1.test_nodes_from_file}}
\pysigstartsignatures
\pysiglinewithargsret
{\sphinxcode{\sphinxupquote{nodes.nodeTests.test\_1.}}\sphinxbfcode{\sphinxupquote{test\_nodes\_from\_file}}}
{}
{}
\pysigstopsignatures
\end{fulllineitems}



\subsubsection{Module contents}
\label{\detokenize{nodes.nodeTests:module-nodes.nodeTests}}\label{\detokenize{nodes.nodeTests:module-contents}}\index{module@\spxentry{module}!nodes.nodeTests@\spxentry{nodes.nodeTests}}\index{nodes.nodeTests@\spxentry{nodes.nodeTests}!module@\spxentry{module}}

\section{Submodules}
\label{\detokenize{nodes:submodules}}

\section{nodes.nodeBuilder module}
\label{\detokenize{nodes:module-nodes.nodeBuilder}}\label{\detokenize{nodes:nodes-nodebuilder-module}}\index{module@\spxentry{module}!nodes.nodeBuilder@\spxentry{nodes.nodeBuilder}}\index{nodes.nodeBuilder@\spxentry{nodes.nodeBuilder}!module@\spxentry{module}}\index{Build\_children (class in nodes.nodeBuilder)@\spxentry{Build\_children}\spxextra{class in nodes.nodeBuilder}}

\begin{fulllineitems}
\phantomsection\label{\detokenize{nodes:nodes.nodeBuilder.Build_children}}
\pysigstartsignatures
\pysiglinewithargsret
{\sphinxbfcode{\sphinxupquote{\DUrole{k}{class}\DUrole{w}{ }}}\sphinxcode{\sphinxupquote{nodes.nodeBuilder.}}\sphinxbfcode{\sphinxupquote{Build\_children}}}
{\sphinxparam{\DUrole{n}{set}\DUrole{p}{:}\DUrole{w}{ }\DUrole{n}{{\hyperref[\detokenize{nodes:nodes.nodeEnums.Set}]{\sphinxcrossref{Set}}}}}\sphinxparamcomma \sphinxparam{\DUrole{n}{tokens}\DUrole{p}{:}\DUrole{w}{ }\DUrole{n}{{\hyperref[\detokenize{nodes:nodes.nodeBuilder.Token_set}]{\sphinxcrossref{Token\_set}}}}}\sphinxparamcomma \sphinxparam{\DUrole{n}{sems}\DUrole{p}{:}\DUrole{w}{ }\DUrole{n}{{\hyperref[\detokenize{nodes:nodes.nodeBuilder.Sem_set}]{\sphinxcrossref{Sem\_set}}}}}\sphinxparamcomma \sphinxparam{\DUrole{n}{symProps}\DUrole{p}{:}\DUrole{w}{ }\DUrole{n}{list\DUrole{p}{{[}}dict\DUrole{p}{{]}}}}}
{}
\pysigstopsignatures
\sphinxAtStartPar
Bases: \sphinxcode{\sphinxupquote{object}}

\sphinxAtStartPar
A class for building the list of children for each token.
\index{set (nodes.nodeBuilder.Build\_children attribute)@\spxentry{set}\spxextra{nodes.nodeBuilder.Build\_children attribute}}

\begin{fulllineitems}
\phantomsection\label{\detokenize{nodes:nodes.nodeBuilder.Build_children.set}}
\pysigstartsignatures
\pysigline
{\sphinxbfcode{\sphinxupquote{set}}}
\pysigstopsignatures
\sphinxAtStartPar
The set of the tokens.
\begin{quote}\begin{description}
\sphinxlineitem{Type}
\sphinxAtStartPar
{\hyperref[\detokenize{nodes:nodes.nodeEnums.Set}]{\sphinxcrossref{Set}}}

\end{description}\end{quote}

\end{fulllineitems}

\index{tokens (nodes.nodeBuilder.Build\_children attribute)@\spxentry{tokens}\spxextra{nodes.nodeBuilder.Build\_children attribute}}

\begin{fulllineitems}
\phantomsection\label{\detokenize{nodes:nodes.nodeBuilder.Build_children.tokens}}
\pysigstartsignatures
\pysigline
{\sphinxbfcode{\sphinxupquote{tokens}}}
\pysigstopsignatures
\sphinxAtStartPar
The token set.
\begin{quote}\begin{description}
\sphinxlineitem{Type}
\sphinxAtStartPar
{\hyperref[\detokenize{nodes:nodes.nodeBuilder.Token_set}]{\sphinxcrossref{Token\_set}}}

\end{description}\end{quote}

\end{fulllineitems}

\index{sems (nodes.nodeBuilder.Build\_children attribute)@\spxentry{sems}\spxextra{nodes.nodeBuilder.Build\_children attribute}}

\begin{fulllineitems}
\phantomsection\label{\detokenize{nodes:nodes.nodeBuilder.Build_children.sems}}
\pysigstartsignatures
\pysigline
{\sphinxbfcode{\sphinxupquote{sems}}}
\pysigstopsignatures
\sphinxAtStartPar
The semantic set.
\begin{quote}\begin{description}
\sphinxlineitem{Type}
\sphinxAtStartPar
{\hyperref[\detokenize{nodes:nodes.nodeBuilder.Sem_set}]{\sphinxcrossref{Sem\_set}}}

\end{description}\end{quote}

\end{fulllineitems}

\index{symProps (nodes.nodeBuilder.Build\_children attribute)@\spxentry{symProps}\spxextra{nodes.nodeBuilder.Build\_children attribute}}

\begin{fulllineitems}
\phantomsection\label{\detokenize{nodes:nodes.nodeBuilder.Build_children.symProps}}
\pysigstartsignatures
\pysigline
{\sphinxbfcode{\sphinxupquote{symProps}}}
\pysigstopsignatures
\sphinxAtStartPar
A list of symProps relating to the set.
\begin{quote}\begin{description}
\sphinxlineitem{Type}
\sphinxAtStartPar
list

\end{description}\end{quote}

\end{fulllineitems}

\index{get\_children() (nodes.nodeBuilder.Build\_children method)@\spxentry{get\_children()}\spxextra{nodes.nodeBuilder.Build\_children method}}

\begin{fulllineitems}
\phantomsection\label{\detokenize{nodes:nodes.nodeBuilder.Build_children.get_children}}
\pysigstartsignatures
\pysiglinewithargsret
{\sphinxbfcode{\sphinxupquote{get\_children}}}
{}
{}
\pysigstopsignatures
\sphinxAtStartPar
Recursively add child nodes IDs to each token objects children list.

\end{fulllineitems}

\index{get\_object() (nodes.nodeBuilder.Build\_children method)@\spxentry{get\_object()}\spxextra{nodes.nodeBuilder.Build\_children method}}

\begin{fulllineitems}
\phantomsection\label{\detokenize{nodes:nodes.nodeBuilder.Build_children.get_object}}
\pysigstartsignatures
\pysiglinewithargsret
{\sphinxbfcode{\sphinxupquote{get\_object}}}
{\sphinxparam{\DUrole{n}{name}}}
{}
\pysigstopsignatures
\sphinxAtStartPar
Return token object if it exists. Else return None
\begin{quote}\begin{description}
\sphinxlineitem{Returns}
\sphinxAtStartPar
The token object.
None: If the token does not exist.

\sphinxlineitem{Return type}
\sphinxAtStartPar
token ({\hyperref[\detokenize{nodes:nodes.nodeBuilder.Token}]{\sphinxcrossref{Token}}})

\end{description}\end{quote}

\end{fulllineitems}

\index{get\_po\_children() (nodes.nodeBuilder.Build\_children method)@\spxentry{get\_po\_children()}\spxextra{nodes.nodeBuilder.Build\_children method}}

\begin{fulllineitems}
\phantomsection\label{\detokenize{nodes:nodes.nodeBuilder.Build_children.get_po_children}}
\pysigstartsignatures
\pysiglinewithargsret
{\sphinxbfcode{\sphinxupquote{get\_po\_children}}}
{\sphinxparam{\DUrole{n}{name}}\sphinxparamcomma \sphinxparam{\DUrole{n}{sems}\DUrole{p}{:}\DUrole{w}{ }\DUrole{n}{list}}}
{}
\pysigstopsignatures
\sphinxAtStartPar
Step three in recursively adding child nodes IDs to each token objects children list.

\end{fulllineitems}

\index{get\_prop\_children() (nodes.nodeBuilder.Build\_children method)@\spxentry{get\_prop\_children()}\spxextra{nodes.nodeBuilder.Build\_children method}}

\begin{fulllineitems}
\phantomsection\label{\detokenize{nodes:nodes.nodeBuilder.Build_children.get_prop_children}}
\pysigstartsignatures
\pysiglinewithargsret
{\sphinxbfcode{\sphinxupquote{get\_prop\_children}}}
{\sphinxparam{\DUrole{n}{prop}\DUrole{p}{:}\DUrole{w}{ }\DUrole{n}{dict}}}
{}
\pysigstopsignatures
\sphinxAtStartPar
Step one in recursively adding child nodes IDs to each token objects children list.

\end{fulllineitems}

\index{get\_rb\_children() (nodes.nodeBuilder.Build\_children method)@\spxentry{get\_rb\_children()}\spxextra{nodes.nodeBuilder.Build\_children method}}

\begin{fulllineitems}
\phantomsection\label{\detokenize{nodes:nodes.nodeBuilder.Build_children.get_rb_children}}
\pysigstartsignatures
\pysiglinewithargsret
{\sphinxbfcode{\sphinxupquote{get\_rb\_children}}}
{\sphinxparam{\DUrole{n}{rb}\DUrole{p}{:}\DUrole{w}{ }\DUrole{n}{dict}}}
{}
\pysigstopsignatures
\sphinxAtStartPar
Step two in recursively adding child nodes IDs to each token objects children list.

\end{fulllineitems}


\end{fulllineitems}

\index{Build\_connections (class in nodes.nodeBuilder)@\spxentry{Build\_connections}\spxextra{class in nodes.nodeBuilder}}

\begin{fulllineitems}
\phantomsection\label{\detokenize{nodes:nodes.nodeBuilder.Build_connections}}
\pysigstartsignatures
\pysiglinewithargsret
{\sphinxbfcode{\sphinxupquote{\DUrole{k}{class}\DUrole{w}{ }}}\sphinxcode{\sphinxupquote{nodes.nodeBuilder.}}\sphinxbfcode{\sphinxupquote{Build\_connections}}}
{\sphinxparam{\DUrole{n}{token\_sets}\DUrole{p}{:}\DUrole{w}{ }\DUrole{n}{dict\DUrole{p}{{[}}{\hyperref[\detokenize{nodes:nodes.nodeEnums.Set}]{\sphinxcrossref{Set}}}\DUrole{p}{,}\DUrole{w}{ }{\hyperref[\detokenize{nodes:nodes.nodeBuilder.Token_set}]{\sphinxcrossref{Token\_set}}}\DUrole{p}{{]}}}}\sphinxparamcomma \sphinxparam{\DUrole{n}{sems}\DUrole{p}{:}\DUrole{w}{ }\DUrole{n}{{\hyperref[\detokenize{nodes:nodes.nodeBuilder.Sem_set}]{\sphinxcrossref{Sem\_set}}}}}}
{}
\pysigstopsignatures
\sphinxAtStartPar
Bases: \sphinxcode{\sphinxupquote{object}}

\sphinxAtStartPar
A class for building the links and connections for each set.
\index{token\_sets (nodes.nodeBuilder.Build\_connections attribute)@\spxentry{token\_sets}\spxextra{nodes.nodeBuilder.Build\_connections attribute}}

\begin{fulllineitems}
\phantomsection\label{\detokenize{nodes:nodes.nodeBuilder.Build_connections.token_sets}}
\pysigstartsignatures
\pysigline
{\sphinxbfcode{\sphinxupquote{token\_sets}}}
\pysigstopsignatures
\sphinxAtStartPar
A dictionary of token sets, mapping set to token set.
\begin{quote}\begin{description}
\sphinxlineitem{Type}
\sphinxAtStartPar
dict

\end{description}\end{quote}

\end{fulllineitems}

\index{sems (nodes.nodeBuilder.Build\_connections attribute)@\spxentry{sems}\spxextra{nodes.nodeBuilder.Build\_connections attribute}}

\begin{fulllineitems}
\phantomsection\label{\detokenize{nodes:nodes.nodeBuilder.Build_connections.sems}}
\pysigstartsignatures
\pysigline
{\sphinxbfcode{\sphinxupquote{sems}}}
\pysigstopsignatures
\sphinxAtStartPar
The semantic set object.
\begin{quote}\begin{description}
\sphinxlineitem{Type}
\sphinxAtStartPar
{\hyperref[\detokenize{nodes:nodes.nodeBuilder.Sem_set}]{\sphinxcrossref{Sem\_set}}}

\end{description}\end{quote}

\end{fulllineitems}

\index{build\_connections\_links() (nodes.nodeBuilder.Build\_connections method)@\spxentry{build\_connections\_links()}\spxextra{nodes.nodeBuilder.Build\_connections method}}

\begin{fulllineitems}
\phantomsection\label{\detokenize{nodes:nodes.nodeBuilder.Build_connections.build_connections_links}}
\pysigstartsignatures
\pysiglinewithargsret
{\sphinxbfcode{\sphinxupquote{build\_connections\_links}}}
{}
{}
\pysigstopsignatures
\sphinxAtStartPar
Build the connections and links for each set.

\end{fulllineitems}

\index{build\_set\_connections() (nodes.nodeBuilder.Build\_connections method)@\spxentry{build\_set\_connections()}\spxextra{nodes.nodeBuilder.Build\_connections method}}

\begin{fulllineitems}
\phantomsection\label{\detokenize{nodes:nodes.nodeBuilder.Build_connections.build_set_connections}}
\pysigstartsignatures
\pysiglinewithargsret
{\sphinxbfcode{\sphinxupquote{build\_set\_connections}}}
{\sphinxparam{\DUrole{n}{token\_set}\DUrole{p}{:}\DUrole{w}{ }\DUrole{n}{{\hyperref[\detokenize{nodes:nodes.nodeBuilder.Token_set}]{\sphinxcrossref{Token\_set}}}}}}
{}
\pysigstopsignatures
\sphinxAtStartPar
Build the connections matrix for a given set.
\begin{quote}\begin{description}
\sphinxlineitem{Returns}
\sphinxAtStartPar
The NxN connections matrix for the set.

\sphinxlineitem{Return type}
\sphinxAtStartPar
connections (np.ndarray)

\end{description}\end{quote}

\end{fulllineitems}

\index{build\_set\_links() (nodes.nodeBuilder.Build\_connections method)@\spxentry{build\_set\_links()}\spxextra{nodes.nodeBuilder.Build\_connections method}}

\begin{fulllineitems}
\phantomsection\label{\detokenize{nodes:nodes.nodeBuilder.Build_connections.build_set_links}}
\pysigstartsignatures
\pysiglinewithargsret
{\sphinxbfcode{\sphinxupquote{build\_set\_links}}}
{\sphinxparam{\DUrole{n}{token\_set}\DUrole{p}{:}\DUrole{w}{ }\DUrole{n}{{\hyperref[\detokenize{nodes:nodes.nodeBuilder.Token_set}]{\sphinxcrossref{Token\_set}}}}}}
{}
\pysigstopsignatures
\sphinxAtStartPar
Build the links matrix for a given set.
\begin{quote}\begin{description}
\sphinxlineitem{Returns}
\sphinxAtStartPar
The NxM links matrix for the set.

\sphinxlineitem{Return type}
\sphinxAtStartPar
links (np.ndarray)

\end{description}\end{quote}

\end{fulllineitems}


\end{fulllineitems}

\index{Build\_sems (class in nodes.nodeBuilder)@\spxentry{Build\_sems}\spxextra{class in nodes.nodeBuilder}}

\begin{fulllineitems}
\phantomsection\label{\detokenize{nodes:nodes.nodeBuilder.Build_sems}}
\pysigstartsignatures
\pysiglinewithargsret
{\sphinxbfcode{\sphinxupquote{\DUrole{k}{class}\DUrole{w}{ }}}\sphinxcode{\sphinxupquote{nodes.nodeBuilder.}}\sphinxbfcode{\sphinxupquote{Build\_sems}}}
{\sphinxparam{\DUrole{n}{symProps}\DUrole{p}{:}\DUrole{w}{ }\DUrole{n}{list\DUrole{p}{{[}}dict\DUrole{p}{{]}}}}}
{}
\pysigstopsignatures
\sphinxAtStartPar
Bases: \sphinxcode{\sphinxupquote{object}}

\sphinxAtStartPar
A class for building the semantic objects.
\index{sems (nodes.nodeBuilder.Build\_sems attribute)@\spxentry{sems}\spxextra{nodes.nodeBuilder.Build\_sems attribute}}

\begin{fulllineitems}
\phantomsection\label{\detokenize{nodes:nodes.nodeBuilder.Build_sems.sems}}
\pysigstartsignatures
\pysigline
{\sphinxbfcode{\sphinxupquote{sems}}}
\pysigstopsignatures
\sphinxAtStartPar
A list of semantic names.
\begin{quote}\begin{description}
\sphinxlineitem{Type}
\sphinxAtStartPar
list

\end{description}\end{quote}

\end{fulllineitems}

\index{nodes (nodes.nodeBuilder.Build\_sems attribute)@\spxentry{nodes}\spxextra{nodes.nodeBuilder.Build\_sems attribute}}

\begin{fulllineitems}
\phantomsection\label{\detokenize{nodes:nodes.nodeBuilder.Build_sems.nodes}}
\pysigstartsignatures
\pysigline
{\sphinxbfcode{\sphinxupquote{nodes}}}
\pysigstopsignatures
\sphinxAtStartPar
A list of semantic objects.
\begin{quote}\begin{description}
\sphinxlineitem{Type}
\sphinxAtStartPar
list

\end{description}\end{quote}

\end{fulllineitems}

\index{name\_dict (nodes.nodeBuilder.Build\_sems attribute)@\spxentry{name\_dict}\spxextra{nodes.nodeBuilder.Build\_sems attribute}}

\begin{fulllineitems}
\phantomsection\label{\detokenize{nodes:nodes.nodeBuilder.Build_sems.name_dict}}
\pysigstartsignatures
\pysigline
{\sphinxbfcode{\sphinxupquote{name\_dict}}}
\pysigstopsignatures
\sphinxAtStartPar
A dictionary of semantic objects, mapping name to semantic object in nodes.
\begin{quote}\begin{description}
\sphinxlineitem{Type}
\sphinxAtStartPar
dict

\end{description}\end{quote}

\end{fulllineitems}

\index{id\_dict (nodes.nodeBuilder.Build\_sems attribute)@\spxentry{id\_dict}\spxextra{nodes.nodeBuilder.Build\_sems attribute}}

\begin{fulllineitems}
\phantomsection\label{\detokenize{nodes:nodes.nodeBuilder.Build_sems.id_dict}}
\pysigstartsignatures
\pysigline
{\sphinxbfcode{\sphinxupquote{id\_dict}}}
\pysigstopsignatures
\sphinxAtStartPar
A dictionary of semantic objects, mapping ID to semantic object in nodes.
\begin{quote}\begin{description}
\sphinxlineitem{Type}
\sphinxAtStartPar
dict

\end{description}\end{quote}

\end{fulllineitems}

\index{num\_sems (nodes.nodeBuilder.Build\_sems attribute)@\spxentry{num\_sems}\spxextra{nodes.nodeBuilder.Build\_sems attribute}}

\begin{fulllineitems}
\phantomsection\label{\detokenize{nodes:nodes.nodeBuilder.Build_sems.num_sems}}
\pysigstartsignatures
\pysigline
{\sphinxbfcode{\sphinxupquote{num\_sems}}}
\pysigstopsignatures
\sphinxAtStartPar
The number of semantics, iterated from 0 when assigning IDs.
\begin{quote}\begin{description}
\sphinxlineitem{Type}
\sphinxAtStartPar
int

\end{description}\end{quote}

\end{fulllineitems}

\index{symProps (nodes.nodeBuilder.Build\_sems attribute)@\spxentry{symProps}\spxextra{nodes.nodeBuilder.Build\_sems attribute}}

\begin{fulllineitems}
\phantomsection\label{\detokenize{nodes:nodes.nodeBuilder.Build_sems.symProps}}
\pysigstartsignatures
\pysigline
{\sphinxbfcode{\sphinxupquote{symProps}}}
\pysigstopsignatures
\sphinxAtStartPar
A list of symProps relating to the set.
\begin{quote}\begin{description}
\sphinxlineitem{Type}
\sphinxAtStartPar
list

\end{description}\end{quote}

\end{fulllineitems}

\index{build\_sems() (nodes.nodeBuilder.Build\_sems method)@\spxentry{build\_sems()}\spxextra{nodes.nodeBuilder.Build\_sems method}}

\begin{fulllineitems}
\phantomsection\label{\detokenize{nodes:nodes.nodeBuilder.Build_sems.build_sems}}
\pysigstartsignatures
\pysiglinewithargsret
{\sphinxbfcode{\sphinxupquote{build\_sems}}}
{}
{}
\pysigstopsignatures
\sphinxAtStartPar
Create the sem\_set object.

\end{fulllineitems}

\index{get\_sems() (nodes.nodeBuilder.Build\_sems method)@\spxentry{get\_sems()}\spxextra{nodes.nodeBuilder.Build\_sems method}}

\begin{fulllineitems}
\phantomsection\label{\detokenize{nodes:nodes.nodeBuilder.Build_sems.get_sems}}
\pysigstartsignatures
\pysiglinewithargsret
{\sphinxbfcode{\sphinxupquote{get\_sems}}}
{\sphinxparam{\DUrole{n}{symProps}\DUrole{p}{:}\DUrole{w}{ }\DUrole{n}{list\DUrole{p}{{[}}dict\DUrole{p}{{]}}}}}
{}
\pysigstopsignatures
\sphinxAtStartPar
Get the list of all semantic names in the symProps.

\end{fulllineitems}

\index{nodulate() (nodes.nodeBuilder.Build\_sems method)@\spxentry{nodulate()}\spxextra{nodes.nodeBuilder.Build\_sems method}}

\begin{fulllineitems}
\phantomsection\label{\detokenize{nodes:nodes.nodeBuilder.Build_sems.nodulate}}
\pysigstartsignatures
\pysiglinewithargsret
{\sphinxbfcode{\sphinxupquote{nodulate}}}
{}
{}
\pysigstopsignatures
\sphinxAtStartPar
Turn each unique semantic into a semantic object (node) with a unique ID.

\end{fulllineitems}


\end{fulllineitems}

\index{Build\_set (class in nodes.nodeBuilder)@\spxentry{Build\_set}\spxextra{class in nodes.nodeBuilder}}

\begin{fulllineitems}
\phantomsection\label{\detokenize{nodes:nodes.nodeBuilder.Build_set}}
\pysigstartsignatures
\pysiglinewithargsret
{\sphinxbfcode{\sphinxupquote{\DUrole{k}{class}\DUrole{w}{ }}}\sphinxcode{\sphinxupquote{nodes.nodeBuilder.}}\sphinxbfcode{\sphinxupquote{Build\_set}}}
{\sphinxparam{\DUrole{n}{symProps}\DUrole{p}{:}\DUrole{w}{ }\DUrole{n}{list\DUrole{p}{{[}}dict\DUrole{p}{{]}}}}\sphinxparamcomma \sphinxparam{\DUrole{n}{set}\DUrole{p}{:}\DUrole{w}{ }\DUrole{n}{{\hyperref[\detokenize{nodes:nodes.nodeEnums.Set}]{\sphinxcrossref{Set}}}}}}
{}
\pysigstopsignatures
\sphinxAtStartPar
Bases: \sphinxcode{\sphinxupquote{object}}

\sphinxAtStartPar
A class for building the nodes for a given set.
\index{symProps (nodes.nodeBuilder.Build\_set attribute)@\spxentry{symProps}\spxextra{nodes.nodeBuilder.Build\_set attribute}}

\begin{fulllineitems}
\phantomsection\label{\detokenize{nodes:nodes.nodeBuilder.Build_set.symProps}}
\pysigstartsignatures
\pysigline
{\sphinxbfcode{\sphinxupquote{symProps}}}
\pysigstopsignatures
\sphinxAtStartPar
A list of symProps relating to the set.
\begin{quote}\begin{description}
\sphinxlineitem{Type}
\sphinxAtStartPar
list

\end{description}\end{quote}

\end{fulllineitems}

\index{tokens (nodes.nodeBuilder.Build\_set attribute)@\spxentry{tokens}\spxextra{nodes.nodeBuilder.Build\_set attribute}}

\begin{fulllineitems}
\phantomsection\label{\detokenize{nodes:nodes.nodeBuilder.Build_set.tokens}}
\pysigstartsignatures
\pysigline
{\sphinxbfcode{\sphinxupquote{tokens}}}
\pysigstopsignatures
\sphinxAtStartPar
A dictionary of tokens, mapping type to list of tokens.
\begin{quote}\begin{description}
\sphinxlineitem{Type}
\sphinxAtStartPar
dict

\end{description}\end{quote}

\end{fulllineitems}

\index{set (nodes.nodeBuilder.Build\_set attribute)@\spxentry{set}\spxextra{nodes.nodeBuilder.Build\_set attribute}}

\begin{fulllineitems}
\phantomsection\label{\detokenize{nodes:nodes.nodeBuilder.Build_set.set}}
\pysigstartsignatures
\pysigline
{\sphinxbfcode{\sphinxupquote{set}}}
\pysigstopsignatures
\sphinxAtStartPar
The set to be built.
\begin{quote}\begin{description}
\sphinxlineitem{Type}
\sphinxAtStartPar
{\hyperref[\detokenize{nodes:nodes.nodeEnums.Set}]{\sphinxcrossref{Set}}}

\end{description}\end{quote}

\end{fulllineitems}

\index{name\_dict (nodes.nodeBuilder.Build\_set attribute)@\spxentry{name\_dict}\spxextra{nodes.nodeBuilder.Build\_set attribute}}

\begin{fulllineitems}
\phantomsection\label{\detokenize{nodes:nodes.nodeBuilder.Build_set.name_dict}}
\pysigstartsignatures
\pysigline
{\sphinxbfcode{\sphinxupquote{name\_dict}}}
\pysigstopsignatures
\sphinxAtStartPar
A dictionary of tokens, mapping name to token in tokens.
\begin{quote}\begin{description}
\sphinxlineitem{Type}
\sphinxAtStartPar
dict

\end{description}\end{quote}

\end{fulllineitems}

\index{id\_dict (nodes.nodeBuilder.Build\_set attribute)@\spxentry{id\_dict}\spxextra{nodes.nodeBuilder.Build\_set attribute}}

\begin{fulllineitems}
\phantomsection\label{\detokenize{nodes:nodes.nodeBuilder.Build_set.id_dict}}
\pysigstartsignatures
\pysigline
{\sphinxbfcode{\sphinxupquote{id\_dict}}}
\pysigstopsignatures
\sphinxAtStartPar
A dictionary of tokens, mapping ID to token in tokens.
\begin{quote}\begin{description}
\sphinxlineitem{Type}
\sphinxAtStartPar
dict

\end{description}\end{quote}

\end{fulllineitems}

\index{build\_set() (nodes.nodeBuilder.Build\_set method)@\spxentry{build\_set()}\spxextra{nodes.nodeBuilder.Build\_set method}}

\begin{fulllineitems}
\phantomsection\label{\detokenize{nodes:nodes.nodeBuilder.Build_set.build_set}}
\pysigstartsignatures
\pysiglinewithargsret
{\sphinxbfcode{\sphinxupquote{build\_set}}}
{}
{}
\pysigstopsignatures
\sphinxAtStartPar
Returns a new token\_set object

\end{fulllineitems}

\index{create\_token() (nodes.nodeBuilder.Build\_set method)@\spxentry{create\_token()}\spxextra{nodes.nodeBuilder.Build\_set method}}

\begin{fulllineitems}
\phantomsection\label{\detokenize{nodes:nodes.nodeBuilder.Build_set.create_token}}
\pysigstartsignatures
\pysiglinewithargsret
{\sphinxbfcode{\sphinxupquote{create\_token}}}
{\sphinxparam{\DUrole{n}{name}}\sphinxparamcomma \sphinxparam{\DUrole{n}{token\_class}}\sphinxparamcomma \sphinxparam{\DUrole{n}{analog}}\sphinxparamcomma \sphinxparam{\DUrole{n}{is\_pred}\DUrole{o}{=}\DUrole{default_value}{None}}}
{}
\pysigstopsignatures
\sphinxAtStartPar
Create a token object and add it to the name/dict.
\begin{quote}\begin{description}
\sphinxlineitem{Parameters}\begin{itemize}
\item {} 
\sphinxAtStartPar
\sphinxstyleliteralstrong{\sphinxupquote{name}} (\sphinxstyleliteralemphasis{\sphinxupquote{str}}) \textendash{} The name of the token.

\item {} 
\sphinxAtStartPar
\sphinxstyleliteralstrong{\sphinxupquote{token\_class}} ({\hyperref[\detokenize{nodes:nodes.nodeBuilder.Token}]{\sphinxcrossref{\sphinxstyleliteralemphasis{\sphinxupquote{Token}}}}}) \textendash{} The class of the token.

\item {} 
\sphinxAtStartPar
\sphinxstyleliteralstrong{\sphinxupquote{analog}} (\sphinxstyleliteralemphasis{\sphinxupquote{int}}) \textendash{} The analog of the token.

\item {} 
\sphinxAtStartPar
\sphinxstyleliteralstrong{\sphinxupquote{is\_pred}} (\sphinxstyleliteralemphasis{\sphinxupquote{bool}}) \textendash{} Whether the token is a predicate.

\end{itemize}

\end{description}\end{quote}

\end{fulllineitems}

\index{get\_nodes() (nodes.nodeBuilder.Build\_set method)@\spxentry{get\_nodes()}\spxextra{nodes.nodeBuilder.Build\_set method}}

\begin{fulllineitems}
\phantomsection\label{\detokenize{nodes:nodes.nodeBuilder.Build_set.get_nodes}}
\pysigstartsignatures
\pysiglinewithargsret
{\sphinxbfcode{\sphinxupquote{get\_nodes}}}
{}
{}
\pysigstopsignatures
\sphinxAtStartPar
Gets lists of unique tokens by type.

\end{fulllineitems}

\index{id\_tokens() (nodes.nodeBuilder.Build\_set method)@\spxentry{id\_tokens()}\spxextra{nodes.nodeBuilder.Build\_set method}}

\begin{fulllineitems}
\phantomsection\label{\detokenize{nodes:nodes.nodeBuilder.Build_set.id_tokens}}
\pysigstartsignatures
\pysiglinewithargsret
{\sphinxbfcode{\sphinxupquote{id\_tokens}}}
{}
{}
\pysigstopsignatures
\sphinxAtStartPar
Assign each token an ID, unique for the set.

\end{fulllineitems}


\end{fulllineitems}

\index{Node (class in nodes.nodeBuilder)@\spxentry{Node}\spxextra{class in nodes.nodeBuilder}}

\begin{fulllineitems}
\phantomsection\label{\detokenize{nodes:nodes.nodeBuilder.Node}}
\pysigstartsignatures
\pysiglinewithargsret
{\sphinxbfcode{\sphinxupquote{\DUrole{k}{class}\DUrole{w}{ }}}\sphinxcode{\sphinxupquote{nodes.nodeBuilder.}}\sphinxbfcode{\sphinxupquote{Node}}}
{\sphinxparam{\DUrole{n}{name}}}
{}
\pysigstopsignatures
\sphinxAtStartPar
Bases: \sphinxcode{\sphinxupquote{object}}

\sphinxAtStartPar
An intermediate class for representing a node in the network.
\index{name (nodes.nodeBuilder.Node attribute)@\spxentry{name}\spxextra{nodes.nodeBuilder.Node attribute}}

\begin{fulllineitems}
\phantomsection\label{\detokenize{nodes:nodes.nodeBuilder.Node.name}}
\pysigstartsignatures
\pysigline
{\sphinxbfcode{\sphinxupquote{name}}}
\pysigstopsignatures
\sphinxAtStartPar
The name of the node.
\begin{quote}\begin{description}
\sphinxlineitem{Type}
\sphinxAtStartPar
str

\end{description}\end{quote}

\end{fulllineitems}

\index{features (nodes.nodeBuilder.Node attribute)@\spxentry{features}\spxextra{nodes.nodeBuilder.Node attribute}}

\begin{fulllineitems}
\phantomsection\label{\detokenize{nodes:nodes.nodeBuilder.Node.features}}
\pysigstartsignatures
\pysigline
{\sphinxbfcode{\sphinxupquote{features}}}
\pysigstopsignatures
\sphinxAtStartPar
A list of features for the node.
\begin{quote}\begin{description}
\sphinxlineitem{Type}
\sphinxAtStartPar
list

\end{description}\end{quote}

\end{fulllineitems}

\index{ID (nodes.nodeBuilder.Node attribute)@\spxentry{ID}\spxextra{nodes.nodeBuilder.Node attribute}}

\begin{fulllineitems}
\phantomsection\label{\detokenize{nodes:nodes.nodeBuilder.Node.ID}}
\pysigstartsignatures
\pysigline
{\sphinxbfcode{\sphinxupquote{ID}}}
\pysigstopsignatures
\sphinxAtStartPar
The ID of the node.
\begin{quote}\begin{description}
\sphinxlineitem{Type}
\sphinxAtStartPar
int

\end{description}\end{quote}

\end{fulllineitems}

\index{set() (nodes.nodeBuilder.Node method)@\spxentry{set()}\spxextra{nodes.nodeBuilder.Node method}}

\begin{fulllineitems}
\phantomsection\label{\detokenize{nodes:nodes.nodeBuilder.Node.set}}
\pysigstartsignatures
\pysiglinewithargsret
{\sphinxbfcode{\sphinxupquote{set}}}
{\sphinxparam{\DUrole{n}{feature}}\sphinxparamcomma \sphinxparam{\DUrole{n}{value}\DUrole{p}{:}\DUrole{w}{ }\DUrole{n}{float}}}
{}
\pysigstopsignatures
\sphinxAtStartPar
Set the feature of the node.
\begin{quote}\begin{description}
\sphinxlineitem{Parameters}\begin{itemize}
\item {} 
\sphinxAtStartPar
\sphinxstyleliteralstrong{\sphinxupquote{feature}} (\sphinxstyleliteralemphasis{\sphinxupquote{str}}) \textendash{} The feature to set.

\item {} 
\sphinxAtStartPar
\sphinxstyleliteralstrong{\sphinxupquote{value}} (\sphinxstyleliteralemphasis{\sphinxupquote{float}}) \textendash{} The value to set the feature to.

\end{itemize}

\end{description}\end{quote}

\end{fulllineitems}

\index{set\_ID() (nodes.nodeBuilder.Node method)@\spxentry{set\_ID()}\spxextra{nodes.nodeBuilder.Node method}}

\begin{fulllineitems}
\phantomsection\label{\detokenize{nodes:nodes.nodeBuilder.Node.set_ID}}
\pysigstartsignatures
\pysiglinewithargsret
{\sphinxbfcode{\sphinxupquote{set\_ID}}}
{\sphinxparam{\DUrole{n}{ID}}}
{}
\pysigstopsignatures
\sphinxAtStartPar
Set the ID of the node.
\begin{quote}\begin{description}
\sphinxlineitem{Parameters}
\sphinxAtStartPar
\sphinxstyleliteralstrong{\sphinxupquote{ID}} (\sphinxstyleliteralemphasis{\sphinxupquote{int}}) \textendash{} The ID to set the node to.

\end{description}\end{quote}

\end{fulllineitems}


\end{fulllineitems}

\index{PO (class in nodes.nodeBuilder)@\spxentry{PO}\spxextra{class in nodes.nodeBuilder}}

\begin{fulllineitems}
\phantomsection\label{\detokenize{nodes:nodes.nodeBuilder.PO}}
\pysigstartsignatures
\pysiglinewithargsret
{\sphinxbfcode{\sphinxupquote{\DUrole{k}{class}\DUrole{w}{ }}}\sphinxcode{\sphinxupquote{nodes.nodeBuilder.}}\sphinxbfcode{\sphinxupquote{PO}}}
{\sphinxparam{\DUrole{n}{name}}\sphinxparamcomma \sphinxparam{\DUrole{n}{set}}\sphinxparamcomma \sphinxparam{\DUrole{n}{analog}}\sphinxparamcomma \sphinxparam{\DUrole{n}{is\_pred}\DUrole{p}{:}\DUrole{w}{ }\DUrole{n}{bool}}}
{}
\pysigstopsignatures
\sphinxAtStartPar
Bases: {\hyperref[\detokenize{nodes:nodes.nodeBuilder.Token}]{\sphinxcrossref{\sphinxcode{\sphinxupquote{Token}}}}}

\sphinxAtStartPar
An intermediate class for representing a PO node.
\index{name (nodes.nodeBuilder.PO attribute)@\spxentry{name}\spxextra{nodes.nodeBuilder.PO attribute}}

\begin{fulllineitems}
\phantomsection\label{\detokenize{nodes:nodes.nodeBuilder.PO.name}}
\pysigstartsignatures
\pysigline
{\sphinxbfcode{\sphinxupquote{name}}}
\pysigstopsignatures
\sphinxAtStartPar
The name of the PO.
\begin{quote}\begin{description}
\sphinxlineitem{Type}
\sphinxAtStartPar
str

\end{description}\end{quote}

\end{fulllineitems}

\index{features (nodes.nodeBuilder.PO attribute)@\spxentry{features}\spxextra{nodes.nodeBuilder.PO attribute}}

\begin{fulllineitems}
\phantomsection\label{\detokenize{nodes:nodes.nodeBuilder.PO.features}}
\pysigstartsignatures
\pysigline
{\sphinxbfcode{\sphinxupquote{features}}}
\pysigstopsignatures
\sphinxAtStartPar
A list of features for the PO, indexed by TF.
\begin{quote}\begin{description}
\sphinxlineitem{Type}
\sphinxAtStartPar
list

\end{description}\end{quote}

\end{fulllineitems}

\index{ID (nodes.nodeBuilder.PO attribute)@\spxentry{ID}\spxextra{nodes.nodeBuilder.PO attribute}}

\begin{fulllineitems}
\phantomsection\label{\detokenize{nodes:nodes.nodeBuilder.PO.ID}}
\pysigstartsignatures
\pysigline
{\sphinxbfcode{\sphinxupquote{ID}}}
\pysigstopsignatures
\sphinxAtStartPar
The ID of the PO.
\begin{quote}\begin{description}
\sphinxlineitem{Type}
\sphinxAtStartPar
int

\end{description}\end{quote}

\end{fulllineitems}

\index{children (nodes.nodeBuilder.PO attribute)@\spxentry{children}\spxextra{nodes.nodeBuilder.PO attribute}}

\begin{fulllineitems}
\phantomsection\label{\detokenize{nodes:nodes.nodeBuilder.PO.children}}
\pysigstartsignatures
\pysigline
{\sphinxbfcode{\sphinxupquote{children}}}
\pysigstopsignatures
\sphinxAtStartPar
A list of children of the PO, for use in building the connections matrix.
\begin{quote}\begin{description}
\sphinxlineitem{Type}
\sphinxAtStartPar
list

\end{description}\end{quote}

\end{fulllineitems}


\end{fulllineitems}

\index{Prop (class in nodes.nodeBuilder)@\spxentry{Prop}\spxextra{class in nodes.nodeBuilder}}

\begin{fulllineitems}
\phantomsection\label{\detokenize{nodes:nodes.nodeBuilder.Prop}}
\pysigstartsignatures
\pysiglinewithargsret
{\sphinxbfcode{\sphinxupquote{\DUrole{k}{class}\DUrole{w}{ }}}\sphinxcode{\sphinxupquote{nodes.nodeBuilder.}}\sphinxbfcode{\sphinxupquote{Prop}}}
{\sphinxparam{\DUrole{n}{name}}\sphinxparamcomma \sphinxparam{\DUrole{n}{set}}\sphinxparamcomma \sphinxparam{\DUrole{n}{analog}}}
{}
\pysigstopsignatures
\sphinxAtStartPar
Bases: {\hyperref[\detokenize{nodes:nodes.nodeBuilder.Token}]{\sphinxcrossref{\sphinxcode{\sphinxupquote{Token}}}}}

\sphinxAtStartPar
An intermediate class for representing a Prop node.
\index{name (nodes.nodeBuilder.Prop attribute)@\spxentry{name}\spxextra{nodes.nodeBuilder.Prop attribute}}

\begin{fulllineitems}
\phantomsection\label{\detokenize{nodes:nodes.nodeBuilder.Prop.name}}
\pysigstartsignatures
\pysigline
{\sphinxbfcode{\sphinxupquote{name}}}
\pysigstopsignatures
\sphinxAtStartPar
The name of the Prop.
\begin{quote}\begin{description}
\sphinxlineitem{Type}
\sphinxAtStartPar
str

\end{description}\end{quote}

\end{fulllineitems}

\index{features (nodes.nodeBuilder.Prop attribute)@\spxentry{features}\spxextra{nodes.nodeBuilder.Prop attribute}}

\begin{fulllineitems}
\phantomsection\label{\detokenize{nodes:nodes.nodeBuilder.Prop.features}}
\pysigstartsignatures
\pysigline
{\sphinxbfcode{\sphinxupquote{features}}}
\pysigstopsignatures
\sphinxAtStartPar
A list of features for the Prop, indexed by TF.
\begin{quote}\begin{description}
\sphinxlineitem{Type}
\sphinxAtStartPar
list

\end{description}\end{quote}

\end{fulllineitems}

\index{ID (nodes.nodeBuilder.Prop attribute)@\spxentry{ID}\spxextra{nodes.nodeBuilder.Prop attribute}}

\begin{fulllineitems}
\phantomsection\label{\detokenize{nodes:nodes.nodeBuilder.Prop.ID}}
\pysigstartsignatures
\pysigline
{\sphinxbfcode{\sphinxupquote{ID}}}
\pysigstopsignatures
\sphinxAtStartPar
The ID of the Prop.
\begin{quote}\begin{description}
\sphinxlineitem{Type}
\sphinxAtStartPar
int

\end{description}\end{quote}

\end{fulllineitems}

\index{children (nodes.nodeBuilder.Prop attribute)@\spxentry{children}\spxextra{nodes.nodeBuilder.Prop attribute}}

\begin{fulllineitems}
\phantomsection\label{\detokenize{nodes:nodes.nodeBuilder.Prop.children}}
\pysigstartsignatures
\pysigline
{\sphinxbfcode{\sphinxupquote{children}}}
\pysigstopsignatures
\sphinxAtStartPar
A list of children of the Prop, for use in building the connections matrix.
\begin{quote}\begin{description}
\sphinxlineitem{Type}
\sphinxAtStartPar
list

\end{description}\end{quote}

\end{fulllineitems}


\end{fulllineitems}

\index{RB (class in nodes.nodeBuilder)@\spxentry{RB}\spxextra{class in nodes.nodeBuilder}}

\begin{fulllineitems}
\phantomsection\label{\detokenize{nodes:nodes.nodeBuilder.RB}}
\pysigstartsignatures
\pysiglinewithargsret
{\sphinxbfcode{\sphinxupquote{\DUrole{k}{class}\DUrole{w}{ }}}\sphinxcode{\sphinxupquote{nodes.nodeBuilder.}}\sphinxbfcode{\sphinxupquote{RB}}}
{\sphinxparam{\DUrole{n}{name}}\sphinxparamcomma \sphinxparam{\DUrole{n}{set}}\sphinxparamcomma \sphinxparam{\DUrole{n}{analog}}}
{}
\pysigstopsignatures
\sphinxAtStartPar
Bases: {\hyperref[\detokenize{nodes:nodes.nodeBuilder.Token}]{\sphinxcrossref{\sphinxcode{\sphinxupquote{Token}}}}}

\sphinxAtStartPar
An intermediate class for representing a RB node.
\index{name (nodes.nodeBuilder.RB attribute)@\spxentry{name}\spxextra{nodes.nodeBuilder.RB attribute}}

\begin{fulllineitems}
\phantomsection\label{\detokenize{nodes:nodes.nodeBuilder.RB.name}}
\pysigstartsignatures
\pysigline
{\sphinxbfcode{\sphinxupquote{name}}}
\pysigstopsignatures
\sphinxAtStartPar
The name of the RB.
\begin{quote}\begin{description}
\sphinxlineitem{Type}
\sphinxAtStartPar
str

\end{description}\end{quote}

\end{fulllineitems}

\index{features (nodes.nodeBuilder.RB attribute)@\spxentry{features}\spxextra{nodes.nodeBuilder.RB attribute}}

\begin{fulllineitems}
\phantomsection\label{\detokenize{nodes:nodes.nodeBuilder.RB.features}}
\pysigstartsignatures
\pysigline
{\sphinxbfcode{\sphinxupquote{features}}}
\pysigstopsignatures
\sphinxAtStartPar
A list of features for the RB, indexed by TF.
\begin{quote}\begin{description}
\sphinxlineitem{Type}
\sphinxAtStartPar
list

\end{description}\end{quote}

\end{fulllineitems}

\index{ID (nodes.nodeBuilder.RB attribute)@\spxentry{ID}\spxextra{nodes.nodeBuilder.RB attribute}}

\begin{fulllineitems}
\phantomsection\label{\detokenize{nodes:nodes.nodeBuilder.RB.ID}}
\pysigstartsignatures
\pysigline
{\sphinxbfcode{\sphinxupquote{ID}}}
\pysigstopsignatures
\sphinxAtStartPar
The ID of the RB.
\begin{quote}\begin{description}
\sphinxlineitem{Type}
\sphinxAtStartPar
int

\end{description}\end{quote}

\end{fulllineitems}

\index{children (nodes.nodeBuilder.RB attribute)@\spxentry{children}\spxextra{nodes.nodeBuilder.RB attribute}}

\begin{fulllineitems}
\phantomsection\label{\detokenize{nodes:nodes.nodeBuilder.RB.children}}
\pysigstartsignatures
\pysigline
{\sphinxbfcode{\sphinxupquote{children}}}
\pysigstopsignatures
\sphinxAtStartPar
A list of children of the RB, for use in building the connections matrix.
\begin{quote}\begin{description}
\sphinxlineitem{Type}
\sphinxAtStartPar
list

\end{description}\end{quote}

\end{fulllineitems}


\end{fulllineitems}

\index{Sem\_set (class in nodes.nodeBuilder)@\spxentry{Sem\_set}\spxextra{class in nodes.nodeBuilder}}

\begin{fulllineitems}
\phantomsection\label{\detokenize{nodes:nodes.nodeBuilder.Sem_set}}
\pysigstartsignatures
\pysiglinewithargsret
{\sphinxbfcode{\sphinxupquote{\DUrole{k}{class}\DUrole{w}{ }}}\sphinxcode{\sphinxupquote{nodes.nodeBuilder.}}\sphinxbfcode{\sphinxupquote{Sem\_set}}}
{\sphinxparam{\DUrole{n}{sems}\DUrole{p}{:}\DUrole{w}{ }\DUrole{n}{list\DUrole{p}{{[}}{\hyperref[\detokenize{nodes:nodes.nodeBuilder.Semantic}]{\sphinxcrossref{Semantic}}}\DUrole{p}{{]}}}}\sphinxparamcomma \sphinxparam{\DUrole{n}{name\_dict}\DUrole{p}{:}\DUrole{w}{ }\DUrole{n}{dict\DUrole{p}{{[}}str\DUrole{p}{,}\DUrole{w}{ }{\hyperref[\detokenize{nodes:nodes.nodeBuilder.Semantic}]{\sphinxcrossref{Semantic}}}\DUrole{p}{{]}}}}\sphinxparamcomma \sphinxparam{\DUrole{n}{id\_dict}\DUrole{p}{:}\DUrole{w}{ }\DUrole{n}{dict\DUrole{p}{{[}}int\DUrole{p}{,}\DUrole{w}{ }{\hyperref[\detokenize{nodes:nodes.nodeBuilder.Semantic}]{\sphinxcrossref{Semantic}}}\DUrole{p}{{]}}}}}
{}
\pysigstopsignatures
\sphinxAtStartPar
Bases: \sphinxcode{\sphinxupquote{object}}

\sphinxAtStartPar
An intermediate class for representing a set of semantics.
\index{sems (nodes.nodeBuilder.Sem\_set attribute)@\spxentry{sems}\spxextra{nodes.nodeBuilder.Sem\_set attribute}}

\begin{fulllineitems}
\phantomsection\label{\detokenize{nodes:nodes.nodeBuilder.Sem_set.sems}}
\pysigstartsignatures
\pysigline
{\sphinxbfcode{\sphinxupquote{sems}}}
\pysigstopsignatures
\sphinxAtStartPar
A list of semantics.
\begin{quote}\begin{description}
\sphinxlineitem{Type}
\sphinxAtStartPar
list

\end{description}\end{quote}

\end{fulllineitems}

\index{name\_dict (nodes.nodeBuilder.Sem\_set attribute)@\spxentry{name\_dict}\spxextra{nodes.nodeBuilder.Sem\_set attribute}}

\begin{fulllineitems}
\phantomsection\label{\detokenize{nodes:nodes.nodeBuilder.Sem_set.name_dict}}
\pysigstartsignatures
\pysigline
{\sphinxbfcode{\sphinxupquote{name\_dict}}}
\pysigstopsignatures
\sphinxAtStartPar
A dictionary of semantics, mapping name to semantic in sems.
\begin{quote}\begin{description}
\sphinxlineitem{Type}
\sphinxAtStartPar
dict

\end{description}\end{quote}

\end{fulllineitems}

\index{id\_dict (nodes.nodeBuilder.Sem\_set attribute)@\spxentry{id\_dict}\spxextra{nodes.nodeBuilder.Sem\_set attribute}}

\begin{fulllineitems}
\phantomsection\label{\detokenize{nodes:nodes.nodeBuilder.Sem_set.id_dict}}
\pysigstartsignatures
\pysigline
{\sphinxbfcode{\sphinxupquote{id\_dict}}}
\pysigstopsignatures
\sphinxAtStartPar
A dictionary of semantics, mapping ID to semantic in sems.
\begin{quote}\begin{description}
\sphinxlineitem{Type}
\sphinxAtStartPar
dict

\end{description}\end{quote}

\end{fulllineitems}

\index{num\_sems (nodes.nodeBuilder.Sem\_set attribute)@\spxentry{num\_sems}\spxextra{nodes.nodeBuilder.Sem\_set attribute}}

\begin{fulllineitems}
\phantomsection\label{\detokenize{nodes:nodes.nodeBuilder.Sem_set.num_sems}}
\pysigstartsignatures
\pysigline
{\sphinxbfcode{\sphinxupquote{num\_sems}}}
\pysigstopsignatures
\sphinxAtStartPar
The number of semantics in the set.
\begin{quote}\begin{description}
\sphinxlineitem{Type}
\sphinxAtStartPar
int

\end{description}\end{quote}

\end{fulllineitems}

\index{connections (nodes.nodeBuilder.Sem\_set attribute)@\spxentry{connections}\spxextra{nodes.nodeBuilder.Sem\_set attribute}}

\begin{fulllineitems}
\phantomsection\label{\detokenize{nodes:nodes.nodeBuilder.Sem_set.connections}}
\pysigstartsignatures
\pysigline
{\sphinxbfcode{\sphinxupquote{connections}}}
\pysigstopsignatures
\sphinxAtStartPar
A matrix of connections between semantics.
\begin{quote}\begin{description}
\sphinxlineitem{Type}
\sphinxAtStartPar
np.ndarray

\end{description}\end{quote}

\end{fulllineitems}

\index{get\_sem() (nodes.nodeBuilder.Sem\_set method)@\spxentry{get\_sem()}\spxextra{nodes.nodeBuilder.Sem\_set method}}

\begin{fulllineitems}
\phantomsection\label{\detokenize{nodes:nodes.nodeBuilder.Sem_set.get_sem}}
\pysigstartsignatures
\pysiglinewithargsret
{\sphinxbfcode{\sphinxupquote{get\_sem}}}
{\sphinxparam{\DUrole{n}{name}}}
{}
\pysigstopsignatures
\sphinxAtStartPar
Get a semantic from the semantic set by name.
\begin{quote}\begin{description}
\sphinxlineitem{Parameters}
\sphinxAtStartPar
\sphinxstyleliteralstrong{\sphinxupquote{name}} (\sphinxstyleliteralemphasis{\sphinxupquote{str}}) \textendash{} The name of the semantic.

\end{description}\end{quote}

\end{fulllineitems}

\index{get\_sem\_by\_id() (nodes.nodeBuilder.Sem\_set method)@\spxentry{get\_sem\_by\_id()}\spxextra{nodes.nodeBuilder.Sem\_set method}}

\begin{fulllineitems}
\phantomsection\label{\detokenize{nodes:nodes.nodeBuilder.Sem_set.get_sem_by_id}}
\pysigstartsignatures
\pysiglinewithargsret
{\sphinxbfcode{\sphinxupquote{get\_sem\_by\_id}}}
{\sphinxparam{\DUrole{n}{ID}}}
{}
\pysigstopsignatures
\sphinxAtStartPar
Get a semantic from the semantic set by ID.
\begin{quote}\begin{description}
\sphinxlineitem{Parameters}
\sphinxAtStartPar
\sphinxstyleliteralstrong{\sphinxupquote{ID}} (\sphinxstyleliteralemphasis{\sphinxupquote{int}}) \textendash{} The ID of the semantic.

\end{description}\end{quote}

\end{fulllineitems}

\index{tensorise() (nodes.nodeBuilder.Sem\_set method)@\spxentry{tensorise()}\spxextra{nodes.nodeBuilder.Sem\_set method}}

\begin{fulllineitems}
\phantomsection\label{\detokenize{nodes:nodes.nodeBuilder.Sem_set.tensorise}}
\pysigstartsignatures
\pysiglinewithargsret
{\sphinxbfcode{\sphinxupquote{tensorise}}}
{}
{}
\pysigstopsignatures
\sphinxAtStartPar
Tensorise the semantic set, creating a tensor of semantics, and a tensor of connections between semantics.

\end{fulllineitems}


\end{fulllineitems}

\index{Semantic (class in nodes.nodeBuilder)@\spxentry{Semantic}\spxextra{class in nodes.nodeBuilder}}

\begin{fulllineitems}
\phantomsection\label{\detokenize{nodes:nodes.nodeBuilder.Semantic}}
\pysigstartsignatures
\pysiglinewithargsret
{\sphinxbfcode{\sphinxupquote{\DUrole{k}{class}\DUrole{w}{ }}}\sphinxcode{\sphinxupquote{nodes.nodeBuilder.}}\sphinxbfcode{\sphinxupquote{Semantic}}}
{\sphinxparam{\DUrole{n}{name}}}
{}
\pysigstopsignatures
\sphinxAtStartPar
Bases: {\hyperref[\detokenize{nodes:nodes.nodeBuilder.Node}]{\sphinxcrossref{\sphinxcode{\sphinxupquote{Node}}}}}

\sphinxAtStartPar
An intermediate class for representing a semantic node.
\index{name (nodes.nodeBuilder.Semantic attribute)@\spxentry{name}\spxextra{nodes.nodeBuilder.Semantic attribute}}

\begin{fulllineitems}
\phantomsection\label{\detokenize{nodes:nodes.nodeBuilder.Semantic.name}}
\pysigstartsignatures
\pysigline
{\sphinxbfcode{\sphinxupquote{name}}}
\pysigstopsignatures
\sphinxAtStartPar
The name of the semantic.
\begin{quote}\begin{description}
\sphinxlineitem{Type}
\sphinxAtStartPar
str

\end{description}\end{quote}

\end{fulllineitems}

\index{features (nodes.nodeBuilder.Semantic attribute)@\spxentry{features}\spxextra{nodes.nodeBuilder.Semantic attribute}}

\begin{fulllineitems}
\phantomsection\label{\detokenize{nodes:nodes.nodeBuilder.Semantic.features}}
\pysigstartsignatures
\pysigline
{\sphinxbfcode{\sphinxupquote{features}}}
\pysigstopsignatures
\sphinxAtStartPar
A list of features for the semantic, indexed by SF.
\begin{quote}\begin{description}
\sphinxlineitem{Type}
\sphinxAtStartPar
list

\end{description}\end{quote}

\end{fulllineitems}

\index{ID (nodes.nodeBuilder.Semantic attribute)@\spxentry{ID}\spxextra{nodes.nodeBuilder.Semantic attribute}}

\begin{fulllineitems}
\phantomsection\label{\detokenize{nodes:nodes.nodeBuilder.Semantic.ID}}
\pysigstartsignatures
\pysigline
{\sphinxbfcode{\sphinxupquote{ID}}}
\pysigstopsignatures
\sphinxAtStartPar
The ID of the semantic.
\begin{quote}\begin{description}
\sphinxlineitem{Type}
\sphinxAtStartPar
int

\end{description}\end{quote}

\end{fulllineitems}

\index{floatate\_features() (nodes.nodeBuilder.Semantic method)@\spxentry{floatate\_features()}\spxextra{nodes.nodeBuilder.Semantic method}}

\begin{fulllineitems}
\phantomsection\label{\detokenize{nodes:nodes.nodeBuilder.Semantic.floatate_features}}
\pysigstartsignatures
\pysiglinewithargsret
{\sphinxbfcode{\sphinxupquote{floatate\_features}}}
{}
{}
\pysigstopsignatures
\sphinxAtStartPar
Convert semantic features to floats, required for tensorisation.

\end{fulllineitems}

\index{initialise\_defaults() (nodes.nodeBuilder.Semantic method)@\spxentry{initialise\_defaults()}\spxextra{nodes.nodeBuilder.Semantic method}}

\begin{fulllineitems}
\phantomsection\label{\detokenize{nodes:nodes.nodeBuilder.Semantic.initialise_defaults}}
\pysigstartsignatures
\pysiglinewithargsret
{\sphinxbfcode{\sphinxupquote{initialise\_defaults}}}
{}
{}
\pysigstopsignatures
\sphinxAtStartPar
Initialise the default features for the semantic.

\end{fulllineitems}


\end{fulllineitems}

\index{Token (class in nodes.nodeBuilder)@\spxentry{Token}\spxextra{class in nodes.nodeBuilder}}

\begin{fulllineitems}
\phantomsection\label{\detokenize{nodes:nodes.nodeBuilder.Token}}
\pysigstartsignatures
\pysiglinewithargsret
{\sphinxbfcode{\sphinxupquote{\DUrole{k}{class}\DUrole{w}{ }}}\sphinxcode{\sphinxupquote{nodes.nodeBuilder.}}\sphinxbfcode{\sphinxupquote{Token}}}
{\sphinxparam{\DUrole{n}{name}}\sphinxparamcomma \sphinxparam{\DUrole{n}{set}\DUrole{p}{:}\DUrole{w}{ }\DUrole{n}{{\hyperref[\detokenize{nodes:nodes.nodeEnums.Set}]{\sphinxcrossref{Set}}}}}\sphinxparamcomma \sphinxparam{\DUrole{n}{analog}\DUrole{p}{:}\DUrole{w}{ }\DUrole{n}{int}}}
{}
\pysigstopsignatures
\sphinxAtStartPar
Bases: {\hyperref[\detokenize{nodes:nodes.nodeBuilder.Node}]{\sphinxcrossref{\sphinxcode{\sphinxupquote{Node}}}}}

\sphinxAtStartPar
An intermediate class for representing a token node.
\index{name (nodes.nodeBuilder.Token attribute)@\spxentry{name}\spxextra{nodes.nodeBuilder.Token attribute}}

\begin{fulllineitems}
\phantomsection\label{\detokenize{nodes:nodes.nodeBuilder.Token.name}}
\pysigstartsignatures
\pysigline
{\sphinxbfcode{\sphinxupquote{name}}}
\pysigstopsignatures
\sphinxAtStartPar
The name of the token.
\begin{quote}\begin{description}
\sphinxlineitem{Type}
\sphinxAtStartPar
str

\end{description}\end{quote}

\end{fulllineitems}

\index{features (nodes.nodeBuilder.Token attribute)@\spxentry{features}\spxextra{nodes.nodeBuilder.Token attribute}}

\begin{fulllineitems}
\phantomsection\label{\detokenize{nodes:nodes.nodeBuilder.Token.features}}
\pysigstartsignatures
\pysigline
{\sphinxbfcode{\sphinxupquote{features}}}
\pysigstopsignatures
\sphinxAtStartPar
A list of features for the token, indexed by TF.
\begin{quote}\begin{description}
\sphinxlineitem{Type}
\sphinxAtStartPar
list

\end{description}\end{quote}

\end{fulllineitems}

\index{ID (nodes.nodeBuilder.Token attribute)@\spxentry{ID}\spxextra{nodes.nodeBuilder.Token attribute}}

\begin{fulllineitems}
\phantomsection\label{\detokenize{nodes:nodes.nodeBuilder.Token.ID}}
\pysigstartsignatures
\pysigline
{\sphinxbfcode{\sphinxupquote{ID}}}
\pysigstopsignatures
\sphinxAtStartPar
The ID of the token.
\begin{quote}\begin{description}
\sphinxlineitem{Type}
\sphinxAtStartPar
int

\end{description}\end{quote}

\end{fulllineitems}

\index{children (nodes.nodeBuilder.Token attribute)@\spxentry{children}\spxextra{nodes.nodeBuilder.Token attribute}}

\begin{fulllineitems}
\phantomsection\label{\detokenize{nodes:nodes.nodeBuilder.Token.children}}
\pysigstartsignatures
\pysigline
{\sphinxbfcode{\sphinxupquote{children}}}
\pysigstopsignatures
\sphinxAtStartPar
A list of children of the token, for use in building the connections matrix.
\begin{quote}\begin{description}
\sphinxlineitem{Type}
\sphinxAtStartPar
list

\end{description}\end{quote}

\end{fulllineitems}

\index{floatate\_features() (nodes.nodeBuilder.Token method)@\spxentry{floatate\_features()}\spxextra{nodes.nodeBuilder.Token method}}

\begin{fulllineitems}
\phantomsection\label{\detokenize{nodes:nodes.nodeBuilder.Token.floatate_features}}
\pysigstartsignatures
\pysiglinewithargsret
{\sphinxbfcode{\sphinxupquote{floatate\_features}}}
{}
{}
\pysigstopsignatures
\sphinxAtStartPar
Convert token features to floats, required for tensorisation.

\end{fulllineitems}

\index{initialise\_defaults() (nodes.nodeBuilder.Token method)@\spxentry{initialise\_defaults()}\spxextra{nodes.nodeBuilder.Token method}}

\begin{fulllineitems}
\phantomsection\label{\detokenize{nodes:nodes.nodeBuilder.Token.initialise_defaults}}
\pysigstartsignatures
\pysiglinewithargsret
{\sphinxbfcode{\sphinxupquote{initialise\_defaults}}}
{}
{}
\pysigstopsignatures
\sphinxAtStartPar
Initialise the default features for the token.

\end{fulllineitems}


\end{fulllineitems}

\index{Token\_set (class in nodes.nodeBuilder)@\spxentry{Token\_set}\spxextra{class in nodes.nodeBuilder}}

\begin{fulllineitems}
\phantomsection\label{\detokenize{nodes:nodes.nodeBuilder.Token_set}}
\pysigstartsignatures
\pysiglinewithargsret
{\sphinxbfcode{\sphinxupquote{\DUrole{k}{class}\DUrole{w}{ }}}\sphinxcode{\sphinxupquote{nodes.nodeBuilder.}}\sphinxbfcode{\sphinxupquote{Token\_set}}}
{\sphinxparam{\DUrole{n}{set}\DUrole{p}{:}\DUrole{w}{ }\DUrole{n}{{\hyperref[\detokenize{nodes:nodes.nodeEnums.Set}]{\sphinxcrossref{Set}}}}}\sphinxparamcomma \sphinxparam{\DUrole{n}{tokens}\DUrole{p}{:}\DUrole{w}{ }\DUrole{n}{dict\DUrole{p}{{[}}{\hyperref[\detokenize{nodes:nodes.nodeEnums.Type}]{\sphinxcrossref{Type}}}\DUrole{p}{,}\DUrole{w}{ }list\DUrole{p}{{[}}{\hyperref[\detokenize{nodes:nodes.nodeBuilder.Token}]{\sphinxcrossref{Token}}}\DUrole{p}{{]}}\DUrole{p}{{]}}}}\sphinxparamcomma \sphinxparam{\DUrole{n}{name\_dict}\DUrole{p}{:}\DUrole{w}{ }\DUrole{n}{dict\DUrole{p}{{[}}str\DUrole{p}{,}\DUrole{w}{ }{\hyperref[\detokenize{nodes:nodes.nodeBuilder.Token}]{\sphinxcrossref{Token}}}\DUrole{p}{{]}}}}\sphinxparamcomma \sphinxparam{\DUrole{n}{id\_dict}\DUrole{p}{:}\DUrole{w}{ }\DUrole{n}{dict\DUrole{p}{{[}}int\DUrole{p}{,}\DUrole{w}{ }{\hyperref[\detokenize{nodes:nodes.nodeBuilder.Token}]{\sphinxcrossref{Token}}}\DUrole{p}{{]}}}}}
{}
\pysigstopsignatures
\sphinxAtStartPar
Bases: \sphinxcode{\sphinxupquote{object}}

\sphinxAtStartPar
An intermediate class for representing a set of tokens.
\index{set (nodes.nodeBuilder.Token\_set attribute)@\spxentry{set}\spxextra{nodes.nodeBuilder.Token\_set attribute}}

\begin{fulllineitems}
\phantomsection\label{\detokenize{nodes:nodes.nodeBuilder.Token_set.set}}
\pysigstartsignatures
\pysigline
{\sphinxbfcode{\sphinxupquote{set}}}
\pysigstopsignatures
\sphinxAtStartPar
The set of the tokens.
\begin{quote}\begin{description}
\sphinxlineitem{Type}
\sphinxAtStartPar
{\hyperref[\detokenize{nodes:nodes.nodeEnums.Set}]{\sphinxcrossref{Set}}}

\end{description}\end{quote}

\end{fulllineitems}

\index{tokens (nodes.nodeBuilder.Token\_set attribute)@\spxentry{tokens}\spxextra{nodes.nodeBuilder.Token\_set attribute}}

\begin{fulllineitems}
\phantomsection\label{\detokenize{nodes:nodes.nodeBuilder.Token_set.tokens}}
\pysigstartsignatures
\pysigline
{\sphinxbfcode{\sphinxupquote{tokens}}}
\pysigstopsignatures
\sphinxAtStartPar
A dictionary of tokens, mapping type to list of tokens.
\begin{quote}\begin{description}
\sphinxlineitem{Type}
\sphinxAtStartPar
dict

\end{description}\end{quote}

\end{fulllineitems}

\index{name\_dict (nodes.nodeBuilder.Token\_set attribute)@\spxentry{name\_dict}\spxextra{nodes.nodeBuilder.Token\_set attribute}}

\begin{fulllineitems}
\phantomsection\label{\detokenize{nodes:nodes.nodeBuilder.Token_set.name_dict}}
\pysigstartsignatures
\pysigline
{\sphinxbfcode{\sphinxupquote{name\_dict}}}
\pysigstopsignatures
\sphinxAtStartPar
A dictionary of tokens, mapping name to token in tokens.
\begin{quote}\begin{description}
\sphinxlineitem{Type}
\sphinxAtStartPar
dict

\end{description}\end{quote}

\end{fulllineitems}

\index{id\_dict (nodes.nodeBuilder.Token\_set attribute)@\spxentry{id\_dict}\spxextra{nodes.nodeBuilder.Token\_set attribute}}

\begin{fulllineitems}
\phantomsection\label{\detokenize{nodes:nodes.nodeBuilder.Token_set.id_dict}}
\pysigstartsignatures
\pysigline
{\sphinxbfcode{\sphinxupquote{id\_dict}}}
\pysigstopsignatures
\sphinxAtStartPar
A dictionary of tokens, mapping ID to token in tokens.
\begin{quote}\begin{description}
\sphinxlineitem{Type}
\sphinxAtStartPar
dict

\end{description}\end{quote}

\end{fulllineitems}

\index{num\_tokens (nodes.nodeBuilder.Token\_set attribute)@\spxentry{num\_tokens}\spxextra{nodes.nodeBuilder.Token\_set attribute}}

\begin{fulllineitems}
\phantomsection\label{\detokenize{nodes:nodes.nodeBuilder.Token_set.num_tokens}}
\pysigstartsignatures
\pysigline
{\sphinxbfcode{\sphinxupquote{num\_tokens}}}
\pysigstopsignatures
\sphinxAtStartPar
The number of tokens in the set.
\begin{quote}\begin{description}
\sphinxlineitem{Type}
\sphinxAtStartPar
int

\end{description}\end{quote}

\end{fulllineitems}

\index{connections (nodes.nodeBuilder.Token\_set attribute)@\spxentry{connections}\spxextra{nodes.nodeBuilder.Token\_set attribute}}

\begin{fulllineitems}
\phantomsection\label{\detokenize{nodes:nodes.nodeBuilder.Token_set.connections}}
\pysigstartsignatures
\pysigline
{\sphinxbfcode{\sphinxupquote{connections}}}
\pysigstopsignatures
\sphinxAtStartPar
A matrix of connections between tokens.
\begin{quote}\begin{description}
\sphinxlineitem{Type}
\sphinxAtStartPar
np.ndarray

\end{description}\end{quote}

\end{fulllineitems}

\index{links (nodes.nodeBuilder.Token\_set attribute)@\spxentry{links}\spxextra{nodes.nodeBuilder.Token\_set attribute}}

\begin{fulllineitems}
\phantomsection\label{\detokenize{nodes:nodes.nodeBuilder.Token_set.links}}
\pysigstartsignatures
\pysigline
{\sphinxbfcode{\sphinxupquote{links}}}
\pysigstopsignatures
\sphinxAtStartPar
A matrix of links between tokens and semantics.
\begin{quote}\begin{description}
\sphinxlineitem{Type}
\sphinxAtStartPar
np.ndarray

\end{description}\end{quote}

\end{fulllineitems}

\index{get\_token() (nodes.nodeBuilder.Token\_set method)@\spxentry{get\_token()}\spxextra{nodes.nodeBuilder.Token\_set method}}

\begin{fulllineitems}
\phantomsection\label{\detokenize{nodes:nodes.nodeBuilder.Token_set.get_token}}
\pysigstartsignatures
\pysiglinewithargsret
{\sphinxbfcode{\sphinxupquote{get\_token}}}
{\sphinxparam{\DUrole{n}{name}}}
{}
\pysigstopsignatures
\sphinxAtStartPar
Get a token from the token set by name.
\begin{quote}\begin{description}
\sphinxlineitem{Parameters}
\sphinxAtStartPar
\sphinxstyleliteralstrong{\sphinxupquote{name}} (\sphinxstyleliteralemphasis{\sphinxupquote{str}}) \textendash{} The name of the token.

\end{description}\end{quote}

\end{fulllineitems}

\index{get\_token\_by\_id() (nodes.nodeBuilder.Token\_set method)@\spxentry{get\_token\_by\_id()}\spxextra{nodes.nodeBuilder.Token\_set method}}

\begin{fulllineitems}
\phantomsection\label{\detokenize{nodes:nodes.nodeBuilder.Token_set.get_token_by_id}}
\pysigstartsignatures
\pysiglinewithargsret
{\sphinxbfcode{\sphinxupquote{get\_token\_by\_id}}}
{\sphinxparam{\DUrole{n}{ID}}}
{}
\pysigstopsignatures
\sphinxAtStartPar
Get a token from the token set by ID.
\begin{quote}\begin{description}
\sphinxlineitem{Parameters}
\sphinxAtStartPar
\sphinxstyleliteralstrong{\sphinxupquote{ID}} (\sphinxstyleliteralemphasis{\sphinxupquote{int}}) \textendash{} The ID of the token.

\end{description}\end{quote}

\end{fulllineitems}

\index{get\_token\_tensor() (nodes.nodeBuilder.Token\_set method)@\spxentry{get\_token\_tensor()}\spxextra{nodes.nodeBuilder.Token\_set method}}

\begin{fulllineitems}
\phantomsection\label{\detokenize{nodes:nodes.nodeBuilder.Token_set.get_token_tensor}}
\pysigstartsignatures
\pysiglinewithargsret
{\sphinxbfcode{\sphinxupquote{get\_token\_tensor}}}
{}
{}
\pysigstopsignatures
\sphinxAtStartPar
Get the token tensor for the token set.

\end{fulllineitems}

\index{tensorise() (nodes.nodeBuilder.Token\_set method)@\spxentry{tensorise()}\spxextra{nodes.nodeBuilder.Token\_set method}}

\begin{fulllineitems}
\phantomsection\label{\detokenize{nodes:nodes.nodeBuilder.Token_set.tensorise}}
\pysigstartsignatures
\pysiglinewithargsret
{\sphinxbfcode{\sphinxupquote{tensorise}}}
{}
{}
\pysigstopsignatures
\sphinxAtStartPar
Tensorise the token set, creating a tensor of tokens, and tensors of connections and links to semantics.

\end{fulllineitems}


\end{fulllineitems}

\index{nodeBuilder (class in nodes.nodeBuilder)@\spxentry{nodeBuilder}\spxextra{class in nodes.nodeBuilder}}

\begin{fulllineitems}
\phantomsection\label{\detokenize{nodes:nodes.nodeBuilder.nodeBuilder}}
\pysigstartsignatures
\pysiglinewithargsret
{\sphinxbfcode{\sphinxupquote{\DUrole{k}{class}\DUrole{w}{ }}}\sphinxcode{\sphinxupquote{nodes.nodeBuilder.}}\sphinxbfcode{\sphinxupquote{nodeBuilder}}}
{\sphinxparam{\DUrole{n}{symProps}\DUrole{p}{:}\DUrole{w}{ }\DUrole{n}{list\DUrole{p}{{[}}dict\DUrole{p}{{]}}}\DUrole{w}{ }\DUrole{o}{=}\DUrole{w}{ }\DUrole{default_value}{None}}\sphinxparamcomma \sphinxparam{\DUrole{n}{file\_path}\DUrole{p}{:}\DUrole{w}{ }\DUrole{n}{str}\DUrole{w}{ }\DUrole{o}{=}\DUrole{w}{ }\DUrole{default_value}{None}}}
{}
\pysigstopsignatures
\sphinxAtStartPar
Bases: \sphinxcode{\sphinxupquote{object}}

\sphinxAtStartPar
A class for building the nodes object.
\index{symProps (nodes.nodeBuilder.nodeBuilder attribute)@\spxentry{symProps}\spxextra{nodes.nodeBuilder.nodeBuilder attribute}}

\begin{fulllineitems}
\phantomsection\label{\detokenize{nodes:nodes.nodeBuilder.nodeBuilder.symProps}}
\pysigstartsignatures
\pysigline
{\sphinxbfcode{\sphinxupquote{symProps}}}
\pysigstopsignatures
\sphinxAtStartPar
A list of symProps.
\begin{quote}\begin{description}
\sphinxlineitem{Type}
\sphinxAtStartPar
list

\end{description}\end{quote}

\end{fulllineitems}

\index{file\_path (nodes.nodeBuilder.nodeBuilder attribute)@\spxentry{file\_path}\spxextra{nodes.nodeBuilder.nodeBuilder attribute}}

\begin{fulllineitems}
\phantomsection\label{\detokenize{nodes:nodes.nodeBuilder.nodeBuilder.file_path}}
\pysigstartsignatures
\pysigline
{\sphinxbfcode{\sphinxupquote{file\_path}}}
\pysigstopsignatures
\sphinxAtStartPar
The path to the sym file.
\begin{quote}\begin{description}
\sphinxlineitem{Type}
\sphinxAtStartPar
str

\end{description}\end{quote}

\end{fulllineitems}

\index{token\_sets (nodes.nodeBuilder.nodeBuilder attribute)@\spxentry{token\_sets}\spxextra{nodes.nodeBuilder.nodeBuilder attribute}}

\begin{fulllineitems}
\phantomsection\label{\detokenize{nodes:nodes.nodeBuilder.nodeBuilder.token_sets}}
\pysigstartsignatures
\pysigline
{\sphinxbfcode{\sphinxupquote{token\_sets}}}
\pysigstopsignatures
\sphinxAtStartPar
A dictionary of token sets, mapping set to token set object.
\begin{quote}\begin{description}
\sphinxlineitem{Type}
\sphinxAtStartPar
dict

\end{description}\end{quote}

\end{fulllineitems}

\index{set\_map (nodes.nodeBuilder.nodeBuilder attribute)@\spxentry{set\_map}\spxextra{nodes.nodeBuilder.nodeBuilder attribute}}

\begin{fulllineitems}
\phantomsection\label{\detokenize{nodes:nodes.nodeBuilder.nodeBuilder.set_map}}
\pysigstartsignatures
\pysigline
{\sphinxbfcode{\sphinxupquote{set\_map}}}
\pysigstopsignatures
\sphinxAtStartPar
A dictionary of set mappings, mapping set name to set. Used for reading set from symProps file.
\begin{quote}\begin{description}
\sphinxlineitem{Type}
\sphinxAtStartPar
dict

\end{description}\end{quote}

\end{fulllineitems}

\index{build\_mem\_objects() (nodes.nodeBuilder.nodeBuilder method)@\spxentry{build\_mem\_objects()}\spxextra{nodes.nodeBuilder.nodeBuilder method}}

\begin{fulllineitems}
\phantomsection\label{\detokenize{nodes:nodes.nodeBuilder.nodeBuilder.build_mem_objects}}
\pysigstartsignatures
\pysiglinewithargsret
{\sphinxbfcode{\sphinxupquote{build\_mem\_objects}}}
{}
{}
\pysigstopsignatures
\sphinxAtStartPar
Build the mem objects. (links, mappings)

\end{fulllineitems}

\index{build\_node\_tensors() (nodes.nodeBuilder.nodeBuilder method)@\spxentry{build\_node\_tensors()}\spxextra{nodes.nodeBuilder.nodeBuilder method}}

\begin{fulllineitems}
\phantomsection\label{\detokenize{nodes:nodes.nodeBuilder.nodeBuilder.build_node_tensors}}
\pysigstartsignatures
\pysiglinewithargsret
{\sphinxbfcode{\sphinxupquote{build\_node\_tensors}}}
{}
{}
\pysigstopsignatures
\sphinxAtStartPar
Build the per set tensor objects. (driver, recipient, memory, new\_set, semantics)

\end{fulllineitems}

\index{build\_nodes() (nodes.nodeBuilder.nodeBuilder method)@\spxentry{build\_nodes()}\spxextra{nodes.nodeBuilder.nodeBuilder method}}

\begin{fulllineitems}
\phantomsection\label{\detokenize{nodes:nodes.nodeBuilder.nodeBuilder.build_nodes}}
\pysigstartsignatures
\pysiglinewithargsret
{\sphinxbfcode{\sphinxupquote{build\_nodes}}}
{\sphinxparam{\DUrole{n}{DORA\_mode}\DUrole{o}{=}\DUrole{default_value}{True}}}
{}
\pysigstopsignatures
\sphinxAtStartPar
Build the nodes object.
\begin{quote}\begin{description}
\sphinxlineitem{Parameters}
\sphinxAtStartPar
\sphinxstyleliteralstrong{\sphinxupquote{DORA\_mode}} (\sphinxstyleliteralemphasis{\sphinxupquote{bool}}) \textendash{} Whether to use DORA mode.

\sphinxlineitem{Returns}
\sphinxAtStartPar
The nodes object.

\sphinxlineitem{Return type}
\sphinxAtStartPar
nodes ({\hyperref[\detokenize{nodes:nodes.nodes.Nodes}]{\sphinxcrossref{Nodes}}})

\sphinxlineitem{Raises}
\sphinxAtStartPar
\sphinxstyleliteralstrong{\sphinxupquote{ValueError}} \textendash{} If no symProps or file\_path set.

\end{description}\end{quote}

\end{fulllineitems}

\index{build\_set\_tensors() (nodes.nodeBuilder.nodeBuilder method)@\spxentry{build\_set\_tensors()}\spxextra{nodes.nodeBuilder.nodeBuilder method}}

\begin{fulllineitems}
\phantomsection\label{\detokenize{nodes:nodes.nodeBuilder.nodeBuilder.build_set_tensors}}
\pysigstartsignatures
\pysiglinewithargsret
{\sphinxbfcode{\sphinxupquote{build\_set\_tensors}}}
{}
{}
\pysigstopsignatures
\sphinxAtStartPar
Build the sem\_set and token\_sets.

\end{fulllineitems}

\index{get\_symProps\_from\_file() (nodes.nodeBuilder.nodeBuilder method)@\spxentry{get\_symProps\_from\_file()}\spxextra{nodes.nodeBuilder.nodeBuilder method}}

\begin{fulllineitems}
\phantomsection\label{\detokenize{nodes:nodes.nodeBuilder.nodeBuilder.get_symProps_from_file}}
\pysigstartsignatures
\pysiglinewithargsret
{\sphinxbfcode{\sphinxupquote{get\_symProps\_from\_file}}}
{}
{}
\pysigstopsignatures
\sphinxAtStartPar
Read the symProps from the file into a list of dicts.

\end{fulllineitems}


\end{fulllineitems}



\section{nodes.nodeEnums module}
\label{\detokenize{nodes:module-nodes.nodeEnums}}\label{\detokenize{nodes:nodes-nodeenums-module}}\index{module@\spxentry{module}!nodes.nodeEnums@\spxentry{nodes.nodeEnums}}\index{nodes.nodeEnums@\spxentry{nodes.nodeEnums}!module@\spxentry{module}}\index{B (class in nodes.nodeEnums)@\spxentry{B}\spxextra{class in nodes.nodeEnums}}

\begin{fulllineitems}
\phantomsection\label{\detokenize{nodes:nodes.nodeEnums.B}}
\pysigstartsignatures
\pysiglinewithargsret
{\sphinxbfcode{\sphinxupquote{\DUrole{k}{class}\DUrole{w}{ }}}\sphinxcode{\sphinxupquote{nodes.nodeEnums.}}\sphinxbfcode{\sphinxupquote{B}}}
{\sphinxparam{\DUrole{n}{value}}}
{}
\pysigstopsignatures
\sphinxAtStartPar
Bases: \sphinxcode{\sphinxupquote{IntEnum}}
\index{FALSE (nodes.nodeEnums.B attribute)@\spxentry{FALSE}\spxextra{nodes.nodeEnums.B attribute}}

\begin{fulllineitems}
\phantomsection\label{\detokenize{nodes:nodes.nodeEnums.B.FALSE}}
\pysigstartsignatures
\pysigline
{\sphinxbfcode{\sphinxupquote{FALSE}}\sphinxbfcode{\sphinxupquote{\DUrole{w}{ }\DUrole{p}{=}\DUrole{w}{ }0}}}
\pysigstopsignatures
\end{fulllineitems}

\index{TRUE (nodes.nodeEnums.B attribute)@\spxentry{TRUE}\spxextra{nodes.nodeEnums.B attribute}}

\begin{fulllineitems}
\phantomsection\label{\detokenize{nodes:nodes.nodeEnums.B.TRUE}}
\pysigstartsignatures
\pysigline
{\sphinxbfcode{\sphinxupquote{TRUE}}\sphinxbfcode{\sphinxupquote{\DUrole{w}{ }\DUrole{p}{=}\DUrole{w}{ }1}}}
\pysigstopsignatures
\end{fulllineitems}


\end{fulllineitems}

\index{MappingFields (class in nodes.nodeEnums)@\spxentry{MappingFields}\spxextra{class in nodes.nodeEnums}}

\begin{fulllineitems}
\phantomsection\label{\detokenize{nodes:nodes.nodeEnums.MappingFields}}
\pysigstartsignatures
\pysiglinewithargsret
{\sphinxbfcode{\sphinxupquote{\DUrole{k}{class}\DUrole{w}{ }}}\sphinxcode{\sphinxupquote{nodes.nodeEnums.}}\sphinxbfcode{\sphinxupquote{MappingFields}}}
{\sphinxparam{\DUrole{n}{value}}}
{}
\pysigstopsignatures
\sphinxAtStartPar
Bases: \sphinxcode{\sphinxupquote{IntEnum}}
\index{CONNETIONS (nodes.nodeEnums.MappingFields attribute)@\spxentry{CONNETIONS}\spxextra{nodes.nodeEnums.MappingFields attribute}}

\begin{fulllineitems}
\phantomsection\label{\detokenize{nodes:nodes.nodeEnums.MappingFields.CONNETIONS}}
\pysigstartsignatures
\pysigline
{\sphinxbfcode{\sphinxupquote{CONNETIONS}}\sphinxbfcode{\sphinxupquote{\DUrole{w}{ }\DUrole{p}{=}\DUrole{w}{ }3}}}
\pysigstopsignatures
\end{fulllineitems}

\index{HYPOTHESIS (nodes.nodeEnums.MappingFields attribute)@\spxentry{HYPOTHESIS}\spxextra{nodes.nodeEnums.MappingFields attribute}}

\begin{fulllineitems}
\phantomsection\label{\detokenize{nodes:nodes.nodeEnums.MappingFields.HYPOTHESIS}}
\pysigstartsignatures
\pysigline
{\sphinxbfcode{\sphinxupquote{HYPOTHESIS}}\sphinxbfcode{\sphinxupquote{\DUrole{w}{ }\DUrole{p}{=}\DUrole{w}{ }1}}}
\pysigstopsignatures
\end{fulllineitems}

\index{MAX\_HYP (nodes.nodeEnums.MappingFields attribute)@\spxentry{MAX\_HYP}\spxextra{nodes.nodeEnums.MappingFields attribute}}

\begin{fulllineitems}
\phantomsection\label{\detokenize{nodes:nodes.nodeEnums.MappingFields.MAX_HYP}}
\pysigstartsignatures
\pysigline
{\sphinxbfcode{\sphinxupquote{MAX\_HYP}}\sphinxbfcode{\sphinxupquote{\DUrole{w}{ }\DUrole{p}{=}\DUrole{w}{ }2}}}
\pysigstopsignatures
\end{fulllineitems}

\index{WEIGHT (nodes.nodeEnums.MappingFields attribute)@\spxentry{WEIGHT}\spxextra{nodes.nodeEnums.MappingFields attribute}}

\begin{fulllineitems}
\phantomsection\label{\detokenize{nodes:nodes.nodeEnums.MappingFields.WEIGHT}}
\pysigstartsignatures
\pysigline
{\sphinxbfcode{\sphinxupquote{WEIGHT}}\sphinxbfcode{\sphinxupquote{\DUrole{w}{ }\DUrole{p}{=}\DUrole{w}{ }0}}}
\pysigstopsignatures
\end{fulllineitems}


\end{fulllineitems}

\index{Mode (class in nodes.nodeEnums)@\spxentry{Mode}\spxextra{class in nodes.nodeEnums}}

\begin{fulllineitems}
\phantomsection\label{\detokenize{nodes:nodes.nodeEnums.Mode}}
\pysigstartsignatures
\pysiglinewithargsret
{\sphinxbfcode{\sphinxupquote{\DUrole{k}{class}\DUrole{w}{ }}}\sphinxcode{\sphinxupquote{nodes.nodeEnums.}}\sphinxbfcode{\sphinxupquote{Mode}}}
{\sphinxparam{\DUrole{n}{value}}}
{}
\pysigstopsignatures
\sphinxAtStartPar
Bases: \sphinxcode{\sphinxupquote{IntEnum}}
\index{CHILD (nodes.nodeEnums.Mode attribute)@\spxentry{CHILD}\spxextra{nodes.nodeEnums.Mode attribute}}

\begin{fulllineitems}
\phantomsection\label{\detokenize{nodes:nodes.nodeEnums.Mode.CHILD}}
\pysigstartsignatures
\pysigline
{\sphinxbfcode{\sphinxupquote{CHILD}}\sphinxbfcode{\sphinxupquote{\DUrole{w}{ }\DUrole{p}{=}\DUrole{w}{ }0}}}
\pysigstopsignatures
\end{fulllineitems}

\index{NEUTRAL (nodes.nodeEnums.Mode attribute)@\spxentry{NEUTRAL}\spxextra{nodes.nodeEnums.Mode attribute}}

\begin{fulllineitems}
\phantomsection\label{\detokenize{nodes:nodes.nodeEnums.Mode.NEUTRAL}}
\pysigstartsignatures
\pysigline
{\sphinxbfcode{\sphinxupquote{NEUTRAL}}\sphinxbfcode{\sphinxupquote{\DUrole{w}{ }\DUrole{p}{=}\DUrole{w}{ }1}}}
\pysigstopsignatures
\end{fulllineitems}

\index{PARENT (nodes.nodeEnums.Mode attribute)@\spxentry{PARENT}\spxextra{nodes.nodeEnums.Mode attribute}}

\begin{fulllineitems}
\phantomsection\label{\detokenize{nodes:nodes.nodeEnums.Mode.PARENT}}
\pysigstartsignatures
\pysigline
{\sphinxbfcode{\sphinxupquote{PARENT}}\sphinxbfcode{\sphinxupquote{\DUrole{w}{ }\DUrole{p}{=}\DUrole{w}{ }2}}}
\pysigstopsignatures
\end{fulllineitems}


\end{fulllineitems}

\index{OntStatus (class in nodes.nodeEnums)@\spxentry{OntStatus}\spxextra{class in nodes.nodeEnums}}

\begin{fulllineitems}
\phantomsection\label{\detokenize{nodes:nodes.nodeEnums.OntStatus}}
\pysigstartsignatures
\pysiglinewithargsret
{\sphinxbfcode{\sphinxupquote{\DUrole{k}{class}\DUrole{w}{ }}}\sphinxcode{\sphinxupquote{nodes.nodeEnums.}}\sphinxbfcode{\sphinxupquote{OntStatus}}}
{\sphinxparam{\DUrole{n}{value}}}
{}
\pysigstopsignatures
\sphinxAtStartPar
Bases: \sphinxcode{\sphinxupquote{IntEnum}}
\index{SDM (nodes.nodeEnums.OntStatus attribute)@\spxentry{SDM}\spxextra{nodes.nodeEnums.OntStatus attribute}}

\begin{fulllineitems}
\phantomsection\label{\detokenize{nodes:nodes.nodeEnums.OntStatus.SDM}}
\pysigstartsignatures
\pysigline
{\sphinxbfcode{\sphinxupquote{SDM}}\sphinxbfcode{\sphinxupquote{\DUrole{w}{ }\DUrole{p}{=}\DUrole{w}{ }2}}}
\pysigstopsignatures
\end{fulllineitems}

\index{STATE (nodes.nodeEnums.OntStatus attribute)@\spxentry{STATE}\spxextra{nodes.nodeEnums.OntStatus attribute}}

\begin{fulllineitems}
\phantomsection\label{\detokenize{nodes:nodes.nodeEnums.OntStatus.STATE}}
\pysigstartsignatures
\pysigline
{\sphinxbfcode{\sphinxupquote{STATE}}\sphinxbfcode{\sphinxupquote{\DUrole{w}{ }\DUrole{p}{=}\DUrole{w}{ }0}}}
\pysigstopsignatures
\end{fulllineitems}

\index{VALUE (nodes.nodeEnums.OntStatus attribute)@\spxentry{VALUE}\spxextra{nodes.nodeEnums.OntStatus attribute}}

\begin{fulllineitems}
\phantomsection\label{\detokenize{nodes:nodes.nodeEnums.OntStatus.VALUE}}
\pysigstartsignatures
\pysigline
{\sphinxbfcode{\sphinxupquote{VALUE}}\sphinxbfcode{\sphinxupquote{\DUrole{w}{ }\DUrole{p}{=}\DUrole{w}{ }1}}}
\pysigstopsignatures
\end{fulllineitems}


\end{fulllineitems}

\index{SF (class in nodes.nodeEnums)@\spxentry{SF}\spxextra{class in nodes.nodeEnums}}

\begin{fulllineitems}
\phantomsection\label{\detokenize{nodes:nodes.nodeEnums.SF}}
\pysigstartsignatures
\pysiglinewithargsret
{\sphinxbfcode{\sphinxupquote{\DUrole{k}{class}\DUrole{w}{ }}}\sphinxcode{\sphinxupquote{nodes.nodeEnums.}}\sphinxbfcode{\sphinxupquote{SF}}}
{\sphinxparam{\DUrole{n}{value}}}
{}
\pysigstopsignatures
\sphinxAtStartPar
Bases: \sphinxcode{\sphinxupquote{IntEnum}}
\index{ACT (nodes.nodeEnums.SF attribute)@\spxentry{ACT}\spxextra{nodes.nodeEnums.SF attribute}}

\begin{fulllineitems}
\phantomsection\label{\detokenize{nodes:nodes.nodeEnums.SF.ACT}}
\pysigstartsignatures
\pysigline
{\sphinxbfcode{\sphinxupquote{ACT}}\sphinxbfcode{\sphinxupquote{\DUrole{w}{ }\DUrole{p}{=}\DUrole{w}{ }6}}}
\pysigstopsignatures
\end{fulllineitems}

\index{AMOUNT (nodes.nodeEnums.SF attribute)@\spxentry{AMOUNT}\spxextra{nodes.nodeEnums.SF attribute}}

\begin{fulllineitems}
\phantomsection\label{\detokenize{nodes:nodes.nodeEnums.SF.AMOUNT}}
\pysigstartsignatures
\pysigline
{\sphinxbfcode{\sphinxupquote{AMOUNT}}\sphinxbfcode{\sphinxupquote{\DUrole{w}{ }\DUrole{p}{=}\DUrole{w}{ }3}}}
\pysigstopsignatures
\end{fulllineitems}

\index{ID (nodes.nodeEnums.SF attribute)@\spxentry{ID}\spxextra{nodes.nodeEnums.SF attribute}}

\begin{fulllineitems}
\phantomsection\label{\detokenize{nodes:nodes.nodeEnums.SF.ID}}
\pysigstartsignatures
\pysigline
{\sphinxbfcode{\sphinxupquote{ID}}\sphinxbfcode{\sphinxupquote{\DUrole{w}{ }\DUrole{p}{=}\DUrole{w}{ }0}}}
\pysigstopsignatures
\end{fulllineitems}

\index{INPUT (nodes.nodeEnums.SF attribute)@\spxentry{INPUT}\spxextra{nodes.nodeEnums.SF attribute}}

\begin{fulllineitems}
\phantomsection\label{\detokenize{nodes:nodes.nodeEnums.SF.INPUT}}
\pysigstartsignatures
\pysigline
{\sphinxbfcode{\sphinxupquote{INPUT}}\sphinxbfcode{\sphinxupquote{\DUrole{w}{ }\DUrole{p}{=}\DUrole{w}{ }4}}}
\pysigstopsignatures
\end{fulllineitems}

\index{MAX\_INPUT (nodes.nodeEnums.SF attribute)@\spxentry{MAX\_INPUT}\spxextra{nodes.nodeEnums.SF attribute}}

\begin{fulllineitems}
\phantomsection\label{\detokenize{nodes:nodes.nodeEnums.SF.MAX_INPUT}}
\pysigstartsignatures
\pysigline
{\sphinxbfcode{\sphinxupquote{MAX\_INPUT}}\sphinxbfcode{\sphinxupquote{\DUrole{w}{ }\DUrole{p}{=}\DUrole{w}{ }5}}}
\pysigstopsignatures
\end{fulllineitems}

\index{ONT\_STATUS (nodes.nodeEnums.SF attribute)@\spxentry{ONT\_STATUS}\spxextra{nodes.nodeEnums.SF attribute}}

\begin{fulllineitems}
\phantomsection\label{\detokenize{nodes:nodes.nodeEnums.SF.ONT_STATUS}}
\pysigstartsignatures
\pysigline
{\sphinxbfcode{\sphinxupquote{ONT\_STATUS}}\sphinxbfcode{\sphinxupquote{\DUrole{w}{ }\DUrole{p}{=}\DUrole{w}{ }2}}}
\pysigstopsignatures
\end{fulllineitems}

\index{TYPE (nodes.nodeEnums.SF attribute)@\spxentry{TYPE}\spxextra{nodes.nodeEnums.SF attribute}}

\begin{fulllineitems}
\phantomsection\label{\detokenize{nodes:nodes.nodeEnums.SF.TYPE}}
\pysigstartsignatures
\pysigline
{\sphinxbfcode{\sphinxupquote{TYPE}}\sphinxbfcode{\sphinxupquote{\DUrole{w}{ }\DUrole{p}{=}\DUrole{w}{ }1}}}
\pysigstopsignatures
\end{fulllineitems}


\end{fulllineitems}

\index{Set (class in nodes.nodeEnums)@\spxentry{Set}\spxextra{class in nodes.nodeEnums}}

\begin{fulllineitems}
\phantomsection\label{\detokenize{nodes:nodes.nodeEnums.Set}}
\pysigstartsignatures
\pysiglinewithargsret
{\sphinxbfcode{\sphinxupquote{\DUrole{k}{class}\DUrole{w}{ }}}\sphinxcode{\sphinxupquote{nodes.nodeEnums.}}\sphinxbfcode{\sphinxupquote{Set}}}
{\sphinxparam{\DUrole{n}{value}}}
{}
\pysigstopsignatures
\sphinxAtStartPar
Bases: \sphinxcode{\sphinxupquote{IntEnum}}
\index{DRIVER (nodes.nodeEnums.Set attribute)@\spxentry{DRIVER}\spxextra{nodes.nodeEnums.Set attribute}}

\begin{fulllineitems}
\phantomsection\label{\detokenize{nodes:nodes.nodeEnums.Set.DRIVER}}
\pysigstartsignatures
\pysigline
{\sphinxbfcode{\sphinxupquote{DRIVER}}\sphinxbfcode{\sphinxupquote{\DUrole{w}{ }\DUrole{p}{=}\DUrole{w}{ }0}}}
\pysigstopsignatures
\end{fulllineitems}

\index{MEMORY (nodes.nodeEnums.Set attribute)@\spxentry{MEMORY}\spxextra{nodes.nodeEnums.Set attribute}}

\begin{fulllineitems}
\phantomsection\label{\detokenize{nodes:nodes.nodeEnums.Set.MEMORY}}
\pysigstartsignatures
\pysigline
{\sphinxbfcode{\sphinxupquote{MEMORY}}\sphinxbfcode{\sphinxupquote{\DUrole{w}{ }\DUrole{p}{=}\DUrole{w}{ }2}}}
\pysigstopsignatures
\end{fulllineitems}

\index{NEW\_SET (nodes.nodeEnums.Set attribute)@\spxentry{NEW\_SET}\spxextra{nodes.nodeEnums.Set attribute}}

\begin{fulllineitems}
\phantomsection\label{\detokenize{nodes:nodes.nodeEnums.Set.NEW_SET}}
\pysigstartsignatures
\pysigline
{\sphinxbfcode{\sphinxupquote{NEW\_SET}}\sphinxbfcode{\sphinxupquote{\DUrole{w}{ }\DUrole{p}{=}\DUrole{w}{ }3}}}
\pysigstopsignatures
\end{fulllineitems}

\index{RECIPIENT (nodes.nodeEnums.Set attribute)@\spxentry{RECIPIENT}\spxextra{nodes.nodeEnums.Set attribute}}

\begin{fulllineitems}
\phantomsection\label{\detokenize{nodes:nodes.nodeEnums.Set.RECIPIENT}}
\pysigstartsignatures
\pysigline
{\sphinxbfcode{\sphinxupquote{RECIPIENT}}\sphinxbfcode{\sphinxupquote{\DUrole{w}{ }\DUrole{p}{=}\DUrole{w}{ }1}}}
\pysigstopsignatures
\end{fulllineitems}


\end{fulllineitems}

\index{TF (class in nodes.nodeEnums)@\spxentry{TF}\spxextra{class in nodes.nodeEnums}}

\begin{fulllineitems}
\phantomsection\label{\detokenize{nodes:nodes.nodeEnums.TF}}
\pysigstartsignatures
\pysiglinewithargsret
{\sphinxbfcode{\sphinxupquote{\DUrole{k}{class}\DUrole{w}{ }}}\sphinxcode{\sphinxupquote{nodes.nodeEnums.}}\sphinxbfcode{\sphinxupquote{TF}}}
{\sphinxparam{\DUrole{n}{value}}}
{}
\pysigstopsignatures
\sphinxAtStartPar
Bases: \sphinxcode{\sphinxupquote{IntEnum}}
\index{ACT (nodes.nodeEnums.TF attribute)@\spxentry{ACT}\spxextra{nodes.nodeEnums.TF attribute}}

\begin{fulllineitems}
\phantomsection\label{\detokenize{nodes:nodes.nodeEnums.TF.ACT}}
\pysigstartsignatures
\pysigline
{\sphinxbfcode{\sphinxupquote{ACT}}\sphinxbfcode{\sphinxupquote{\DUrole{w}{ }\DUrole{p}{=}\DUrole{w}{ }12}}}
\pysigstopsignatures
\end{fulllineitems}

\index{ANALOG (nodes.nodeEnums.TF attribute)@\spxentry{ANALOG}\spxextra{nodes.nodeEnums.TF attribute}}

\begin{fulllineitems}
\phantomsection\label{\detokenize{nodes:nodes.nodeEnums.TF.ANALOG}}
\pysigstartsignatures
\pysigline
{\sphinxbfcode{\sphinxupquote{ANALOG}}\sphinxbfcode{\sphinxupquote{\DUrole{w}{ }\DUrole{p}{=}\DUrole{w}{ }3}}}
\pysigstopsignatures
\end{fulllineitems}

\index{BU\_INPUT (nodes.nodeEnums.TF attribute)@\spxentry{BU\_INPUT}\spxextra{nodes.nodeEnums.TF attribute}}

\begin{fulllineitems}
\phantomsection\label{\detokenize{nodes:nodes.nodeEnums.TF.BU_INPUT}}
\pysigstartsignatures
\pysigline
{\sphinxbfcode{\sphinxupquote{BU\_INPUT}}\sphinxbfcode{\sphinxupquote{\DUrole{w}{ }\DUrole{p}{=}\DUrole{w}{ }18}}}
\pysigstopsignatures
\end{fulllineitems}

\index{COPIED\_DR\_INDEX (nodes.nodeEnums.TF attribute)@\spxentry{COPIED\_DR\_INDEX}\spxextra{nodes.nodeEnums.TF attribute}}

\begin{fulllineitems}
\phantomsection\label{\detokenize{nodes:nodes.nodeEnums.TF.COPIED_DR_INDEX}}
\pysigstartsignatures
\pysigline
{\sphinxbfcode{\sphinxupquote{COPIED\_DR\_INDEX}}\sphinxbfcode{\sphinxupquote{\DUrole{w}{ }\DUrole{p}{=}\DUrole{w}{ }26}}}
\pysigstopsignatures
\end{fulllineitems}

\index{COPY\_FOR\_DR (nodes.nodeEnums.TF attribute)@\spxentry{COPY\_FOR\_DR}\spxextra{nodes.nodeEnums.TF attribute}}

\begin{fulllineitems}
\phantomsection\label{\detokenize{nodes:nodes.nodeEnums.TF.COPY_FOR_DR}}
\pysigstartsignatures
\pysigline
{\sphinxbfcode{\sphinxupquote{COPY\_FOR\_DR}}\sphinxbfcode{\sphinxupquote{\DUrole{w}{ }\DUrole{p}{=}\DUrole{w}{ }25}}}
\pysigstopsignatures
\end{fulllineitems}

\index{DELETED (nodes.nodeEnums.TF attribute)@\spxentry{DELETED}\spxextra{nodes.nodeEnums.TF attribute}}

\begin{fulllineitems}
\phantomsection\label{\detokenize{nodes:nodes.nodeEnums.TF.DELETED}}
\pysigstartsignatures
\pysigline
{\sphinxbfcode{\sphinxupquote{DELETED}}\sphinxbfcode{\sphinxupquote{\DUrole{w}{ }\DUrole{p}{=}\DUrole{w}{ }28}}}
\pysigstopsignatures
\end{fulllineitems}

\index{GROUP\_LAYER (nodes.nodeEnums.TF attribute)@\spxentry{GROUP\_LAYER}\spxextra{nodes.nodeEnums.TF attribute}}

\begin{fulllineitems}
\phantomsection\label{\detokenize{nodes:nodes.nodeEnums.TF.GROUP_LAYER}}
\pysigstartsignatures
\pysigline
{\sphinxbfcode{\sphinxupquote{GROUP\_LAYER}}\sphinxbfcode{\sphinxupquote{\DUrole{w}{ }\DUrole{p}{=}\DUrole{w}{ }8}}}
\pysigstopsignatures
\end{fulllineitems}

\index{ID (nodes.nodeEnums.TF attribute)@\spxentry{ID}\spxextra{nodes.nodeEnums.TF attribute}}

\begin{fulllineitems}
\phantomsection\label{\detokenize{nodes:nodes.nodeEnums.TF.ID}}
\pysigstartsignatures
\pysigline
{\sphinxbfcode{\sphinxupquote{ID}}\sphinxbfcode{\sphinxupquote{\DUrole{w}{ }\DUrole{p}{=}\DUrole{w}{ }0}}}
\pysigstopsignatures
\end{fulllineitems}

\index{INFERRED (nodes.nodeEnums.TF attribute)@\spxentry{INFERRED}\spxextra{nodes.nodeEnums.TF attribute}}

\begin{fulllineitems}
\phantomsection\label{\detokenize{nodes:nodes.nodeEnums.TF.INFERRED}}
\pysigstartsignatures
\pysigline
{\sphinxbfcode{\sphinxupquote{INFERRED}}\sphinxbfcode{\sphinxupquote{\DUrole{w}{ }\DUrole{p}{=}\DUrole{w}{ }23}}}
\pysigstopsignatures
\end{fulllineitems}

\index{INHIBITOR\_ACT (nodes.nodeEnums.TF attribute)@\spxentry{INHIBITOR\_ACT}\spxextra{nodes.nodeEnums.TF attribute}}

\begin{fulllineitems}
\phantomsection\label{\detokenize{nodes:nodes.nodeEnums.TF.INHIBITOR_ACT}}
\pysigstartsignatures
\pysigline
{\sphinxbfcode{\sphinxupquote{INHIBITOR\_ACT}}\sphinxbfcode{\sphinxupquote{\DUrole{w}{ }\DUrole{p}{=}\DUrole{w}{ }15}}}
\pysigstopsignatures
\end{fulllineitems}

\index{INHIBITOR\_INPUT (nodes.nodeEnums.TF attribute)@\spxentry{INHIBITOR\_INPUT}\spxextra{nodes.nodeEnums.TF attribute}}

\begin{fulllineitems}
\phantomsection\label{\detokenize{nodes:nodes.nodeEnums.TF.INHIBITOR_INPUT}}
\pysigstartsignatures
\pysigline
{\sphinxbfcode{\sphinxupquote{INHIBITOR\_INPUT}}\sphinxbfcode{\sphinxupquote{\DUrole{w}{ }\DUrole{p}{=}\DUrole{w}{ }14}}}
\pysigstopsignatures
\end{fulllineitems}

\index{INHIBITOR\_THRESHOLD (nodes.nodeEnums.TF attribute)@\spxentry{INHIBITOR\_THRESHOLD}\spxextra{nodes.nodeEnums.TF attribute}}

\begin{fulllineitems}
\phantomsection\label{\detokenize{nodes:nodes.nodeEnums.TF.INHIBITOR_THRESHOLD}}
\pysigstartsignatures
\pysigline
{\sphinxbfcode{\sphinxupquote{INHIBITOR\_THRESHOLD}}\sphinxbfcode{\sphinxupquote{\DUrole{w}{ }\DUrole{p}{=}\DUrole{w}{ }7}}}
\pysigstopsignatures
\end{fulllineitems}

\index{LATERAL\_INPUT (nodes.nodeEnums.TF attribute)@\spxentry{LATERAL\_INPUT}\spxextra{nodes.nodeEnums.TF attribute}}

\begin{fulllineitems}
\phantomsection\label{\detokenize{nodes:nodes.nodeEnums.TF.LATERAL_INPUT}}
\pysigstartsignatures
\pysigline
{\sphinxbfcode{\sphinxupquote{LATERAL\_INPUT}}\sphinxbfcode{\sphinxupquote{\DUrole{w}{ }\DUrole{p}{=}\DUrole{w}{ }19}}}
\pysigstopsignatures
\end{fulllineitems}

\index{MADE\_UNIT (nodes.nodeEnums.TF attribute)@\spxentry{MADE\_UNIT}\spxextra{nodes.nodeEnums.TF attribute}}

\begin{fulllineitems}
\phantomsection\label{\detokenize{nodes:nodes.nodeEnums.TF.MADE_UNIT}}
\pysigstartsignatures
\pysigline
{\sphinxbfcode{\sphinxupquote{MADE\_UNIT}}\sphinxbfcode{\sphinxupquote{\DUrole{w}{ }\DUrole{p}{=}\DUrole{w}{ }5}}}
\pysigstopsignatures
\end{fulllineitems}

\index{MAKER\_UNIT (nodes.nodeEnums.TF attribute)@\spxentry{MAKER\_UNIT}\spxextra{nodes.nodeEnums.TF attribute}}

\begin{fulllineitems}
\phantomsection\label{\detokenize{nodes:nodes.nodeEnums.TF.MAKER_UNIT}}
\pysigstartsignatures
\pysigline
{\sphinxbfcode{\sphinxupquote{MAKER\_UNIT}}\sphinxbfcode{\sphinxupquote{\DUrole{w}{ }\DUrole{p}{=}\DUrole{w}{ }6}}}
\pysigstopsignatures
\end{fulllineitems}

\index{MAP\_INPUT (nodes.nodeEnums.TF attribute)@\spxentry{MAP\_INPUT}\spxextra{nodes.nodeEnums.TF attribute}}

\begin{fulllineitems}
\phantomsection\label{\detokenize{nodes:nodes.nodeEnums.TF.MAP_INPUT}}
\pysigstartsignatures
\pysigline
{\sphinxbfcode{\sphinxupquote{MAP\_INPUT}}\sphinxbfcode{\sphinxupquote{\DUrole{w}{ }\DUrole{p}{=}\DUrole{w}{ }20}}}
\pysigstopsignatures
\end{fulllineitems}

\index{MAX\_ACT (nodes.nodeEnums.TF attribute)@\spxentry{MAX\_ACT}\spxextra{nodes.nodeEnums.TF attribute}}

\begin{fulllineitems}
\phantomsection\label{\detokenize{nodes:nodes.nodeEnums.TF.MAX_ACT}}
\pysigstartsignatures
\pysigline
{\sphinxbfcode{\sphinxupquote{MAX\_ACT}}\sphinxbfcode{\sphinxupquote{\DUrole{w}{ }\DUrole{p}{=}\DUrole{w}{ }13}}}
\pysigstopsignatures
\end{fulllineitems}

\index{MAX\_MAP (nodes.nodeEnums.TF attribute)@\spxentry{MAX\_MAP}\spxextra{nodes.nodeEnums.TF attribute}}

\begin{fulllineitems}
\phantomsection\label{\detokenize{nodes:nodes.nodeEnums.TF.MAX_MAP}}
\pysigstartsignatures
\pysigline
{\sphinxbfcode{\sphinxupquote{MAX\_MAP}}\sphinxbfcode{\sphinxupquote{\DUrole{w}{ }\DUrole{p}{=}\DUrole{w}{ }16}}}
\pysigstopsignatures
\end{fulllineitems}

\index{MAX\_MAP\_UNIT (nodes.nodeEnums.TF attribute)@\spxentry{MAX\_MAP\_UNIT}\spxextra{nodes.nodeEnums.TF attribute}}

\begin{fulllineitems}
\phantomsection\label{\detokenize{nodes:nodes.nodeEnums.TF.MAX_MAP_UNIT}}
\pysigstartsignatures
\pysigline
{\sphinxbfcode{\sphinxupquote{MAX\_MAP\_UNIT}}\sphinxbfcode{\sphinxupquote{\DUrole{w}{ }\DUrole{p}{=}\DUrole{w}{ }4}}}
\pysigstopsignatures
\end{fulllineitems}

\index{MAX\_SEM\_WEIGHT (nodes.nodeEnums.TF attribute)@\spxentry{MAX\_SEM\_WEIGHT}\spxextra{nodes.nodeEnums.TF attribute}}

\begin{fulllineitems}
\phantomsection\label{\detokenize{nodes:nodes.nodeEnums.TF.MAX_SEM_WEIGHT}}
\pysigstartsignatures
\pysigline
{\sphinxbfcode{\sphinxupquote{MAX\_SEM\_WEIGHT}}\sphinxbfcode{\sphinxupquote{\DUrole{w}{ }\DUrole{p}{=}\DUrole{w}{ }22}}}
\pysigstopsignatures
\end{fulllineitems}

\index{MODE (nodes.nodeEnums.TF attribute)@\spxentry{MODE}\spxextra{nodes.nodeEnums.TF attribute}}

\begin{fulllineitems}
\phantomsection\label{\detokenize{nodes:nodes.nodeEnums.TF.MODE}}
\pysigstartsignatures
\pysigline
{\sphinxbfcode{\sphinxupquote{MODE}}\sphinxbfcode{\sphinxupquote{\DUrole{w}{ }\DUrole{p}{=}\DUrole{w}{ }9}}}
\pysigstopsignatures
\end{fulllineitems}

\index{NET\_INPUT (nodes.nodeEnums.TF attribute)@\spxentry{NET\_INPUT}\spxextra{nodes.nodeEnums.TF attribute}}

\begin{fulllineitems}
\phantomsection\label{\detokenize{nodes:nodes.nodeEnums.TF.NET_INPUT}}
\pysigstartsignatures
\pysigline
{\sphinxbfcode{\sphinxupquote{NET\_INPUT}}\sphinxbfcode{\sphinxupquote{\DUrole{w}{ }\DUrole{p}{=}\DUrole{w}{ }21}}}
\pysigstopsignatures
\end{fulllineitems}

\index{PRED (nodes.nodeEnums.TF attribute)@\spxentry{PRED}\spxextra{nodes.nodeEnums.TF attribute}}

\begin{fulllineitems}
\phantomsection\label{\detokenize{nodes:nodes.nodeEnums.TF.PRED}}
\pysigstartsignatures
\pysigline
{\sphinxbfcode{\sphinxupquote{PRED}}\sphinxbfcode{\sphinxupquote{\DUrole{w}{ }\DUrole{p}{=}\DUrole{w}{ }29}}}
\pysigstopsignatures
\end{fulllineitems}

\index{RETRIEVED (nodes.nodeEnums.TF attribute)@\spxentry{RETRIEVED}\spxextra{nodes.nodeEnums.TF attribute}}

\begin{fulllineitems}
\phantomsection\label{\detokenize{nodes:nodes.nodeEnums.TF.RETRIEVED}}
\pysigstartsignatures
\pysigline
{\sphinxbfcode{\sphinxupquote{RETRIEVED}}\sphinxbfcode{\sphinxupquote{\DUrole{w}{ }\DUrole{p}{=}\DUrole{w}{ }24}}}
\pysigstopsignatures
\end{fulllineitems}

\index{SEM\_COUNT (nodes.nodeEnums.TF attribute)@\spxentry{SEM\_COUNT}\spxextra{nodes.nodeEnums.TF attribute}}

\begin{fulllineitems}
\phantomsection\label{\detokenize{nodes:nodes.nodeEnums.TF.SEM_COUNT}}
\pysigstartsignatures
\pysigline
{\sphinxbfcode{\sphinxupquote{SEM\_COUNT}}\sphinxbfcode{\sphinxupquote{\DUrole{w}{ }\DUrole{p}{=}\DUrole{w}{ }11}}}
\pysigstopsignatures
\end{fulllineitems}

\index{SET (nodes.nodeEnums.TF attribute)@\spxentry{SET}\spxextra{nodes.nodeEnums.TF attribute}}

\begin{fulllineitems}
\phantomsection\label{\detokenize{nodes:nodes.nodeEnums.TF.SET}}
\pysigstartsignatures
\pysigline
{\sphinxbfcode{\sphinxupquote{SET}}\sphinxbfcode{\sphinxupquote{\DUrole{w}{ }\DUrole{p}{=}\DUrole{w}{ }2}}}
\pysigstopsignatures
\end{fulllineitems}

\index{SIM\_MADE (nodes.nodeEnums.TF attribute)@\spxentry{SIM\_MADE}\spxextra{nodes.nodeEnums.TF attribute}}

\begin{fulllineitems}
\phantomsection\label{\detokenize{nodes:nodes.nodeEnums.TF.SIM_MADE}}
\pysigstartsignatures
\pysigline
{\sphinxbfcode{\sphinxupquote{SIM\_MADE}}\sphinxbfcode{\sphinxupquote{\DUrole{w}{ }\DUrole{p}{=}\DUrole{w}{ }27}}}
\pysigstopsignatures
\end{fulllineitems}

\index{TD\_INPUT (nodes.nodeEnums.TF attribute)@\spxentry{TD\_INPUT}\spxextra{nodes.nodeEnums.TF attribute}}

\begin{fulllineitems}
\phantomsection\label{\detokenize{nodes:nodes.nodeEnums.TF.TD_INPUT}}
\pysigstartsignatures
\pysigline
{\sphinxbfcode{\sphinxupquote{TD\_INPUT}}\sphinxbfcode{\sphinxupquote{\DUrole{w}{ }\DUrole{p}{=}\DUrole{w}{ }17}}}
\pysigstopsignatures
\end{fulllineitems}

\index{TIMES\_FIRED (nodes.nodeEnums.TF attribute)@\spxentry{TIMES\_FIRED}\spxextra{nodes.nodeEnums.TF attribute}}

\begin{fulllineitems}
\phantomsection\label{\detokenize{nodes:nodes.nodeEnums.TF.TIMES_FIRED}}
\pysigstartsignatures
\pysigline
{\sphinxbfcode{\sphinxupquote{TIMES\_FIRED}}\sphinxbfcode{\sphinxupquote{\DUrole{w}{ }\DUrole{p}{=}\DUrole{w}{ }10}}}
\pysigstopsignatures
\end{fulllineitems}

\index{TYPE (nodes.nodeEnums.TF attribute)@\spxentry{TYPE}\spxextra{nodes.nodeEnums.TF attribute}}

\begin{fulllineitems}
\phantomsection\label{\detokenize{nodes:nodes.nodeEnums.TF.TYPE}}
\pysigstartsignatures
\pysigline
{\sphinxbfcode{\sphinxupquote{TYPE}}\sphinxbfcode{\sphinxupquote{\DUrole{w}{ }\DUrole{p}{=}\DUrole{w}{ }1}}}
\pysigstopsignatures
\end{fulllineitems}


\end{fulllineitems}

\index{Type (class in nodes.nodeEnums)@\spxentry{Type}\spxextra{class in nodes.nodeEnums}}

\begin{fulllineitems}
\phantomsection\label{\detokenize{nodes:nodes.nodeEnums.Type}}
\pysigstartsignatures
\pysiglinewithargsret
{\sphinxbfcode{\sphinxupquote{\DUrole{k}{class}\DUrole{w}{ }}}\sphinxcode{\sphinxupquote{nodes.nodeEnums.}}\sphinxbfcode{\sphinxupquote{Type}}}
{\sphinxparam{\DUrole{n}{value}}}
{}
\pysigstopsignatures
\sphinxAtStartPar
Bases: \sphinxcode{\sphinxupquote{IntEnum}}
\index{GROUP (nodes.nodeEnums.Type attribute)@\spxentry{GROUP}\spxextra{nodes.nodeEnums.Type attribute}}

\begin{fulllineitems}
\phantomsection\label{\detokenize{nodes:nodes.nodeEnums.Type.GROUP}}
\pysigstartsignatures
\pysigline
{\sphinxbfcode{\sphinxupquote{GROUP}}\sphinxbfcode{\sphinxupquote{\DUrole{w}{ }\DUrole{p}{=}\DUrole{w}{ }3}}}
\pysigstopsignatures
\end{fulllineitems}

\index{P (nodes.nodeEnums.Type attribute)@\spxentry{P}\spxextra{nodes.nodeEnums.Type attribute}}

\begin{fulllineitems}
\phantomsection\label{\detokenize{nodes:nodes.nodeEnums.Type.P}}
\pysigstartsignatures
\pysigline
{\sphinxbfcode{\sphinxupquote{P}}\sphinxbfcode{\sphinxupquote{\DUrole{w}{ }\DUrole{p}{=}\DUrole{w}{ }2}}}
\pysigstopsignatures
\end{fulllineitems}

\index{PO (nodes.nodeEnums.Type attribute)@\spxentry{PO}\spxextra{nodes.nodeEnums.Type attribute}}

\begin{fulllineitems}
\phantomsection\label{\detokenize{nodes:nodes.nodeEnums.Type.PO}}
\pysigstartsignatures
\pysigline
{\sphinxbfcode{\sphinxupquote{PO}}\sphinxbfcode{\sphinxupquote{\DUrole{w}{ }\DUrole{p}{=}\DUrole{w}{ }0}}}
\pysigstopsignatures
\end{fulllineitems}

\index{RB (nodes.nodeEnums.Type attribute)@\spxentry{RB}\spxextra{nodes.nodeEnums.Type attribute}}

\begin{fulllineitems}
\phantomsection\label{\detokenize{nodes:nodes.nodeEnums.Type.RB}}
\pysigstartsignatures
\pysigline
{\sphinxbfcode{\sphinxupquote{RB}}\sphinxbfcode{\sphinxupquote{\DUrole{w}{ }\DUrole{p}{=}\DUrole{w}{ }1}}}
\pysigstopsignatures
\end{fulllineitems}

\index{SEMANTIC (nodes.nodeEnums.Type attribute)@\spxentry{SEMANTIC}\spxextra{nodes.nodeEnums.Type attribute}}

\begin{fulllineitems}
\phantomsection\label{\detokenize{nodes:nodes.nodeEnums.Type.SEMANTIC}}
\pysigstartsignatures
\pysigline
{\sphinxbfcode{\sphinxupquote{SEMANTIC}}\sphinxbfcode{\sphinxupquote{\DUrole{w}{ }\DUrole{p}{=}\DUrole{w}{ }4}}}
\pysigstopsignatures
\end{fulllineitems}


\end{fulllineitems}



\section{nodes.nodeMemObjects module}
\label{\detokenize{nodes:module-nodes.nodeMemObjects}}\label{\detokenize{nodes:nodes-nodememobjects-module}}\index{module@\spxentry{module}!nodes.nodeMemObjects@\spxentry{nodes.nodeMemObjects}}\index{nodes.nodeMemObjects@\spxentry{nodes.nodeMemObjects}!module@\spxentry{module}}\index{Links (class in nodes.nodeMemObjects)@\spxentry{Links}\spxextra{class in nodes.nodeMemObjects}}

\begin{fulllineitems}
\phantomsection\label{\detokenize{nodes:nodes.nodeMemObjects.Links}}
\pysigstartsignatures
\pysiglinewithargsret
{\sphinxbfcode{\sphinxupquote{\DUrole{k}{class}\DUrole{w}{ }}}\sphinxcode{\sphinxupquote{nodes.nodeMemObjects.}}\sphinxbfcode{\sphinxupquote{Links}}}
{\sphinxparam{\DUrole{n}{driver\_links}}\sphinxparamcomma \sphinxparam{\DUrole{n}{recipient\_links}}\sphinxparamcomma \sphinxparam{\DUrole{n}{memory\_links}}}
{}
\pysigstopsignatures
\sphinxAtStartPar
Bases: \sphinxcode{\sphinxupquote{object}}

\sphinxAtStartPar
A class for representing weighted connections between token sets and semantics.
\index{add\_links() (nodes.nodeMemObjects.Links method)@\spxentry{add\_links()}\spxextra{nodes.nodeMemObjects.Links method}}

\begin{fulllineitems}
\phantomsection\label{\detokenize{nodes:nodes.nodeMemObjects.Links.add_links}}
\pysigstartsignatures
\pysiglinewithargsret
{\sphinxbfcode{\sphinxupquote{add\_links}}}
{\sphinxparam{\DUrole{n}{set}\DUrole{p}{:}\DUrole{w}{ }\DUrole{n}{{\hyperref[\detokenize{nodes:nodes.nodeEnums.Set}]{\sphinxcrossref{Set}}}}}\sphinxparamcomma \sphinxparam{\DUrole{n}{links}}}
{}
\pysigstopsignatures
\sphinxAtStartPar
Add links to the adjacency matrix.
TODO: implement

\end{fulllineitems}


\end{fulllineitems}

\index{Mappings (class in nodes.nodeMemObjects)@\spxentry{Mappings}\spxextra{class in nodes.nodeMemObjects}}

\begin{fulllineitems}
\phantomsection\label{\detokenize{nodes:nodes.nodeMemObjects.Mappings}}
\pysigstartsignatures
\pysiglinewithargsret
{\sphinxbfcode{\sphinxupquote{\DUrole{k}{class}\DUrole{w}{ }}}\sphinxcode{\sphinxupquote{nodes.nodeMemObjects.}}\sphinxbfcode{\sphinxupquote{Mappings}}}
{\sphinxparam{\DUrole{n}{connections}}\sphinxparamcomma \sphinxparam{\DUrole{n}{weights}}\sphinxparamcomma \sphinxparam{\DUrole{n}{hypotheses}}\sphinxparamcomma \sphinxparam{\DUrole{n}{max\_hyps}}}
{}
\pysigstopsignatures
\sphinxAtStartPar
Bases: \sphinxcode{\sphinxupquote{object}}

\sphinxAtStartPar
A class for storing mappings and hypothesis information.
\index{add\_mappings() (nodes.nodeMemObjects.Mappings method)@\spxentry{add\_mappings()}\spxextra{nodes.nodeMemObjects.Mappings method}}

\begin{fulllineitems}
\phantomsection\label{\detokenize{nodes:nodes.nodeMemObjects.Mappings.add_mappings}}
\pysigstartsignatures
\pysiglinewithargsret
{\sphinxbfcode{\sphinxupquote{add\_mappings}}}
{\sphinxparam{\DUrole{n}{mappings}}}
{}
\pysigstopsignatures
\sphinxAtStartPar
Add mappings to the adjacency matrix.
TODO: implement

\end{fulllineitems}

\index{connections() (nodes.nodeMemObjects.Mappings method)@\spxentry{connections()}\spxextra{nodes.nodeMemObjects.Mappings method}}

\begin{fulllineitems}
\phantomsection\label{\detokenize{nodes:nodes.nodeMemObjects.Mappings.connections}}
\pysigstartsignatures
\pysiglinewithargsret
{\sphinxbfcode{\sphinxupquote{connections}}}
{}
{}
\pysigstopsignatures
\sphinxAtStartPar
Return the connections matrix from the adjacency matrix.

\end{fulllineitems}

\index{hypotheses() (nodes.nodeMemObjects.Mappings method)@\spxentry{hypotheses()}\spxextra{nodes.nodeMemObjects.Mappings method}}

\begin{fulllineitems}
\phantomsection\label{\detokenize{nodes:nodes.nodeMemObjects.Mappings.hypotheses}}
\pysigstartsignatures
\pysiglinewithargsret
{\sphinxbfcode{\sphinxupquote{hypotheses}}}
{}
{}
\pysigstopsignatures
\sphinxAtStartPar
Return the hypotheses matrix from the adjacency matrix.

\end{fulllineitems}

\index{max\_hyps() (nodes.nodeMemObjects.Mappings method)@\spxentry{max\_hyps()}\spxextra{nodes.nodeMemObjects.Mappings method}}

\begin{fulllineitems}
\phantomsection\label{\detokenize{nodes:nodes.nodeMemObjects.Mappings.max_hyps}}
\pysigstartsignatures
\pysiglinewithargsret
{\sphinxbfcode{\sphinxupquote{max\_hyps}}}
{}
{}
\pysigstopsignatures
\sphinxAtStartPar
Return the max hypotheses matrix from the adjacency matrix.

\end{fulllineitems}

\index{updateHypotheses() (nodes.nodeMemObjects.Mappings method)@\spxentry{updateHypotheses()}\spxextra{nodes.nodeMemObjects.Mappings method}}

\begin{fulllineitems}
\phantomsection\label{\detokenize{nodes:nodes.nodeMemObjects.Mappings.updateHypotheses}}
\pysigstartsignatures
\pysiglinewithargsret
{\sphinxbfcode{\sphinxupquote{updateHypotheses}}}
{\sphinxparam{\DUrole{n}{hypotheses}}}
{}
\pysigstopsignatures
\sphinxAtStartPar
Update the hypotheses matrix.
TODO: implement

\end{fulllineitems}

\index{weights() (nodes.nodeMemObjects.Mappings method)@\spxentry{weights()}\spxextra{nodes.nodeMemObjects.Mappings method}}

\begin{fulllineitems}
\phantomsection\label{\detokenize{nodes:nodes.nodeMemObjects.Mappings.weights}}
\pysigstartsignatures
\pysiglinewithargsret
{\sphinxbfcode{\sphinxupquote{weights}}}
{}
{}
\pysigstopsignatures
\sphinxAtStartPar
Return the weights matrix from the adjacency matrix.

\end{fulllineitems}


\end{fulllineitems}



\section{nodes.nodePrinter module}
\label{\detokenize{nodes:module-nodes.nodePrinter}}\label{\detokenize{nodes:nodes-nodeprinter-module}}\index{module@\spxentry{module}!nodes.nodePrinter@\spxentry{nodes.nodePrinter}}\index{nodes.nodePrinter@\spxentry{nodes.nodePrinter}!module@\spxentry{module}}\index{C (class in nodes.nodePrinter)@\spxentry{C}\spxextra{class in nodes.nodePrinter}}

\begin{fulllineitems}
\phantomsection\label{\detokenize{nodes:nodes.nodePrinter.C}}
\pysigstartsignatures
\pysiglinewithargsret
{\sphinxbfcode{\sphinxupquote{\DUrole{k}{class}\DUrole{w}{ }}}\sphinxcode{\sphinxupquote{nodes.nodePrinter.}}\sphinxbfcode{\sphinxupquote{C}}}
{\sphinxparam{\DUrole{n}{value}}}
{}
\pysigstopsignatures
\sphinxAtStartPar
Bases: \sphinxcode{\sphinxupquote{IntEnum}}

\sphinxAtStartPar
Enum for the characters to print.
\index{BOTTOM\_LEFT (nodes.nodePrinter.C attribute)@\spxentry{BOTTOM\_LEFT}\spxextra{nodes.nodePrinter.C attribute}}

\begin{fulllineitems}
\phantomsection\label{\detokenize{nodes:nodes.nodePrinter.C.BOTTOM_LEFT}}
\pysigstartsignatures
\pysigline
{\sphinxbfcode{\sphinxupquote{BOTTOM\_LEFT}}\sphinxbfcode{\sphinxupquote{\DUrole{w}{ }\DUrole{p}{=}\DUrole{w}{ }2}}}
\pysigstopsignatures
\end{fulllineitems}

\index{BOTTOM\_RIGHT (nodes.nodePrinter.C attribute)@\spxentry{BOTTOM\_RIGHT}\spxextra{nodes.nodePrinter.C attribute}}

\begin{fulllineitems}
\phantomsection\label{\detokenize{nodes:nodes.nodePrinter.C.BOTTOM_RIGHT}}
\pysigstartsignatures
\pysigline
{\sphinxbfcode{\sphinxupquote{BOTTOM\_RIGHT}}\sphinxbfcode{\sphinxupquote{\DUrole{w}{ }\DUrole{p}{=}\DUrole{w}{ }3}}}
\pysigstopsignatures
\end{fulllineitems}

\index{CROSS (nodes.nodePrinter.C attribute)@\spxentry{CROSS}\spxextra{nodes.nodePrinter.C attribute}}

\begin{fulllineitems}
\phantomsection\label{\detokenize{nodes:nodes.nodePrinter.C.CROSS}}
\pysigstartsignatures
\pysigline
{\sphinxbfcode{\sphinxupquote{CROSS}}\sphinxbfcode{\sphinxupquote{\DUrole{w}{ }\DUrole{p}{=}\DUrole{w}{ }6}}}
\pysigstopsignatures
\end{fulllineitems}

\index{HORIZONTAL (nodes.nodePrinter.C attribute)@\spxentry{HORIZONTAL}\spxextra{nodes.nodePrinter.C attribute}}

\begin{fulllineitems}
\phantomsection\label{\detokenize{nodes:nodes.nodePrinter.C.HORIZONTAL}}
\pysigstartsignatures
\pysigline
{\sphinxbfcode{\sphinxupquote{HORIZONTAL}}\sphinxbfcode{\sphinxupquote{\DUrole{w}{ }\DUrole{p}{=}\DUrole{w}{ }4}}}
\pysigstopsignatures
\end{fulllineitems}

\index{HORIZONTAL\_DOWN (nodes.nodePrinter.C attribute)@\spxentry{HORIZONTAL\_DOWN}\spxextra{nodes.nodePrinter.C attribute}}

\begin{fulllineitems}
\phantomsection\label{\detokenize{nodes:nodes.nodePrinter.C.HORIZONTAL_DOWN}}
\pysigstartsignatures
\pysigline
{\sphinxbfcode{\sphinxupquote{HORIZONTAL\_DOWN}}\sphinxbfcode{\sphinxupquote{\DUrole{w}{ }\DUrole{p}{=}\DUrole{w}{ }7}}}
\pysigstopsignatures
\end{fulllineitems}

\index{HORIZONTAL\_UP (nodes.nodePrinter.C attribute)@\spxentry{HORIZONTAL\_UP}\spxextra{nodes.nodePrinter.C attribute}}

\begin{fulllineitems}
\phantomsection\label{\detokenize{nodes:nodes.nodePrinter.C.HORIZONTAL_UP}}
\pysigstartsignatures
\pysigline
{\sphinxbfcode{\sphinxupquote{HORIZONTAL\_UP}}\sphinxbfcode{\sphinxupquote{\DUrole{w}{ }\DUrole{p}{=}\DUrole{w}{ }8}}}
\pysigstopsignatures
\end{fulllineitems}

\index{TOP\_LEFT (nodes.nodePrinter.C attribute)@\spxentry{TOP\_LEFT}\spxextra{nodes.nodePrinter.C attribute}}

\begin{fulllineitems}
\phantomsection\label{\detokenize{nodes:nodes.nodePrinter.C.TOP_LEFT}}
\pysigstartsignatures
\pysigline
{\sphinxbfcode{\sphinxupquote{TOP\_LEFT}}\sphinxbfcode{\sphinxupquote{\DUrole{w}{ }\DUrole{p}{=}\DUrole{w}{ }0}}}
\pysigstopsignatures
\end{fulllineitems}

\index{TOP\_RIGHT (nodes.nodePrinter.C attribute)@\spxentry{TOP\_RIGHT}\spxextra{nodes.nodePrinter.C attribute}}

\begin{fulllineitems}
\phantomsection\label{\detokenize{nodes:nodes.nodePrinter.C.TOP_RIGHT}}
\pysigstartsignatures
\pysigline
{\sphinxbfcode{\sphinxupquote{TOP\_RIGHT}}\sphinxbfcode{\sphinxupquote{\DUrole{w}{ }\DUrole{p}{=}\DUrole{w}{ }1}}}
\pysigstopsignatures
\end{fulllineitems}

\index{VERTICAL (nodes.nodePrinter.C attribute)@\spxentry{VERTICAL}\spxextra{nodes.nodePrinter.C attribute}}

\begin{fulllineitems}
\phantomsection\label{\detokenize{nodes:nodes.nodePrinter.C.VERTICAL}}
\pysigstartsignatures
\pysigline
{\sphinxbfcode{\sphinxupquote{VERTICAL}}\sphinxbfcode{\sphinxupquote{\DUrole{w}{ }\DUrole{p}{=}\DUrole{w}{ }5}}}
\pysigstopsignatures
\end{fulllineitems}

\index{VERTICAL\_LEFT (nodes.nodePrinter.C attribute)@\spxentry{VERTICAL\_LEFT}\spxextra{nodes.nodePrinter.C attribute}}

\begin{fulllineitems}
\phantomsection\label{\detokenize{nodes:nodes.nodePrinter.C.VERTICAL_LEFT}}
\pysigstartsignatures
\pysigline
{\sphinxbfcode{\sphinxupquote{VERTICAL\_LEFT}}\sphinxbfcode{\sphinxupquote{\DUrole{w}{ }\DUrole{p}{=}\DUrole{w}{ }9}}}
\pysigstopsignatures
\end{fulllineitems}

\index{VERTICAL\_RIGHT (nodes.nodePrinter.C attribute)@\spxentry{VERTICAL\_RIGHT}\spxextra{nodes.nodePrinter.C attribute}}

\begin{fulllineitems}
\phantomsection\label{\detokenize{nodes:nodes.nodePrinter.C.VERTICAL_RIGHT}}
\pysigstartsignatures
\pysigline
{\sphinxbfcode{\sphinxupquote{VERTICAL\_RIGHT}}\sphinxbfcode{\sphinxupquote{\DUrole{w}{ }\DUrole{p}{=}\DUrole{w}{ }10}}}
\pysigstopsignatures
\end{fulllineitems}


\end{fulllineitems}

\index{lineTypes (class in nodes.nodePrinter)@\spxentry{lineTypes}\spxextra{class in nodes.nodePrinter}}

\begin{fulllineitems}
\phantomsection\label{\detokenize{nodes:nodes.nodePrinter.lineTypes}}
\pysigstartsignatures
\pysiglinewithargsret
{\sphinxbfcode{\sphinxupquote{\DUrole{k}{class}\DUrole{w}{ }}}\sphinxcode{\sphinxupquote{nodes.nodePrinter.}}\sphinxbfcode{\sphinxupquote{lineTypes}}}
{\sphinxparam{\DUrole{n}{value}}}
{}
\pysigstopsignatures
\sphinxAtStartPar
Bases: \sphinxcode{\sphinxupquote{IntEnum}}

\sphinxAtStartPar
Enum for the type of line to print.
\index{BOTTOM (nodes.nodePrinter.lineTypes attribute)@\spxentry{BOTTOM}\spxextra{nodes.nodePrinter.lineTypes attribute}}

\begin{fulllineitems}
\phantomsection\label{\detokenize{nodes:nodes.nodePrinter.lineTypes.BOTTOM}}
\pysigstartsignatures
\pysigline
{\sphinxbfcode{\sphinxupquote{BOTTOM}}\sphinxbfcode{\sphinxupquote{\DUrole{w}{ }\DUrole{p}{=}\DUrole{w}{ }2}}}
\pysigstopsignatures
\end{fulllineitems}

\index{MIDDLE (nodes.nodePrinter.lineTypes attribute)@\spxentry{MIDDLE}\spxextra{nodes.nodePrinter.lineTypes attribute}}

\begin{fulllineitems}
\phantomsection\label{\detokenize{nodes:nodes.nodePrinter.lineTypes.MIDDLE}}
\pysigstartsignatures
\pysigline
{\sphinxbfcode{\sphinxupquote{MIDDLE}}\sphinxbfcode{\sphinxupquote{\DUrole{w}{ }\DUrole{p}{=}\DUrole{w}{ }1}}}
\pysigstopsignatures
\end{fulllineitems}

\index{SPLIT (nodes.nodePrinter.lineTypes attribute)@\spxentry{SPLIT}\spxextra{nodes.nodePrinter.lineTypes attribute}}

\begin{fulllineitems}
\phantomsection\label{\detokenize{nodes:nodes.nodePrinter.lineTypes.SPLIT}}
\pysigstartsignatures
\pysigline
{\sphinxbfcode{\sphinxupquote{SPLIT}}\sphinxbfcode{\sphinxupquote{\DUrole{w}{ }\DUrole{p}{=}\DUrole{w}{ }3}}}
\pysigstopsignatures
\end{fulllineitems}

\index{TOP (nodes.nodePrinter.lineTypes attribute)@\spxentry{TOP}\spxextra{nodes.nodePrinter.lineTypes attribute}}

\begin{fulllineitems}
\phantomsection\label{\detokenize{nodes:nodes.nodePrinter.lineTypes.TOP}}
\pysigstartsignatures
\pysigline
{\sphinxbfcode{\sphinxupquote{TOP}}\sphinxbfcode{\sphinxupquote{\DUrole{w}{ }\DUrole{p}{=}\DUrole{w}{ }0}}}
\pysigstopsignatures
\end{fulllineitems}


\end{fulllineitems}

\index{nodePrinter (class in nodes.nodePrinter)@\spxentry{nodePrinter}\spxextra{class in nodes.nodePrinter}}

\begin{fulllineitems}
\phantomsection\label{\detokenize{nodes:nodes.nodePrinter.nodePrinter}}
\pysigstartsignatures
\pysiglinewithargsret
{\sphinxbfcode{\sphinxupquote{\DUrole{k}{class}\DUrole{w}{ }}}\sphinxcode{\sphinxupquote{nodes.nodePrinter.}}\sphinxbfcode{\sphinxupquote{nodePrinter}}}
{\sphinxparam{\DUrole{n}{nodes}\DUrole{p}{:}\DUrole{w}{ }\DUrole{n}{{\hyperref[\detokenize{nodes:nodes.nodes.Nodes}]{\sphinxcrossref{Nodes}}}}}\sphinxparamcomma \sphinxparam{\DUrole{n}{print\_to\_console}\DUrole{p}{:}\DUrole{w}{ }\DUrole{n}{bool}\DUrole{w}{ }\DUrole{o}{=}\DUrole{w}{ }\DUrole{default_value}{True}}\sphinxparamcomma \sphinxparam{\DUrole{n}{log\_file}\DUrole{p}{:}\DUrole{w}{ }\DUrole{n}{str}\DUrole{w}{ }\DUrole{o}{=}\DUrole{w}{ }\DUrole{default_value}{None}}}
{}
\pysigstopsignatures
\sphinxAtStartPar
Bases: \sphinxcode{\sphinxupquote{object}}

\sphinxAtStartPar
This class is used to print the nodes and their tensors to the console or a file.
\index{nodes (nodes.nodePrinter.nodePrinter attribute)@\spxentry{nodes}\spxextra{nodes.nodePrinter.nodePrinter attribute}}

\begin{fulllineitems}
\phantomsection\label{\detokenize{nodes:nodes.nodePrinter.nodePrinter.nodes}}
\pysigstartsignatures
\pysigline
{\sphinxbfcode{\sphinxupquote{nodes}}}
\pysigstopsignatures
\sphinxAtStartPar
The nodes object to print.
\begin{quote}\begin{description}
\sphinxlineitem{Type}
\sphinxAtStartPar
{\hyperref[\detokenize{nodes:nodes.nodes.Nodes}]{\sphinxcrossref{Nodes}}}

\end{description}\end{quote}

\end{fulllineitems}

\index{print\_to\_console (nodes.nodePrinter.nodePrinter attribute)@\spxentry{print\_to\_console}\spxextra{nodes.nodePrinter.nodePrinter attribute}}

\begin{fulllineitems}
\phantomsection\label{\detokenize{nodes:nodes.nodePrinter.nodePrinter.print_to_console}}
\pysigstartsignatures
\pysigline
{\sphinxbfcode{\sphinxupquote{print\_to\_console}}}
\pysigstopsignatures
\sphinxAtStartPar
Whether to print to the console.
\begin{quote}\begin{description}
\sphinxlineitem{Type}
\sphinxAtStartPar
bool

\end{description}\end{quote}

\end{fulllineitems}

\index{log\_file (nodes.nodePrinter.nodePrinter attribute)@\spxentry{log\_file}\spxextra{nodes.nodePrinter.nodePrinter attribute}}

\begin{fulllineitems}
\phantomsection\label{\detokenize{nodes:nodes.nodePrinter.nodePrinter.log_file}}
\pysigstartsignatures
\pysigline
{\sphinxbfcode{\sphinxupquote{log\_file}}}
\pysigstopsignatures
\sphinxAtStartPar
The file to print to.
\begin{quote}\begin{description}
\sphinxlineitem{Type}
\sphinxAtStartPar
str

\end{description}\end{quote}

\end{fulllineitems}

\index{label\_values() (nodes.nodePrinter.nodePrinter method)@\spxentry{label\_values()}\spxextra{nodes.nodePrinter.nodePrinter method}}

\begin{fulllineitems}
\phantomsection\label{\detokenize{nodes:nodes.nodePrinter.nodePrinter.label_values}}
\pysigstartsignatures
\pysiglinewithargsret
{\sphinxbfcode{\sphinxupquote{label\_values}}}
{\sphinxparam{\DUrole{n}{row}\DUrole{p}{:}\DUrole{w}{ }\DUrole{n}{list\DUrole{p}{{[}}float\DUrole{p}{{]}}}}\sphinxparamcomma \sphinxparam{\DUrole{n}{types}\DUrole{p}{:}\DUrole{w}{ }\DUrole{n}{list\DUrole{p}{{[}}{\hyperref[\detokenize{nodes:nodes.nodeEnums.TF}]{\sphinxcrossref{TF}}}\DUrole{p}{{]}}}}\sphinxparamcomma \sphinxparam{\DUrole{n}{names}\DUrole{p}{:}\DUrole{w}{ }\DUrole{n}{dict\DUrole{p}{{[}}int\DUrole{p}{,}\DUrole{w}{ }str\DUrole{p}{{]}}}}}
{}
\pysigstopsignatures
\sphinxAtStartPar
Label the values and names of the given row.
:param row: The row to label.
:type row: list
:param types: The types of the features to label. (IE: TF.SET, TF.ID, etc.)
:type types: list
:param names: The names of the tokens. If None, the names will not be added to the row.
:type names: dict

\end{fulllineitems}

\index{print\_con\_tensor() (nodes.nodePrinter.nodePrinter method)@\spxentry{print\_con\_tensor()}\spxextra{nodes.nodePrinter.nodePrinter method}}

\begin{fulllineitems}
\phantomsection\label{\detokenize{nodes:nodes.nodePrinter.nodePrinter.print_con_tensor}}
\pysigstartsignatures
\pysiglinewithargsret
{\sphinxbfcode{\sphinxupquote{print\_con\_tensor}}}
{\sphinxparam{\DUrole{n}{tensor}\DUrole{p}{:}\DUrole{w}{ }\DUrole{n}{Tensor}}\sphinxparamcomma \sphinxparam{\DUrole{n}{mask}\DUrole{o}{=}\DUrole{default_value}{None}}\sphinxparamcomma \sphinxparam{\DUrole{n}{names}\DUrole{o}{=}\DUrole{default_value}{None}}\sphinxparamcomma \sphinxparam{\DUrole{n}{headers}\DUrole{o}{=}\DUrole{default_value}{None}}}
{}
\pysigstopsignatures
\sphinxAtStartPar
Print the given connections tensor.
:param tensor: The tensor to print.
:type tensor: torch.Tensor
:param mask: The mask to apply to the tensor.
:type mask: torch.Tensor
:param names: The names of the tokens. If None, the names will not be printed.
:type names: dict
:param headers: The headers to print, defaults to “Connections:” if left as None.
:type headers: list

\end{fulllineitems}

\index{print\_links\_tensor() (nodes.nodePrinter.nodePrinter method)@\spxentry{print\_links\_tensor()}\spxextra{nodes.nodePrinter.nodePrinter method}}

\begin{fulllineitems}
\phantomsection\label{\detokenize{nodes:nodes.nodePrinter.nodePrinter.print_links_tensor}}
\pysigstartsignatures
\pysiglinewithargsret
{\sphinxbfcode{\sphinxupquote{print\_links\_tensor}}}
{\sphinxparam{\DUrole{n}{tensor}\DUrole{p}{:}\DUrole{w}{ }\DUrole{n}{Tensor}}\sphinxparamcomma \sphinxparam{\DUrole{n}{mask}\DUrole{o}{=}\DUrole{default_value}{None}}\sphinxparamcomma \sphinxparam{\DUrole{n}{names}\DUrole{o}{=}\DUrole{default_value}{None}}\sphinxparamcomma \sphinxparam{\DUrole{n}{headers}\DUrole{o}{=}\DUrole{default_value}{None}}}
{}
\pysigstopsignatures
\sphinxAtStartPar
Print the given links tensor.
:param tensor: The tensor to print.
:type tensor: torch.Tensor
:param mask: The mask to apply to the tensor.
:type mask: torch.Tensor
:param names: The names of the tokens. If None, the names will not be printed.
:type names: dict
:param headers: The headers to print, defaults to “Links:” if left as None.
:type headers: list

\end{fulllineitems}

\index{print\_tk\_tensor() (nodes.nodePrinter.nodePrinter method)@\spxentry{print\_tk\_tensor()}\spxextra{nodes.nodePrinter.nodePrinter method}}

\begin{fulllineitems}
\phantomsection\label{\detokenize{nodes:nodes.nodePrinter.nodePrinter.print_tk_tensor}}
\pysigstartsignatures
\pysiglinewithargsret
{\sphinxbfcode{\sphinxupquote{print\_tk\_tensor}}}
{\sphinxparam{\DUrole{n}{tensor}\DUrole{p}{:}\DUrole{w}{ }\DUrole{n}{Tensor}}\sphinxparamcomma \sphinxparam{\DUrole{n}{types}\DUrole{o}{=}\DUrole{default_value}{None}}\sphinxparamcomma \sphinxparam{\DUrole{n}{label\_values}\DUrole{o}{=}\DUrole{default_value}{True}}\sphinxparamcomma \sphinxparam{\DUrole{n}{names}\DUrole{o}{=}\DUrole{default_value}{None}}\sphinxparamcomma \sphinxparam{\DUrole{n}{headers}\DUrole{o}{=}\DUrole{default_value}{None}}}
{}
\pysigstopsignatures
\sphinxAtStartPar
Print the given tensor of tokens.
:param tensor: The tensor to print.
:type tensor: torch.Tensor
:param types: List of features to print. (IE: TF.SET, TF.ID, etc.)
:type types: list
:param label\_values: Whether to convert feature floats to their enum names. (IE: TF.TYPE == 0.0 \sphinxhyphen{}\textgreater{} TYPE(0.0).name)
:type label\_values: bool
:param names: The names of the tokens. If None, the names will not be printed.
:type names: dict
:param headers: The headers to print, defaults to “Tokens Tensor:” if left as None.
:type headers: list

\end{fulllineitems}

\index{print\_token\_tensor() (nodes.nodePrinter.nodePrinter method)@\spxentry{print\_token\_tensor()}\spxextra{nodes.nodePrinter.nodePrinter method}}

\begin{fulllineitems}
\phantomsection\label{\detokenize{nodes:nodes.nodePrinter.nodePrinter.print_token_tensor}}
\pysigstartsignatures
\pysiglinewithargsret
{\sphinxbfcode{\sphinxupquote{print\_token\_tensor}}}
{\sphinxparam{\DUrole{n}{set}\DUrole{p}{:}\DUrole{w}{ }\DUrole{n}{{\hyperref[\detokenize{nodes:nodes.nodeEnums.Set}]{\sphinxcrossref{Set}}}}}\sphinxparamcomma \sphinxparam{\DUrole{n}{feature\_types}\DUrole{o}{=}\DUrole{default_value}{None}}\sphinxparamcomma \sphinxparam{\DUrole{n}{mask}\DUrole{o}{=}\DUrole{default_value}{None}}\sphinxparamcomma \sphinxparam{\DUrole{n}{label\_values}\DUrole{o}{=}\DUrole{default_value}{True}}\sphinxparamcomma \sphinxparam{\DUrole{n}{label\_names}\DUrole{o}{=}\DUrole{default_value}{True}}\sphinxparamcomma \sphinxparam{\DUrole{n}{headers}\DUrole{o}{=}\DUrole{default_value}{None}}\sphinxparamcomma \sphinxparam{\DUrole{n}{print\_cons}\DUrole{o}{=}\DUrole{default_value}{True}}\sphinxparamcomma \sphinxparam{\DUrole{n}{cons\_headers}\DUrole{o}{=}\DUrole{default_value}{None}}\sphinxparamcomma \sphinxparam{\DUrole{n}{links\_headers}\DUrole{o}{=}\DUrole{default_value}{None}}}
{}
\pysigstopsignatures
\sphinxAtStartPar
Print the token tensor for a given set.
:param set: The set to print.
:type set: Set
:param feature\_types: List of features to print. (IE: TF.SET, TF.ID, etc.)
:type feature\_types: list
:param mask: Mask of subtensor to print.
:type mask: torch.Tensor
:param label\_values: Whether to convert features floats to their enum names. (IE: TF.TYPE == 0.0 \sphinxhyphen{}\textgreater{} TYPE(0.0).name)
:type label\_values: bool
:param label\_names: Whether to include the names for each node. (IE: ID==0 \sphinxhyphen{}\textgreater{} tensor.names{[}0{]})
:type label\_names: bool
:param headers: The headers to print, defaults to “Set: \{set.name\} Tokens” if left as None.
:type headers: list
:param print\_cons: Whether to print the connections tensor.
:type print\_cons: bool
:param cons\_headers: The connections headers to print, defaults to “Set: \{set.name\} Connections” if left as None.
:type cons\_headers: list
:param links\_headers: The links headers to print, defaults to “Set: \{set.name\} Links” if left as None.
:type links\_headers: list

\end{fulllineitems}

\index{print\_tokens() (nodes.nodePrinter.nodePrinter method)@\spxentry{print\_tokens()}\spxextra{nodes.nodePrinter.nodePrinter method}}

\begin{fulllineitems}
\phantomsection\label{\detokenize{nodes:nodes.nodePrinter.nodePrinter.print_tokens}}
\pysigstartsignatures
\pysiglinewithargsret
{\sphinxbfcode{\sphinxupquote{print\_tokens}}}
{\sphinxparam{\DUrole{n}{set}\DUrole{p}{:}\DUrole{w}{ }\DUrole{n}{{\hyperref[\detokenize{nodes:nodes.nodeEnums.Set}]{\sphinxcrossref{Set}}}}}\sphinxparamcomma \sphinxparam{\DUrole{n}{token\_ids}\DUrole{p}{:}\DUrole{w}{ }\DUrole{n}{list\DUrole{p}{{[}}int\DUrole{p}{{]}}}}\sphinxparamcomma \sphinxparam{\DUrole{n}{types}\DUrole{p}{:}\DUrole{w}{ }\DUrole{n}{list\DUrole{p}{{[}}{\hyperref[\detokenize{nodes:nodes.nodeEnums.TF}]{\sphinxcrossref{TF}}}\DUrole{p}{{]}}}}}
{}
\pysigstopsignatures
\sphinxAtStartPar
Print the given tokens.
:param set: The set to print.
:type set: Set
:param token\_ids: The ids of the tokens to print.
:type token\_ids: list
:param types: The types of the features to print. (IE: TF.SET, TF.ID, etc.)
:type types: list

\end{fulllineitems}


\end{fulllineitems}

\index{tablePrinter (class in nodes.nodePrinter)@\spxentry{tablePrinter}\spxextra{class in nodes.nodePrinter}}

\begin{fulllineitems}
\phantomsection\label{\detokenize{nodes:nodes.nodePrinter.tablePrinter}}
\pysigstartsignatures
\pysiglinewithargsret
{\sphinxbfcode{\sphinxupquote{\DUrole{k}{class}\DUrole{w}{ }}}\sphinxcode{\sphinxupquote{nodes.nodePrinter.}}\sphinxbfcode{\sphinxupquote{tablePrinter}}}
{\sphinxparam{\DUrole{n}{columns}\DUrole{p}{:}\DUrole{w}{ }\DUrole{n}{list\DUrole{p}{{[}}str\DUrole{p}{{]}}}}\sphinxparamcomma \sphinxparam{\DUrole{n}{rows}\DUrole{p}{:}\DUrole{w}{ }\DUrole{n}{list\DUrole{p}{{[}}list\DUrole{p}{{[}}str\DUrole{p}{{]}}\DUrole{p}{{]}}}}\sphinxparamcomma \sphinxparam{\DUrole{n}{headers}\DUrole{p}{:}\DUrole{w}{ }\DUrole{n}{list\DUrole{p}{{[}}str\DUrole{p}{{]}}}}\sphinxparamcomma \sphinxparam{\DUrole{n}{log\_file}\DUrole{p}{:}\DUrole{w}{ }\DUrole{n}{str}\DUrole{w}{ }\DUrole{o}{=}\DUrole{w}{ }\DUrole{default_value}{None}}\sphinxparamcomma \sphinxparam{\DUrole{n}{print\_to\_console}\DUrole{p}{:}\DUrole{w}{ }\DUrole{n}{bool}\DUrole{w}{ }\DUrole{o}{=}\DUrole{w}{ }\DUrole{default_value}{True}}}
{}
\pysigstopsignatures
\sphinxAtStartPar
Bases: \sphinxcode{\sphinxupquote{object}}

\sphinxAtStartPar
Print a table of data.
\index{columns (nodes.nodePrinter.tablePrinter attribute)@\spxentry{columns}\spxextra{nodes.nodePrinter.tablePrinter attribute}}

\begin{fulllineitems}
\phantomsection\label{\detokenize{nodes:nodes.nodePrinter.tablePrinter.columns}}
\pysigstartsignatures
\pysigline
{\sphinxbfcode{\sphinxupquote{columns}}}
\pysigstopsignatures
\sphinxAtStartPar
The columns of the table.
\begin{quote}\begin{description}
\sphinxlineitem{Type}
\sphinxAtStartPar
list

\end{description}\end{quote}

\end{fulllineitems}

\index{rows (nodes.nodePrinter.tablePrinter attribute)@\spxentry{rows}\spxextra{nodes.nodePrinter.tablePrinter attribute}}

\begin{fulllineitems}
\phantomsection\label{\detokenize{nodes:nodes.nodePrinter.tablePrinter.rows}}
\pysigstartsignatures
\pysigline
{\sphinxbfcode{\sphinxupquote{rows}}}
\pysigstopsignatures
\sphinxAtStartPar
The rows of the table.
\begin{quote}\begin{description}
\sphinxlineitem{Type}
\sphinxAtStartPar
list

\end{description}\end{quote}

\end{fulllineitems}

\index{headers (nodes.nodePrinter.tablePrinter attribute)@\spxentry{headers}\spxextra{nodes.nodePrinter.tablePrinter attribute}}

\begin{fulllineitems}
\phantomsection\label{\detokenize{nodes:nodes.nodePrinter.tablePrinter.headers}}
\pysigstartsignatures
\pysigline
{\sphinxbfcode{\sphinxupquote{headers}}}
\pysigstopsignatures
\sphinxAtStartPar
The headers of the table.
\begin{quote}\begin{description}
\sphinxlineitem{Type}
\sphinxAtStartPar
list

\end{description}\end{quote}

\end{fulllineitems}

\index{log\_file (nodes.nodePrinter.tablePrinter attribute)@\spxentry{log\_file}\spxextra{nodes.nodePrinter.tablePrinter attribute}}

\begin{fulllineitems}
\phantomsection\label{\detokenize{nodes:nodes.nodePrinter.tablePrinter.log_file}}
\pysigstartsignatures
\pysigline
{\sphinxbfcode{\sphinxupquote{log\_file}}}
\pysigstopsignatures
\sphinxAtStartPar
The file to log to. Only logs if provided.
\begin{quote}\begin{description}
\sphinxlineitem{Type}
\sphinxAtStartPar
str

\end{description}\end{quote}

\end{fulllineitems}

\index{print\_to\_console (nodes.nodePrinter.tablePrinter attribute)@\spxentry{print\_to\_console}\spxextra{nodes.nodePrinter.tablePrinter attribute}}

\begin{fulllineitems}
\phantomsection\label{\detokenize{nodes:nodes.nodePrinter.tablePrinter.print_to_console}}
\pysigstartsignatures
\pysigline
{\sphinxbfcode{\sphinxupquote{print\_to\_console}}}
\pysigstopsignatures
\sphinxAtStartPar
Whether to print to the console.
\begin{quote}\begin{description}
\sphinxlineitem{Type}
\sphinxAtStartPar
bool

\end{description}\end{quote}

\end{fulllineitems}

\index{calc\_col\_widths() (nodes.nodePrinter.tablePrinter method)@\spxentry{calc\_col\_widths()}\spxextra{nodes.nodePrinter.tablePrinter method}}

\begin{fulllineitems}
\phantomsection\label{\detokenize{nodes:nodes.nodePrinter.tablePrinter.calc_col_widths}}
\pysigstartsignatures
\pysiglinewithargsret
{\sphinxbfcode{\sphinxupquote{calc\_col\_widths}}}
{}
{}
\pysigstopsignatures
\sphinxAtStartPar
Calculate the widths of the columns, based on longest content in each column.

\end{fulllineitems}

\index{calc\_header\_width() (nodes.nodePrinter.tablePrinter method)@\spxentry{calc\_header\_width()}\spxextra{nodes.nodePrinter.tablePrinter method}}

\begin{fulllineitems}
\phantomsection\label{\detokenize{nodes:nodes.nodePrinter.tablePrinter.calc_header_width}}
\pysigstartsignatures
\pysiglinewithargsret
{\sphinxbfcode{\sphinxupquote{calc\_header\_width}}}
{}
{}
\pysigstopsignatures
\sphinxAtStartPar
Calculate the widths of the header strings.

\end{fulllineitems}

\index{check\_row\_column\_lengths() (nodes.nodePrinter.tablePrinter method)@\spxentry{check\_row\_column\_lengths()}\spxextra{nodes.nodePrinter.tablePrinter method}}

\begin{fulllineitems}
\phantomsection\label{\detokenize{nodes:nodes.nodePrinter.tablePrinter.check_row_column_lengths}}
\pysigstartsignatures
\pysiglinewithargsret
{\sphinxbfcode{\sphinxupquote{check\_row\_column\_lengths}}}
{}
{}
\pysigstopsignatures
\sphinxAtStartPar
Check that the number of columns in each row matches the number of columns in the table.

\end{fulllineitems}

\index{format() (nodes.nodePrinter.tablePrinter method)@\spxentry{format()}\spxextra{nodes.nodePrinter.tablePrinter method}}

\begin{fulllineitems}
\phantomsection\label{\detokenize{nodes:nodes.nodePrinter.tablePrinter.format}}
\pysigstartsignatures
\pysiglinewithargsret
{\sphinxbfcode{\sphinxupquote{format}}}
{\sphinxparam{\DUrole{n}{data}\DUrole{p}{:}\DUrole{w}{ }\DUrole{n}{list\DUrole{p}{{[}}str\DUrole{p}{{]}}}}}
{}
\pysigstopsignatures
\sphinxAtStartPar
Format the given data into strings.
:param data: The data to format.
:type data: list

\end{fulllineitems}

\index{format\_rows() (nodes.nodePrinter.tablePrinter method)@\spxentry{format\_rows()}\spxextra{nodes.nodePrinter.tablePrinter method}}

\begin{fulllineitems}
\phantomsection\label{\detokenize{nodes:nodes.nodePrinter.tablePrinter.format_rows}}
\pysigstartsignatures
\pysiglinewithargsret
{\sphinxbfcode{\sphinxupquote{format\_rows}}}
{\sphinxparam{\DUrole{n}{rows}\DUrole{p}{:}\DUrole{w}{ }\DUrole{n}{list\DUrole{p}{{[}}list\DUrole{p}{{[}}str\DUrole{p}{{]}}\DUrole{p}{{]}}}}}
{}
\pysigstopsignatures
\sphinxAtStartPar
Format the given rows data into strings.
:param rows: The rows to format.
:type rows: list

\end{fulllineitems}

\index{get\_col\_string() (nodes.nodePrinter.tablePrinter method)@\spxentry{get\_col\_string()}\spxextra{nodes.nodePrinter.tablePrinter method}}

\begin{fulllineitems}
\phantomsection\label{\detokenize{nodes:nodes.nodePrinter.tablePrinter.get_col_string}}
\pysigstartsignatures
\pysiglinewithargsret
{\sphinxbfcode{\sphinxupquote{get\_col\_string}}}
{\sphinxparam{\DUrole{n}{startc}}\sphinxparamcomma \sphinxparam{\DUrole{n}{fillc}}\sphinxparamcomma \sphinxparam{\DUrole{n}{width}}\sphinxparamcomma \sphinxparam{\DUrole{n}{format\_data}\DUrole{o}{=}\DUrole{default_value}{None}}}
{}
\pysigstopsignatures
\sphinxAtStartPar
Return a string for the given column.
:param startc: The character to start the column with.
:type startc: str
:param fillc: The character to fill the column with.
:type fillc: str
:param width: The width of the column.
:type width: int
:param format\_data: The data to format.
:type format\_data: str
\begin{quote}\begin{description}
\sphinxlineitem{Returns}
\sphinxAtStartPar
String for the given column.

\sphinxlineitem{Return type}
\sphinxAtStartPar
str

\end{description}\end{quote}

\end{fulllineitems}

\index{get\_line() (nodes.nodePrinter.tablePrinter method)@\spxentry{get\_line()}\spxextra{nodes.nodePrinter.tablePrinter method}}

\begin{fulllineitems}
\phantomsection\label{\detokenize{nodes:nodes.nodePrinter.tablePrinter.get_line}}
\pysigstartsignatures
\pysiglinewithargsret
{\sphinxbfcode{\sphinxupquote{get\_line}}}
{\sphinxparam{\DUrole{n}{line\_type}\DUrole{p}{:}\DUrole{w}{ }\DUrole{n}{{\hyperref[\detokenize{nodes:nodes.nodePrinter.lineTypes}]{\sphinxcrossref{lineTypes}}}}}\sphinxparamcomma \sphinxparam{\DUrole{n}{char\_set}\DUrole{p}{:}\DUrole{w}{ }\DUrole{n}{str}}\sphinxparamcomma \sphinxparam{\DUrole{n}{widths}}\sphinxparamcomma \sphinxparam{\DUrole{n}{format\_data}\DUrole{o}{=}\DUrole{default_value}{None}}}
{}
\pysigstopsignatures
\sphinxAtStartPar
Get the line of the given type.
If self.print\_to\_console is True, print the line to the console.
If self.log\_file is not None, write the line to the file.
\begin{quote}\begin{description}
\sphinxlineitem{Parameters}\begin{itemize}
\item {} 
\sphinxAtStartPar
\sphinxstyleliteralstrong{\sphinxupquote{line\_type}} ({\hyperref[\detokenize{nodes:nodes.nodePrinter.lineTypes}]{\sphinxcrossref{\sphinxstyleliteralemphasis{\sphinxupquote{lineTypes}}}}}) \textendash{} The type of line to get.

\item {} 
\sphinxAtStartPar
\sphinxstyleliteralstrong{\sphinxupquote{char\_set}} (\sphinxstyleliteralemphasis{\sphinxupquote{str}}) \textendash{} The character set to use.

\item {} 
\sphinxAtStartPar
\sphinxstyleliteralstrong{\sphinxupquote{widths}} (\sphinxstyleliteralemphasis{\sphinxupquote{list}}) \textendash{} The widths of the columns.

\item {} 
\sphinxAtStartPar
\sphinxstyleliteralstrong{\sphinxupquote{format\_data}} (\sphinxstyleliteralemphasis{\sphinxupquote{list}}) \textendash{} The data to format.

\end{itemize}

\end{description}\end{quote}

\end{fulllineitems}

\index{get\_line\_chars() (nodes.nodePrinter.tablePrinter method)@\spxentry{get\_line\_chars()}\spxextra{nodes.nodePrinter.tablePrinter method}}

\begin{fulllineitems}
\phantomsection\label{\detokenize{nodes:nodes.nodePrinter.tablePrinter.get_line_chars}}
\pysigstartsignatures
\pysiglinewithargsret
{\sphinxbfcode{\sphinxupquote{get\_line\_chars}}}
{\sphinxparam{\DUrole{n}{line\_type}\DUrole{p}{:}\DUrole{w}{ }\DUrole{n}{{\hyperref[\detokenize{nodes:nodes.nodePrinter.lineTypes}]{\sphinxcrossref{lineTypes}}}}}\sphinxparamcomma \sphinxparam{\DUrole{n}{char\_set}\DUrole{p}{:}\DUrole{w}{ }\DUrole{n}{str}}}
{}
\pysigstopsignatures
\sphinxAtStartPar
Get the characters for the given line type and character set.

\end{fulllineitems}

\index{get\_line\_no\_data() (nodes.nodePrinter.tablePrinter method)@\spxentry{get\_line\_no\_data()}\spxextra{nodes.nodePrinter.tablePrinter method}}

\begin{fulllineitems}
\phantomsection\label{\detokenize{nodes:nodes.nodePrinter.tablePrinter.get_line_no_data}}
\pysigstartsignatures
\pysiglinewithargsret
{\sphinxbfcode{\sphinxupquote{get\_line\_no\_data}}}
{\sphinxparam{\DUrole{n}{line\_type}\DUrole{p}{:}\DUrole{w}{ }\DUrole{n}{{\hyperref[\detokenize{nodes:nodes.nodePrinter.lineTypes}]{\sphinxcrossref{lineTypes}}}}}\sphinxparamcomma \sphinxparam{\DUrole{n}{char\_set}\DUrole{p}{:}\DUrole{w}{ }\DUrole{n}{str}}\sphinxparamcomma \sphinxparam{\DUrole{n}{widths}}}
{}
\pysigstopsignatures
\sphinxAtStartPar
Return a string for the given line type.
:param line\_type: The type of line to get.
:type line\_type: lineTypes
:param char\_set: The character set to use.
:type char\_set: str
:param widths: The widths of the columns.
:type widths: list
\begin{quote}\begin{description}
\sphinxlineitem{Returns}
\sphinxAtStartPar
Line of the given type.

\sphinxlineitem{Return type}
\sphinxAtStartPar
str

\end{description}\end{quote}

\end{fulllineitems}

\index{get\_line\_with\_data() (nodes.nodePrinter.tablePrinter method)@\spxentry{get\_line\_with\_data()}\spxextra{nodes.nodePrinter.tablePrinter method}}

\begin{fulllineitems}
\phantomsection\label{\detokenize{nodes:nodes.nodePrinter.tablePrinter.get_line_with_data}}
\pysigstartsignatures
\pysiglinewithargsret
{\sphinxbfcode{\sphinxupquote{get\_line\_with\_data}}}
{\sphinxparam{\DUrole{n}{line\_type}\DUrole{p}{:}\DUrole{w}{ }\DUrole{n}{{\hyperref[\detokenize{nodes:nodes.nodePrinter.lineTypes}]{\sphinxcrossref{lineTypes}}}}}\sphinxparamcomma \sphinxparam{\DUrole{n}{char\_set}\DUrole{p}{:}\DUrole{w}{ }\DUrole{n}{str}}\sphinxparamcomma \sphinxparam{\DUrole{n}{widths}}\sphinxparamcomma \sphinxparam{\DUrole{n}{format\_data}}}
{}
\pysigstopsignatures
\sphinxAtStartPar
Return a string for the given row in the table.
:param line\_type: The type of line to get.
:type line\_type: lineTypes
:param char\_set: The character set to use.
:type char\_set: str
:param widths: The widths of the columns.
:type widths: list
:param format\_data: The data to format.
:type format\_data: list
\begin{quote}\begin{description}
\sphinxlineitem{Returns}
\sphinxAtStartPar
The line of the given type, with data.

\sphinxlineitem{Return type}
\sphinxAtStartPar
str

\end{description}\end{quote}

\end{fulllineitems}

\index{open\_file() (nodes.nodePrinter.tablePrinter method)@\spxentry{open\_file()}\spxextra{nodes.nodePrinter.tablePrinter method}}

\begin{fulllineitems}
\phantomsection\label{\detokenize{nodes:nodes.nodePrinter.tablePrinter.open_file}}
\pysigstartsignatures
\pysiglinewithargsret
{\sphinxbfcode{\sphinxupquote{open\_file}}}
{\sphinxparam{\DUrole{n}{filename}}}
{}
\pysigstopsignatures
\sphinxAtStartPar
Open the given file.
If the file exists, open it in append mode.
If the file does not exist, create it and open it in write mode.
\begin{quote}\begin{description}
\sphinxlineitem{Parameters}
\sphinxAtStartPar
\sphinxstyleliteralstrong{\sphinxupquote{filename}} (\sphinxstyleliteralemphasis{\sphinxupquote{str}}) \textendash{} The name of the file to open.

\sphinxlineitem{Returns}
\sphinxAtStartPar
The file object.

\sphinxlineitem{Return type}
\sphinxAtStartPar
file

\end{description}\end{quote}

\end{fulllineitems}

\index{print\_column\_names() (nodes.nodePrinter.tablePrinter method)@\spxentry{print\_column\_names()}\spxextra{nodes.nodePrinter.tablePrinter method}}

\begin{fulllineitems}
\phantomsection\label{\detokenize{nodes:nodes.nodePrinter.tablePrinter.print_column_names}}
\pysigstartsignatures
\pysiglinewithargsret
{\sphinxbfcode{\sphinxupquote{print\_column\_names}}}
{\sphinxparam{\DUrole{n}{split}\DUrole{o}{=}\DUrole{default_value}{True}}\sphinxparamcomma \sphinxparam{\DUrole{n}{char\_set}\DUrole{o}{=}\DUrole{default_value}{\textquotesingle{}table\textquotesingle{}}}}
{}
\pysigstopsignatures
\sphinxAtStartPar
Print the column names.
:param split: Whether to add a split line after column names, otherwise add a bottom line. Default is True.
:type split: bool
:param char\_set: The character set to use.
:type char\_set: str

\end{fulllineitems}

\index{print\_header() (nodes.nodePrinter.tablePrinter method)@\spxentry{print\_header()}\spxextra{nodes.nodePrinter.tablePrinter method}}

\begin{fulllineitems}
\phantomsection\label{\detokenize{nodes:nodes.nodePrinter.tablePrinter.print_header}}
\pysigstartsignatures
\pysiglinewithargsret
{\sphinxbfcode{\sphinxupquote{print\_header}}}
{\sphinxparam{\DUrole{n}{char\_set}\DUrole{o}{=}\DUrole{default_value}{\textquotesingle{}header\textquotesingle{}}}}
{}
\pysigstopsignatures
\sphinxAtStartPar
Print the header.
:param char\_set: The character set to use.
:type char\_set: str

\end{fulllineitems}

\index{print\_rows() (nodes.nodePrinter.tablePrinter method)@\spxentry{print\_rows()}\spxextra{nodes.nodePrinter.tablePrinter method}}

\begin{fulllineitems}
\phantomsection\label{\detokenize{nodes:nodes.nodePrinter.tablePrinter.print_rows}}
\pysigstartsignatures
\pysiglinewithargsret
{\sphinxbfcode{\sphinxupquote{print\_rows}}}
{\sphinxparam{\DUrole{n}{print\_top}\DUrole{o}{=}\DUrole{default_value}{False}}\sphinxparamcomma \sphinxparam{\DUrole{n}{print\_bottom}\DUrole{o}{=}\DUrole{default_value}{True}}\sphinxparamcomma \sphinxparam{\DUrole{n}{char\_set}\DUrole{o}{=}\DUrole{default_value}{\textquotesingle{}table\textquotesingle{}}}\sphinxparamcomma \sphinxparam{\DUrole{n}{split}\DUrole{o}{=}\DUrole{default_value}{False}}}
{}
\pysigstopsignatures
\sphinxAtStartPar
Print all rows of data in the table.
:param print\_top: Whether to print a top line. Default is False.
:type print\_top: bool
:param print\_bottom: Whether to print a bottom line. Default is True.
:type print\_bottom: bool
:param char\_set: The character set to use.
:type char\_set: str
:param split: Whether to add a split line after each row. Default is False.
:type split: bool

\end{fulllineitems}

\index{print\_table() (nodes.nodePrinter.tablePrinter method)@\spxentry{print\_table()}\spxextra{nodes.nodePrinter.tablePrinter method}}

\begin{fulllineitems}
\phantomsection\label{\detokenize{nodes:nodes.nodePrinter.tablePrinter.print_table}}
\pysigstartsignatures
\pysiglinewithargsret
{\sphinxbfcode{\sphinxupquote{print\_table}}}
{\sphinxparam{\DUrole{n}{header}\DUrole{o}{=}\DUrole{default_value}{True}}\sphinxparamcomma \sphinxparam{\DUrole{n}{column\_names}\DUrole{o}{=}\DUrole{default_value}{True}}\sphinxparamcomma \sphinxparam{\DUrole{n}{header\_char\_set}\DUrole{o}{=}\DUrole{default_value}{\textquotesingle{}header\textquotesingle{}}}\sphinxparamcomma \sphinxparam{\DUrole{n}{column\_char\_set}\DUrole{o}{=}\DUrole{default_value}{\textquotesingle{}table\textquotesingle{}}}\sphinxparamcomma \sphinxparam{\DUrole{n}{row\_char\_set}\DUrole{o}{=}\DUrole{default_value}{\textquotesingle{}table\textquotesingle{}}}\sphinxparamcomma \sphinxparam{\DUrole{n}{split}\DUrole{o}{=}\DUrole{default_value}{False}}}
{}
\pysigstopsignatures
\sphinxAtStartPar
Print the header, column names, and rows of data in the table.
:param header: Whether to print the header. Default is True.
:type header: bool
:param column\_names: Whether to print the column names. Default is True.
:type column\_names: bool
:param header\_char\_set: The character set to use for the header. Default is “header”.
:type header\_char\_set: str
:param column\_char\_set: The character set to use for the column names. Default is “table”.
:type column\_char\_set: str
:param row\_char\_set: The character set to use for the rows. Default is “table”.
:type row\_char\_set: str
:param split: Whether to add a split line after each row. Default is False.
:type split: bool

\end{fulllineitems}


\end{fulllineitems}



\section{nodes.nodeTensors module}
\label{\detokenize{nodes:module-nodes.nodeTensors}}\label{\detokenize{nodes:nodes-nodetensors-module}}\index{module@\spxentry{module}!nodes.nodeTensors@\spxentry{nodes.nodeTensors}}\index{nodes.nodeTensors@\spxentry{nodes.nodeTensors}!module@\spxentry{module}}\index{Driver (class in nodes.nodeTensors)@\spxentry{Driver}\spxextra{class in nodes.nodeTensors}}

\begin{fulllineitems}
\phantomsection\label{\detokenize{nodes:nodes.nodeTensors.Driver}}
\pysigstartsignatures
\pysiglinewithargsret
{\sphinxbfcode{\sphinxupquote{\DUrole{k}{class}\DUrole{w}{ }}}\sphinxcode{\sphinxupquote{nodes.nodeTensors.}}\sphinxbfcode{\sphinxupquote{Driver}}}
{\sphinxparam{\DUrole{n}{floatTensor}}\sphinxparamcomma \sphinxparam{\DUrole{n}{connections}}\sphinxparamcomma \sphinxparam{\DUrole{n}{links}}\sphinxparamcomma \sphinxparam{\DUrole{n}{names}\DUrole{p}{:}\DUrole{w}{ }\DUrole{n}{dict\DUrole{p}{{[}}int\DUrole{p}{,}\DUrole{w}{ }str\DUrole{p}{{]}}}\DUrole{w}{ }\DUrole{o}{=}\DUrole{w}{ }\DUrole{default_value}{None}}}
{}
\pysigstopsignatures
\sphinxAtStartPar
Bases: {\hyperref[\detokenize{nodes:nodes.nodeTensors.Tokens}]{\sphinxcrossref{\sphinxcode{\sphinxupquote{Tokens}}}}}

\sphinxAtStartPar
A class for representing the driver set of tokens.
\index{names (nodes.nodeTensors.Driver attribute)@\spxentry{names}\spxextra{nodes.nodeTensors.Driver attribute}}

\begin{fulllineitems}
\phantomsection\label{\detokenize{nodes:nodes.nodeTensors.Driver.names}}
\pysigstartsignatures
\pysigline
{\sphinxbfcode{\sphinxupquote{names}}}
\pysigstopsignatures
\sphinxAtStartPar
A dictionary mapping token IDs to token names. Defaults to None.
\begin{quote}\begin{description}
\sphinxlineitem{Type}
\sphinxAtStartPar
dict, optional

\end{description}\end{quote}

\end{fulllineitems}

\index{nodes (nodes.nodeTensors.Driver attribute)@\spxentry{nodes}\spxextra{nodes.nodeTensors.Driver attribute}}

\begin{fulllineitems}
\phantomsection\label{\detokenize{nodes:nodes.nodeTensors.Driver.nodes}}
\pysigstartsignatures
\pysigline
{\sphinxbfcode{\sphinxupquote{nodes}}}
\pysigstopsignatures
\sphinxAtStartPar
An NxTokenFeatures tensor of floats representing the tokens.
\begin{quote}\begin{description}
\sphinxlineitem{Type}
\sphinxAtStartPar
torch.Tensor

\end{description}\end{quote}

\end{fulllineitems}

\index{analogs (nodes.nodeTensors.Driver attribute)@\spxentry{analogs}\spxextra{nodes.nodeTensors.Driver attribute}}

\begin{fulllineitems}
\phantomsection\label{\detokenize{nodes:nodes.nodeTensors.Driver.analogs}}
\pysigstartsignatures
\pysigline
{\sphinxbfcode{\sphinxupquote{analogs}}}
\pysigstopsignatures
\sphinxAtStartPar
An Ax1 tensor listing all analogs in the tensor.
\begin{quote}\begin{description}
\sphinxlineitem{Type}
\sphinxAtStartPar
torch.Tensor

\end{description}\end{quote}

\end{fulllineitems}

\index{analog\_counts (nodes.nodeTensors.Driver attribute)@\spxentry{analog\_counts}\spxextra{nodes.nodeTensors.Driver attribute}}

\begin{fulllineitems}
\phantomsection\label{\detokenize{nodes:nodes.nodeTensors.Driver.analog_counts}}
\pysigstartsignatures
\pysigline
{\sphinxbfcode{\sphinxupquote{analog\_counts}}}
\pysigstopsignatures
\sphinxAtStartPar
An Ax1 tensor listing the number of tokens per analog
\begin{quote}\begin{description}
\sphinxlineitem{Type}
\sphinxAtStartPar
torch.Tensor

\end{description}\end{quote}

\end{fulllineitems}

\index{links (nodes.nodeTensors.Driver attribute)@\spxentry{links}\spxextra{nodes.nodeTensors.Driver attribute}}

\begin{fulllineitems}
\phantomsection\label{\detokenize{nodes:nodes.nodeTensors.Driver.links}}
\pysigstartsignatures
\pysigline
{\sphinxbfcode{\sphinxupquote{links}}}
\pysigstopsignatures
\sphinxAtStartPar
A Tensor of links from tokens in this set to the semantics.
\begin{quote}\begin{description}
\sphinxlineitem{Type}
\sphinxAtStartPar
torch.Tensor

\end{description}\end{quote}

\end{fulllineitems}

\index{connections (nodes.nodeTensors.Driver attribute)@\spxentry{connections}\spxextra{nodes.nodeTensors.Driver attribute}}

\begin{fulllineitems}
\phantomsection\label{\detokenize{nodes:nodes.nodeTensors.Driver.connections}}
\pysigstartsignatures
\pysigline
{\sphinxbfcode{\sphinxupquote{connections}}}
\pysigstopsignatures
\sphinxAtStartPar
An NxN tensor of connections from parent to child for tokens in this set.
\begin{quote}\begin{description}
\sphinxlineitem{Type}
\sphinxAtStartPar
torch.Tensor

\end{description}\end{quote}

\end{fulllineitems}

\index{masks (nodes.nodeTensors.Driver attribute)@\spxentry{masks}\spxextra{nodes.nodeTensors.Driver attribute}}

\begin{fulllineitems}
\phantomsection\label{\detokenize{nodes:nodes.nodeTensors.Driver.masks}}
\pysigstartsignatures
\pysigline
{\sphinxbfcode{\sphinxupquote{masks}}}
\pysigstopsignatures
\sphinxAtStartPar
A Tensor of masks for the tokens in this set.
\begin{quote}\begin{description}
\sphinxlineitem{Type}
\sphinxAtStartPar
torch.Tensor

\end{description}\end{quote}

\end{fulllineitems}

\index{check\_global\_inhibitor() (nodes.nodeTensors.Driver method)@\spxentry{check\_global\_inhibitor()}\spxextra{nodes.nodeTensors.Driver method}}

\begin{fulllineitems}
\phantomsection\label{\detokenize{nodes:nodes.nodeTensors.Driver.check_global_inhibitor}}
\pysigstartsignatures
\pysiglinewithargsret
{\sphinxbfcode{\sphinxupquote{check\_global\_inhibitor}}}
{}
{}
\pysigstopsignatures
\sphinxAtStartPar
Return true if any RB.inhibitor\_act == 1.0

\end{fulllineitems}

\index{check\_local\_inhibitor() (nodes.nodeTensors.Driver method)@\spxentry{check\_local\_inhibitor()}\spxextra{nodes.nodeTensors.Driver method}}

\begin{fulllineitems}
\phantomsection\label{\detokenize{nodes:nodes.nodeTensors.Driver.check_local_inhibitor}}
\pysigstartsignatures
\pysiglinewithargsret
{\sphinxbfcode{\sphinxupquote{check\_local\_inhibitor}}}
{}
{}
\pysigstopsignatures
\sphinxAtStartPar
Return true if any PO.inhibitor\_act == 1.0

\end{fulllineitems}

\index{update\_input() (nodes.nodeTensors.Driver method)@\spxentry{update\_input()}\spxextra{nodes.nodeTensors.Driver method}}

\begin{fulllineitems}
\phantomsection\label{\detokenize{nodes:nodes.nodeTensors.Driver.update_input}}
\pysigstartsignatures
\pysiglinewithargsret
{\sphinxbfcode{\sphinxupquote{update\_input}}}
{\sphinxparam{\DUrole{n}{as\_DORA}}}
{}
\pysigstopsignatures
\sphinxAtStartPar
Update all input in driver

\end{fulllineitems}

\index{update\_input\_p\_child() (nodes.nodeTensors.Driver method)@\spxentry{update\_input\_p\_child()}\spxextra{nodes.nodeTensors.Driver method}}

\begin{fulllineitems}
\phantomsection\label{\detokenize{nodes:nodes.nodeTensors.Driver.update_input_p_child}}
\pysigstartsignatures
\pysiglinewithargsret
{\sphinxbfcode{\sphinxupquote{update\_input\_p\_child}}}
{\sphinxparam{\DUrole{n}{as\_DORA}}}
{}
\pysigstopsignatures
\sphinxAtStartPar
Update input in driver for P units in child mode

\end{fulllineitems}

\index{update\_input\_p\_parent() (nodes.nodeTensors.Driver method)@\spxentry{update\_input\_p\_parent()}\spxextra{nodes.nodeTensors.Driver method}}

\begin{fulllineitems}
\phantomsection\label{\detokenize{nodes:nodes.nodeTensors.Driver.update_input_p_parent}}
\pysigstartsignatures
\pysiglinewithargsret
{\sphinxbfcode{\sphinxupquote{update\_input\_p\_parent}}}
{}
{}
\pysigstopsignatures
\sphinxAtStartPar
Update input in driver for P units in parent mode

\end{fulllineitems}

\index{update\_input\_po() (nodes.nodeTensors.Driver method)@\spxentry{update\_input\_po()}\spxextra{nodes.nodeTensors.Driver method}}

\begin{fulllineitems}
\phantomsection\label{\detokenize{nodes:nodes.nodeTensors.Driver.update_input_po}}
\pysigstartsignatures
\pysiglinewithargsret
{\sphinxbfcode{\sphinxupquote{update\_input\_po}}}
{\sphinxparam{\DUrole{n}{as\_DORA}}}
{}
\pysigstopsignatures
\sphinxAtStartPar
Update input in driver for PO units

\end{fulllineitems}

\index{update\_input\_rb() (nodes.nodeTensors.Driver method)@\spxentry{update\_input\_rb()}\spxextra{nodes.nodeTensors.Driver method}}

\begin{fulllineitems}
\phantomsection\label{\detokenize{nodes:nodes.nodeTensors.Driver.update_input_rb}}
\pysigstartsignatures
\pysiglinewithargsret
{\sphinxbfcode{\sphinxupquote{update\_input\_rb}}}
{\sphinxparam{\DUrole{n}{as\_DORA}}}
{}
\pysigstopsignatures
\sphinxAtStartPar
Update input in driver for RB units

\end{fulllineitems}


\end{fulllineitems}

\index{Recipient (class in nodes.nodeTensors)@\spxentry{Recipient}\spxextra{class in nodes.nodeTensors}}

\begin{fulllineitems}
\phantomsection\label{\detokenize{nodes:nodes.nodeTensors.Recipient}}
\pysigstartsignatures
\pysiglinewithargsret
{\sphinxbfcode{\sphinxupquote{\DUrole{k}{class}\DUrole{w}{ }}}\sphinxcode{\sphinxupquote{nodes.nodeTensors.}}\sphinxbfcode{\sphinxupquote{Recipient}}}
{\sphinxparam{\DUrole{n}{floatTensor}}\sphinxparamcomma \sphinxparam{\DUrole{n}{connections}}\sphinxparamcomma \sphinxparam{\DUrole{n}{links}}\sphinxparamcomma \sphinxparam{\DUrole{n}{names}\DUrole{o}{=}\DUrole{default_value}{None}}}
{}
\pysigstopsignatures
\sphinxAtStartPar
Bases: {\hyperref[\detokenize{nodes:nodes.nodeTensors.Tokens}]{\sphinxcrossref{\sphinxcode{\sphinxupquote{Tokens}}}}}

\sphinxAtStartPar
A class for representing the recipient set of tokens.
\index{names (nodes.nodeTensors.Recipient attribute)@\spxentry{names}\spxextra{nodes.nodeTensors.Recipient attribute}}

\begin{fulllineitems}
\phantomsection\label{\detokenize{nodes:nodes.nodeTensors.Recipient.names}}
\pysigstartsignatures
\pysigline
{\sphinxbfcode{\sphinxupquote{names}}}
\pysigstopsignatures
\sphinxAtStartPar
A dictionary mapping token IDs to token names. Defaults to None.
\begin{quote}\begin{description}
\sphinxlineitem{Type}
\sphinxAtStartPar
dict, optional

\end{description}\end{quote}

\end{fulllineitems}

\index{nodes (nodes.nodeTensors.Recipient attribute)@\spxentry{nodes}\spxextra{nodes.nodeTensors.Recipient attribute}}

\begin{fulllineitems}
\phantomsection\label{\detokenize{nodes:nodes.nodeTensors.Recipient.nodes}}
\pysigstartsignatures
\pysigline
{\sphinxbfcode{\sphinxupquote{nodes}}}
\pysigstopsignatures
\sphinxAtStartPar
An NxTokenFeatures tensor of floats representing the tokens.
\begin{quote}\begin{description}
\sphinxlineitem{Type}
\sphinxAtStartPar
torch.Tensor

\end{description}\end{quote}

\end{fulllineitems}

\index{analogs (nodes.nodeTensors.Recipient attribute)@\spxentry{analogs}\spxextra{nodes.nodeTensors.Recipient attribute}}

\begin{fulllineitems}
\phantomsection\label{\detokenize{nodes:nodes.nodeTensors.Recipient.analogs}}
\pysigstartsignatures
\pysigline
{\sphinxbfcode{\sphinxupquote{analogs}}}
\pysigstopsignatures
\sphinxAtStartPar
An Ax1 tensor listing all analogs in the tensor.
\begin{quote}\begin{description}
\sphinxlineitem{Type}
\sphinxAtStartPar
torch.Tensor

\end{description}\end{quote}

\end{fulllineitems}

\index{analog\_counts (nodes.nodeTensors.Recipient attribute)@\spxentry{analog\_counts}\spxextra{nodes.nodeTensors.Recipient attribute}}

\begin{fulllineitems}
\phantomsection\label{\detokenize{nodes:nodes.nodeTensors.Recipient.analog_counts}}
\pysigstartsignatures
\pysigline
{\sphinxbfcode{\sphinxupquote{analog\_counts}}}
\pysigstopsignatures
\sphinxAtStartPar
An Ax1 tensor listing the number of tokens per analog
\begin{quote}\begin{description}
\sphinxlineitem{Type}
\sphinxAtStartPar
torch.Tensor

\end{description}\end{quote}

\end{fulllineitems}

\index{links (nodes.nodeTensors.Recipient attribute)@\spxentry{links}\spxextra{nodes.nodeTensors.Recipient attribute}}

\begin{fulllineitems}
\phantomsection\label{\detokenize{nodes:nodes.nodeTensors.Recipient.links}}
\pysigstartsignatures
\pysigline
{\sphinxbfcode{\sphinxupquote{links}}}
\pysigstopsignatures
\sphinxAtStartPar
A Tensor of links between tokens in this set and the semantics.
\begin{quote}\begin{description}
\sphinxlineitem{Type}
\sphinxAtStartPar
torch.Tensor

\end{description}\end{quote}

\end{fulllineitems}

\index{connections (nodes.nodeTensors.Recipient attribute)@\spxentry{connections}\spxextra{nodes.nodeTensors.Recipient attribute}}

\begin{fulllineitems}
\phantomsection\label{\detokenize{nodes:nodes.nodeTensors.Recipient.connections}}
\pysigstartsignatures
\pysigline
{\sphinxbfcode{\sphinxupquote{connections}}}
\pysigstopsignatures
\sphinxAtStartPar
An NxN tensor of connections from parent to child for tokens in this set.
\begin{quote}\begin{description}
\sphinxlineitem{Type}
\sphinxAtStartPar
torch.Tensor

\end{description}\end{quote}

\end{fulllineitems}

\index{map\_input() (nodes.nodeTensors.Recipient method)@\spxentry{map\_input()}\spxextra{nodes.nodeTensors.Recipient method}}

\begin{fulllineitems}
\phantomsection\label{\detokenize{nodes:nodes.nodeTensors.Recipient.map_input}}
\pysigstartsignatures
\pysiglinewithargsret
{\sphinxbfcode{\sphinxupquote{map\_input}}}
{\sphinxparam{\DUrole{n}{t\_mask}}\sphinxparamcomma \sphinxparam{\DUrole{n}{mappings}\DUrole{p}{:}\DUrole{w}{ }\DUrole{n}{{\hyperref[\detokenize{nodes:nodes.nodeMemObjects.Mappings}]{\sphinxcrossref{Mappings}}}}}\sphinxparamcomma \sphinxparam{\DUrole{n}{driver}\DUrole{p}{:}\DUrole{w}{ }\DUrole{n}{{\hyperref[\detokenize{nodes:nodes.nodeTensors.Driver}]{\sphinxcrossref{Driver}}}}}}
{}
\pysigstopsignatures
\sphinxAtStartPar
Calculate mapping input for tokens in mask
\begin{quote}\begin{description}
\sphinxlineitem{Parameters}\begin{itemize}
\item {} 
\sphinxAtStartPar
\sphinxstyleliteralstrong{\sphinxupquote{t\_mask}} (\sphinxstyleliteralemphasis{\sphinxupquote{torch.Tensor}}) \textendash{} A mask of tokens to calculate mapping input for

\item {} 
\sphinxAtStartPar
\sphinxstyleliteralstrong{\sphinxupquote{mappings}} ({\hyperref[\detokenize{nodes:nodes.nodeMemObjects.Mappings}]{\sphinxcrossref{\sphinxstyleliteralemphasis{\sphinxupquote{Mappings}}}}}) \textendash{} A Mappings object

\item {} 
\sphinxAtStartPar
\sphinxstyleliteralstrong{\sphinxupquote{driver}} ({\hyperref[\detokenize{nodes:nodes.nodeTensors.Driver}]{\sphinxcrossref{\sphinxstyleliteralemphasis{\sphinxupquote{Driver}}}}}) \textendash{} A Driver object

\end{itemize}

\sphinxlineitem{Returns}
\sphinxAtStartPar
A (sum(t\_mask) x 1) matrix of mapping input for tokens in mask

\sphinxlineitem{Return type}
\sphinxAtStartPar
torch.Tensor

\end{description}\end{quote}

\end{fulllineitems}

\index{update\_input() (nodes.nodeTensors.Recipient method)@\spxentry{update\_input()}\spxextra{nodes.nodeTensors.Recipient method}}

\begin{fulllineitems}
\phantomsection\label{\detokenize{nodes:nodes.nodeTensors.Recipient.update_input}}
\pysigstartsignatures
\pysiglinewithargsret
{\sphinxbfcode{\sphinxupquote{update\_input}}}
{\sphinxparam{\DUrole{n}{as\_DORA}}\sphinxparamcomma \sphinxparam{\DUrole{n}{phase\_set}}\sphinxparamcomma \sphinxparam{\DUrole{n}{lateral\_input\_level}}\sphinxparamcomma \sphinxparam{\DUrole{n}{semantics}}\sphinxparamcomma \sphinxparam{\DUrole{n}{mappings}}\sphinxparamcomma \sphinxparam{\DUrole{n}{driver}}\sphinxparamcomma \sphinxparam{\DUrole{n}{ignore\_object\_semantics}\DUrole{o}{=}\DUrole{default_value}{False}}}
{}
\pysigstopsignatures
\sphinxAtStartPar
Update all input in recipient
\begin{quote}\begin{description}
\sphinxlineitem{Parameters}\begin{itemize}
\item {} 
\sphinxAtStartPar
\sphinxstyleliteralstrong{\sphinxupquote{as\_DORA}} (\sphinxstyleliteralemphasis{\sphinxupquote{bool}}) \textendash{} Whether to use DORA mode

\item {} 
\sphinxAtStartPar
\sphinxstyleliteralstrong{\sphinxupquote{phase\_set}} (\sphinxstyleliteralemphasis{\sphinxupquote{int}}) \textendash{} The current phase set

\item {} 
\sphinxAtStartPar
\sphinxstyleliteralstrong{\sphinxupquote{lateral\_input\_level}} (\sphinxstyleliteralemphasis{\sphinxupquote{float}}) \textendash{} The level of lateral input

\item {} 
\sphinxAtStartPar
\sphinxstyleliteralstrong{\sphinxupquote{semantics}} (\sphinxstyleliteralemphasis{\sphinxupquote{Semantics}}) \textendash{} A Semantics object

\item {} 
\sphinxAtStartPar
\sphinxstyleliteralstrong{\sphinxupquote{mappings}} ({\hyperref[\detokenize{nodes:nodes.nodeMemObjects.Mappings}]{\sphinxcrossref{\sphinxstyleliteralemphasis{\sphinxupquote{Mappings}}}}}) \textendash{} A Mappings object

\item {} 
\sphinxAtStartPar
\sphinxstyleliteralstrong{\sphinxupquote{driver}} ({\hyperref[\detokenize{nodes:nodes.nodeTensors.Driver}]{\sphinxcrossref{\sphinxstyleliteralemphasis{\sphinxupquote{Driver}}}}}) \textendash{} A Driver object

\item {} 
\sphinxAtStartPar
\sphinxstyleliteralstrong{\sphinxupquote{ignore\_object\_semantics}} (\sphinxstyleliteralemphasis{\sphinxupquote{bool}}) \textendash{} Whether to ignore object semantics

\end{itemize}

\end{description}\end{quote}

\end{fulllineitems}

\index{update\_input\_p\_child() (nodes.nodeTensors.Recipient method)@\spxentry{update\_input\_p\_child()}\spxextra{nodes.nodeTensors.Recipient method}}

\begin{fulllineitems}
\phantomsection\label{\detokenize{nodes:nodes.nodeTensors.Recipient.update_input_p_child}}
\pysigstartsignatures
\pysiglinewithargsret
{\sphinxbfcode{\sphinxupquote{update\_input\_p\_child}}}
{\sphinxparam{\DUrole{n}{as\_DORA}}\sphinxparamcomma \sphinxparam{\DUrole{n}{phase\_set}}\sphinxparamcomma \sphinxparam{\DUrole{n}{lateral\_input\_level}}\sphinxparamcomma \sphinxparam{\DUrole{n}{mappings}\DUrole{p}{:}\DUrole{w}{ }\DUrole{n}{{\hyperref[\detokenize{nodes:nodes.nodeMemObjects.Mappings}]{\sphinxcrossref{Mappings}}}}}\sphinxparamcomma \sphinxparam{\DUrole{n}{driver}\DUrole{p}{:}\DUrole{w}{ }\DUrole{n}{{\hyperref[\detokenize{nodes:nodes.nodeTensors.Driver}]{\sphinxcrossref{Driver}}}}}}
{}
\pysigstopsignatures
\sphinxAtStartPar
Update input for P units in child mode
\begin{quote}\begin{description}
\sphinxlineitem{Parameters}\begin{itemize}
\item {} 
\sphinxAtStartPar
\sphinxstyleliteralstrong{\sphinxupquote{as\_DORA}} (\sphinxstyleliteralemphasis{\sphinxupquote{bool}}) \textendash{} Whether to use DORA mode

\item {} 
\sphinxAtStartPar
\sphinxstyleliteralstrong{\sphinxupquote{phase\_set}} (\sphinxstyleliteralemphasis{\sphinxupquote{int}}) \textendash{} The current phase set

\item {} 
\sphinxAtStartPar
\sphinxstyleliteralstrong{\sphinxupquote{lateral\_input\_level}} (\sphinxstyleliteralemphasis{\sphinxupquote{float}}) \textendash{} The level of lateral input

\item {} 
\sphinxAtStartPar
\sphinxstyleliteralstrong{\sphinxupquote{mappings}} ({\hyperref[\detokenize{nodes:nodes.nodeMemObjects.Mappings}]{\sphinxcrossref{\sphinxstyleliteralemphasis{\sphinxupquote{Mappings}}}}}) \textendash{} A Mappings object

\item {} 
\sphinxAtStartPar
\sphinxstyleliteralstrong{\sphinxupquote{driver}} ({\hyperref[\detokenize{nodes:nodes.nodeTensors.Driver}]{\sphinxcrossref{\sphinxstyleliteralemphasis{\sphinxupquote{Driver}}}}}) \textendash{} A Driver object

\end{itemize}

\end{description}\end{quote}

\end{fulllineitems}

\index{update\_input\_p\_parent() (nodes.nodeTensors.Recipient method)@\spxentry{update\_input\_p\_parent()}\spxextra{nodes.nodeTensors.Recipient method}}

\begin{fulllineitems}
\phantomsection\label{\detokenize{nodes:nodes.nodeTensors.Recipient.update_input_p_parent}}
\pysigstartsignatures
\pysiglinewithargsret
{\sphinxbfcode{\sphinxupquote{update\_input\_p\_parent}}}
{\sphinxparam{\DUrole{n}{phase\_set}}\sphinxparamcomma \sphinxparam{\DUrole{n}{lateral\_input\_level}}\sphinxparamcomma \sphinxparam{\DUrole{n}{mappings}\DUrole{p}{:}\DUrole{w}{ }\DUrole{n}{{\hyperref[\detokenize{nodes:nodes.nodeMemObjects.Mappings}]{\sphinxcrossref{Mappings}}}}}\sphinxparamcomma \sphinxparam{\DUrole{n}{driver}\DUrole{p}{:}\DUrole{w}{ }\DUrole{n}{{\hyperref[\detokenize{nodes:nodes.nodeTensors.Driver}]{\sphinxcrossref{Driver}}}}}}
{}
\pysigstopsignatures
\sphinxAtStartPar
Update input for P units in parent mode
\begin{quote}\begin{description}
\sphinxlineitem{Parameters}\begin{itemize}
\item {} 
\sphinxAtStartPar
\sphinxstyleliteralstrong{\sphinxupquote{phase\_set}} (\sphinxstyleliteralemphasis{\sphinxupquote{int}}) \textendash{} The current phase set

\item {} 
\sphinxAtStartPar
\sphinxstyleliteralstrong{\sphinxupquote{lateral\_input\_level}} (\sphinxstyleliteralemphasis{\sphinxupquote{float}}) \textendash{} The level of lateral input

\item {} 
\sphinxAtStartPar
\sphinxstyleliteralstrong{\sphinxupquote{mappings}} ({\hyperref[\detokenize{nodes:nodes.nodeMemObjects.Mappings}]{\sphinxcrossref{\sphinxstyleliteralemphasis{\sphinxupquote{Mappings}}}}}) \textendash{} A Mappings object

\item {} 
\sphinxAtStartPar
\sphinxstyleliteralstrong{\sphinxupquote{driver}} ({\hyperref[\detokenize{nodes:nodes.nodeTensors.Driver}]{\sphinxcrossref{\sphinxstyleliteralemphasis{\sphinxupquote{Driver}}}}}) \textendash{} A Driver object

\end{itemize}

\end{description}\end{quote}

\end{fulllineitems}

\index{update\_input\_po() (nodes.nodeTensors.Recipient method)@\spxentry{update\_input\_po()}\spxextra{nodes.nodeTensors.Recipient method}}

\begin{fulllineitems}
\phantomsection\label{\detokenize{nodes:nodes.nodeTensors.Recipient.update_input_po}}
\pysigstartsignatures
\pysiglinewithargsret
{\sphinxbfcode{\sphinxupquote{update\_input\_po}}}
{\sphinxparam{\DUrole{n}{as\_DORA}}\sphinxparamcomma \sphinxparam{\DUrole{n}{phase\_set}}\sphinxparamcomma \sphinxparam{\DUrole{n}{lateral\_input\_level}}\sphinxparamcomma \sphinxparam{\DUrole{n}{semantics}}\sphinxparamcomma \sphinxparam{\DUrole{n}{mappings}}\sphinxparamcomma \sphinxparam{\DUrole{n}{driver}}\sphinxparamcomma \sphinxparam{\DUrole{n}{ignore\_object\_semantics}\DUrole{o}{=}\DUrole{default_value}{False}}}
{}
\pysigstopsignatures
\sphinxAtStartPar
Update input for PO units
\begin{quote}\begin{description}
\sphinxlineitem{Parameters}\begin{itemize}
\item {} 
\sphinxAtStartPar
\sphinxstyleliteralstrong{\sphinxupquote{as\_DORA}} (\sphinxstyleliteralemphasis{\sphinxupquote{bool}}) \textendash{} Whether to use DORA mode

\item {} 
\sphinxAtStartPar
\sphinxstyleliteralstrong{\sphinxupquote{phase\_set}} (\sphinxstyleliteralemphasis{\sphinxupquote{int}}) \textendash{} The current phase set

\item {} 
\sphinxAtStartPar
\sphinxstyleliteralstrong{\sphinxupquote{lateral\_input\_level}} (\sphinxstyleliteralemphasis{\sphinxupquote{float}}) \textendash{} The level of lateral input

\end{itemize}

\end{description}\end{quote}

\end{fulllineitems}

\index{update\_input\_rb() (nodes.nodeTensors.Recipient method)@\spxentry{update\_input\_rb()}\spxextra{nodes.nodeTensors.Recipient method}}

\begin{fulllineitems}
\phantomsection\label{\detokenize{nodes:nodes.nodeTensors.Recipient.update_input_rb}}
\pysigstartsignatures
\pysiglinewithargsret
{\sphinxbfcode{\sphinxupquote{update\_input\_rb}}}
{\sphinxparam{\DUrole{n}{phase\_set}}\sphinxparamcomma \sphinxparam{\DUrole{n}{lateral\_input\_level}}\sphinxparamcomma \sphinxparam{\DUrole{n}{mappings}\DUrole{p}{:}\DUrole{w}{ }\DUrole{n}{{\hyperref[\detokenize{nodes:nodes.nodeMemObjects.Mappings}]{\sphinxcrossref{Mappings}}}}}\sphinxparamcomma \sphinxparam{\DUrole{n}{driver}\DUrole{p}{:}\DUrole{w}{ }\DUrole{n}{{\hyperref[\detokenize{nodes:nodes.nodeTensors.Driver}]{\sphinxcrossref{Driver}}}}}}
{}
\pysigstopsignatures
\sphinxAtStartPar
Update input for RB units
\begin{quote}\begin{description}
\sphinxlineitem{Parameters}\begin{itemize}
\item {} 
\sphinxAtStartPar
\sphinxstyleliteralstrong{\sphinxupquote{phase\_set}} (\sphinxstyleliteralemphasis{\sphinxupquote{int}}) \textendash{} The current phase set

\item {} 
\sphinxAtStartPar
\sphinxstyleliteralstrong{\sphinxupquote{lateral\_input\_level}} (\sphinxstyleliteralemphasis{\sphinxupquote{float}}) \textendash{} The level of lateral input

\item {} 
\sphinxAtStartPar
\sphinxstyleliteralstrong{\sphinxupquote{mappings}} ({\hyperref[\detokenize{nodes:nodes.nodeMemObjects.Mappings}]{\sphinxcrossref{\sphinxstyleliteralemphasis{\sphinxupquote{Mappings}}}}}) \textendash{} A Mappings object

\item {} 
\sphinxAtStartPar
\sphinxstyleliteralstrong{\sphinxupquote{driver}} ({\hyperref[\detokenize{nodes:nodes.nodeTensors.Driver}]{\sphinxcrossref{\sphinxstyleliteralemphasis{\sphinxupquote{Driver}}}}}) \textendash{} A Driver object

\end{itemize}

\end{description}\end{quote}

\end{fulllineitems}


\end{fulllineitems}

\index{Semantic (class in nodes.nodeTensors)@\spxentry{Semantic}\spxextra{class in nodes.nodeTensors}}

\begin{fulllineitems}
\phantomsection\label{\detokenize{nodes:nodes.nodeTensors.Semantic}}
\pysigstartsignatures
\pysiglinewithargsret
{\sphinxbfcode{\sphinxupquote{\DUrole{k}{class}\DUrole{w}{ }}}\sphinxcode{\sphinxupquote{nodes.nodeTensors.}}\sphinxbfcode{\sphinxupquote{Semantic}}}
{\sphinxparam{\DUrole{n}{nodes}}\sphinxparamcomma \sphinxparam{\DUrole{n}{connections}}\sphinxparamcomma \sphinxparam{\DUrole{n}{links}\DUrole{p}{:}\DUrole{w}{ }\DUrole{n}{{\hyperref[\detokenize{nodes:nodes.nodeMemObjects.Links}]{\sphinxcrossref{Links}}}}}\sphinxparamcomma \sphinxparam{\DUrole{n}{names}\DUrole{o}{=}\DUrole{default_value}{None}}}
{}
\pysigstopsignatures
\sphinxAtStartPar
Bases: \sphinxcode{\sphinxupquote{object}}

\sphinxAtStartPar
A class for representing semantics nodes.
\index{names (nodes.nodeTensors.Semantic attribute)@\spxentry{names}\spxextra{nodes.nodeTensors.Semantic attribute}}

\begin{fulllineitems}
\phantomsection\label{\detokenize{nodes:nodes.nodeTensors.Semantic.names}}
\pysigstartsignatures
\pysigline
{\sphinxbfcode{\sphinxupquote{names}}}
\pysigstopsignatures
\sphinxAtStartPar
A dictionary mapping semantic IDs to semantic names. Defaults to None.
\begin{quote}\begin{description}
\sphinxlineitem{Type}
\sphinxAtStartPar
dict, optional

\end{description}\end{quote}

\end{fulllineitems}

\index{nodes (nodes.nodeTensors.Semantic attribute)@\spxentry{nodes}\spxextra{nodes.nodeTensors.Semantic attribute}}

\begin{fulllineitems}
\phantomsection\label{\detokenize{nodes:nodes.nodeTensors.Semantic.nodes}}
\pysigstartsignatures
\pysigline
{\sphinxbfcode{\sphinxupquote{nodes}}}
\pysigstopsignatures
\sphinxAtStartPar
An NxSemanticFeatures tensor of floats representing the semantics.
\begin{quote}\begin{description}
\sphinxlineitem{Type}
\sphinxAtStartPar
torch.Tensor

\end{description}\end{quote}

\end{fulllineitems}

\index{connections (nodes.nodeTensors.Semantic attribute)@\spxentry{connections}\spxextra{nodes.nodeTensors.Semantic attribute}}

\begin{fulllineitems}
\phantomsection\label{\detokenize{nodes:nodes.nodeTensors.Semantic.connections}}
\pysigstartsignatures
\pysigline
{\sphinxbfcode{\sphinxupquote{connections}}}
\pysigstopsignatures
\sphinxAtStartPar
An NxN tensor of connections from parent to child for semantics in this set.
\begin{quote}\begin{description}
\sphinxlineitem{Type}
\sphinxAtStartPar
torch.Tensor

\end{description}\end{quote}

\end{fulllineitems}

\index{links (nodes.nodeTensors.Semantic attribute)@\spxentry{links}\spxextra{nodes.nodeTensors.Semantic attribute}}

\begin{fulllineitems}
\phantomsection\label{\detokenize{nodes:nodes.nodeTensors.Semantic.links}}
\pysigstartsignatures
\pysigline
{\sphinxbfcode{\sphinxupquote{links}}}
\pysigstopsignatures
\sphinxAtStartPar
A Links object containing links from token sets to semantics.
\begin{quote}\begin{description}
\sphinxlineitem{Type}
\sphinxAtStartPar
{\hyperref[\detokenize{nodes:nodes.nodeMemObjects.Links}]{\sphinxcrossref{Links}}}

\end{description}\end{quote}

\end{fulllineitems}

\index{initialise\_input() (nodes.nodeTensors.Semantic method)@\spxentry{initialise\_input()}\spxextra{nodes.nodeTensors.Semantic method}}

\begin{fulllineitems}
\phantomsection\label{\detokenize{nodes:nodes.nodeTensors.Semantic.initialise_input}}
\pysigstartsignatures
\pysiglinewithargsret
{\sphinxbfcode{\sphinxupquote{initialise\_input}}}
{\sphinxparam{\DUrole{n}{refresh}}}
{}
\pysigstopsignatures
\sphinxAtStartPar
Initialise the input of the semantics

\end{fulllineitems}

\index{intitialse\_sem() (nodes.nodeTensors.Semantic method)@\spxentry{intitialse\_sem()}\spxextra{nodes.nodeTensors.Semantic method}}

\begin{fulllineitems}
\phantomsection\label{\detokenize{nodes:nodes.nodeTensors.Semantic.intitialse_sem}}
\pysigstartsignatures
\pysiglinewithargsret
{\sphinxbfcode{\sphinxupquote{intitialse\_sem}}}
{}
{}
\pysigstopsignatures
\sphinxAtStartPar
Initialise the semantics

\end{fulllineitems}

\index{set\_max\_input() (nodes.nodeTensors.Semantic method)@\spxentry{set\_max\_input()}\spxextra{nodes.nodeTensors.Semantic method}}

\begin{fulllineitems}
\phantomsection\label{\detokenize{nodes:nodes.nodeTensors.Semantic.set_max_input}}
\pysigstartsignatures
\pysiglinewithargsret
{\sphinxbfcode{\sphinxupquote{set\_max\_input}}}
{\sphinxparam{\DUrole{n}{max\_input}}}
{}
\pysigstopsignatures
\sphinxAtStartPar
Set the max input of the semantics

\end{fulllineitems}

\index{update\_act() (nodes.nodeTensors.Semantic method)@\spxentry{update\_act()}\spxextra{nodes.nodeTensors.Semantic method}}

\begin{fulllineitems}
\phantomsection\label{\detokenize{nodes:nodes.nodeTensors.Semantic.update_act}}
\pysigstartsignatures
\pysiglinewithargsret
{\sphinxbfcode{\sphinxupquote{update\_act}}}
{}
{}
\pysigstopsignatures
\sphinxAtStartPar
Update the acts of the semantics

\end{fulllineitems}

\index{update\_input() (nodes.nodeTensors.Semantic method)@\spxentry{update\_input()}\spxextra{nodes.nodeTensors.Semantic method}}

\begin{fulllineitems}
\phantomsection\label{\detokenize{nodes:nodes.nodeTensors.Semantic.update_input}}
\pysigstartsignatures
\pysiglinewithargsret
{\sphinxbfcode{\sphinxupquote{update\_input}}}
{\sphinxparam{\DUrole{n}{driver}}\sphinxparamcomma \sphinxparam{\DUrole{n}{recipient}}\sphinxparamcomma \sphinxparam{\DUrole{n}{memory}\DUrole{o}{=}\DUrole{default_value}{None}}\sphinxparamcomma \sphinxparam{\DUrole{n}{ignore\_obj}\DUrole{o}{=}\DUrole{default_value}{False}}\sphinxparamcomma \sphinxparam{\DUrole{n}{ignore\_mem}\DUrole{o}{=}\DUrole{default_value}{False}}}
{}
\pysigstopsignatures
\sphinxAtStartPar
Update the input of the semantics

\end{fulllineitems}

\index{update\_input\_from\_set() (nodes.nodeTensors.Semantic method)@\spxentry{update\_input\_from\_set()}\spxextra{nodes.nodeTensors.Semantic method}}

\begin{fulllineitems}
\phantomsection\label{\detokenize{nodes:nodes.nodeTensors.Semantic.update_input_from_set}}
\pysigstartsignatures
\pysiglinewithargsret
{\sphinxbfcode{\sphinxupquote{update\_input\_from\_set}}}
{\sphinxparam{\DUrole{n}{tensor}\DUrole{p}{:}\DUrole{w}{ }\DUrole{n}{{\hyperref[\detokenize{nodes:nodes.nodeTensors.Tokens}]{\sphinxcrossref{Tokens}}}}}\sphinxparamcomma \sphinxparam{\DUrole{n}{set}\DUrole{p}{:}\DUrole{w}{ }\DUrole{n}{{\hyperref[\detokenize{nodes:nodes.nodeEnums.Set}]{\sphinxcrossref{Set}}}}}\sphinxparamcomma \sphinxparam{\DUrole{n}{ignore\_obj}\DUrole{o}{=}\DUrole{default_value}{False}}}
{}
\pysigstopsignatures
\sphinxAtStartPar
Update the input of the semantics from a set of tokens

\end{fulllineitems}


\end{fulllineitems}

\index{Tokens (class in nodes.nodeTensors)@\spxentry{Tokens}\spxextra{class in nodes.nodeTensors}}

\begin{fulllineitems}
\phantomsection\label{\detokenize{nodes:nodes.nodeTensors.Tokens}}
\pysigstartsignatures
\pysiglinewithargsret
{\sphinxbfcode{\sphinxupquote{\DUrole{k}{class}\DUrole{w}{ }}}\sphinxcode{\sphinxupquote{nodes.nodeTensors.}}\sphinxbfcode{\sphinxupquote{Tokens}}}
{\sphinxparam{\DUrole{n}{floatTensor}}\sphinxparamcomma \sphinxparam{\DUrole{n}{connections}}\sphinxparamcomma \sphinxparam{\DUrole{n}{links}}\sphinxparamcomma \sphinxparam{\DUrole{n}{names}\DUrole{p}{:}\DUrole{w}{ }\DUrole{n}{dict\DUrole{p}{{[}}int\DUrole{p}{,}\DUrole{w}{ }str\DUrole{p}{{]}}}\DUrole{w}{ }\DUrole{o}{=}\DUrole{w}{ }\DUrole{default_value}{None}}}
{}
\pysigstopsignatures
\sphinxAtStartPar
Bases: \sphinxcode{\sphinxupquote{object}}

\sphinxAtStartPar
A class for holding a tensor of tokens, and performing general tensor operations.
\index{names (nodes.nodeTensors.Tokens attribute)@\spxentry{names}\spxextra{nodes.nodeTensors.Tokens attribute}}

\begin{fulllineitems}
\phantomsection\label{\detokenize{nodes:nodes.nodeTensors.Tokens.names}}
\pysigstartsignatures
\pysigline
{\sphinxbfcode{\sphinxupquote{names}}}
\pysigstopsignatures
\sphinxAtStartPar
A dictionary mapping token IDs to token names. Defaults to None.
\begin{quote}\begin{description}
\sphinxlineitem{Type}
\sphinxAtStartPar
dict, optional

\end{description}\end{quote}

\end{fulllineitems}

\index{nodes (nodes.nodeTensors.Tokens attribute)@\spxentry{nodes}\spxextra{nodes.nodeTensors.Tokens attribute}}

\begin{fulllineitems}
\phantomsection\label{\detokenize{nodes:nodes.nodeTensors.Tokens.nodes}}
\pysigstartsignatures
\pysigline
{\sphinxbfcode{\sphinxupquote{nodes}}}
\pysigstopsignatures
\sphinxAtStartPar
An NxTokenFeatures tensor of floats representing the tokens.
\begin{quote}\begin{description}
\sphinxlineitem{Type}
\sphinxAtStartPar
torch.Tensor

\end{description}\end{quote}

\end{fulllineitems}

\index{analogs (nodes.nodeTensors.Tokens attribute)@\spxentry{analogs}\spxextra{nodes.nodeTensors.Tokens attribute}}

\begin{fulllineitems}
\phantomsection\label{\detokenize{nodes:nodes.nodeTensors.Tokens.analogs}}
\pysigstartsignatures
\pysigline
{\sphinxbfcode{\sphinxupquote{analogs}}}
\pysigstopsignatures
\sphinxAtStartPar
An Ax1 tensor listing all analogs in the tensor.
\begin{quote}\begin{description}
\sphinxlineitem{Type}
\sphinxAtStartPar
torch.Tensor

\end{description}\end{quote}

\end{fulllineitems}

\index{analog\_counts (nodes.nodeTensors.Tokens attribute)@\spxentry{analog\_counts}\spxextra{nodes.nodeTensors.Tokens attribute}}

\begin{fulllineitems}
\phantomsection\label{\detokenize{nodes:nodes.nodeTensors.Tokens.analog_counts}}
\pysigstartsignatures
\pysigline
{\sphinxbfcode{\sphinxupquote{analog\_counts}}}
\pysigstopsignatures
\sphinxAtStartPar
An Ax1 tensor listing the number of tokens per analog
\begin{quote}\begin{description}
\sphinxlineitem{Type}
\sphinxAtStartPar
torch.Tensor

\end{description}\end{quote}

\end{fulllineitems}

\index{links (nodes.nodeTensors.Tokens attribute)@\spxentry{links}\spxextra{nodes.nodeTensors.Tokens attribute}}

\begin{fulllineitems}
\phantomsection\label{\detokenize{nodes:nodes.nodeTensors.Tokens.links}}
\pysigstartsignatures
\pysigline
{\sphinxbfcode{\sphinxupquote{links}}}
\pysigstopsignatures
\sphinxAtStartPar
A Tensor of links from tokens in this set to the semantics.
\begin{quote}\begin{description}
\sphinxlineitem{Type}
\sphinxAtStartPar
torch.Tensor

\end{description}\end{quote}

\end{fulllineitems}

\index{connections (nodes.nodeTensors.Tokens attribute)@\spxentry{connections}\spxextra{nodes.nodeTensors.Tokens attribute}}

\begin{fulllineitems}
\phantomsection\label{\detokenize{nodes:nodes.nodeTensors.Tokens.connections}}
\pysigstartsignatures
\pysigline
{\sphinxbfcode{\sphinxupquote{connections}}}
\pysigstopsignatures
\sphinxAtStartPar
An NxN tensor of connections from parent to child for tokens in this set.
\begin{quote}\begin{description}
\sphinxlineitem{Type}
\sphinxAtStartPar
torch.Tensor

\end{description}\end{quote}

\end{fulllineitems}

\index{masks (nodes.nodeTensors.Tokens attribute)@\spxentry{masks}\spxextra{nodes.nodeTensors.Tokens attribute}}

\begin{fulllineitems}
\phantomsection\label{\detokenize{nodes:nodes.nodeTensors.Tokens.masks}}
\pysigstartsignatures
\pysigline
{\sphinxbfcode{\sphinxupquote{masks}}}
\pysigstopsignatures
\sphinxAtStartPar
A Tensor of masks for the tokens in this set.
\begin{quote}\begin{description}
\sphinxlineitem{Type}
\sphinxAtStartPar
torch.Tensor

\end{description}\end{quote}

\end{fulllineitems}

\index{add\_nodes() (nodes.nodeTensors.Tokens method)@\spxentry{add\_nodes()}\spxextra{nodes.nodeTensors.Tokens method}}

\begin{fulllineitems}
\phantomsection\label{\detokenize{nodes:nodes.nodeTensors.Tokens.add_nodes}}
\pysigstartsignatures
\pysiglinewithargsret
{\sphinxbfcode{\sphinxupquote{add\_nodes}}}
{\sphinxparam{\DUrole{n}{nodes}}}
{}
\pysigstopsignatures
\sphinxAtStartPar
Add nodes to tensor

\end{fulllineitems}

\index{analog\_node\_count() (nodes.nodeTensors.Tokens method)@\spxentry{analog\_node\_count()}\spxextra{nodes.nodeTensors.Tokens method}}

\begin{fulllineitems}
\phantomsection\label{\detokenize{nodes:nodes.nodeTensors.Tokens.analog_node_count}}
\pysigstartsignatures
\pysiglinewithargsret
{\sphinxbfcode{\sphinxupquote{analog\_node\_count}}}
{}
{}
\pysigstopsignatures
\sphinxAtStartPar
Update list of analogs in tensor, and their node counts

\end{fulllineitems}

\index{cache\_masks() (nodes.nodeTensors.Tokens method)@\spxentry{cache\_masks()}\spxextra{nodes.nodeTensors.Tokens method}}

\begin{fulllineitems}
\phantomsection\label{\detokenize{nodes:nodes.nodeTensors.Tokens.cache_masks}}
\pysigstartsignatures
\pysiglinewithargsret
{\sphinxbfcode{\sphinxupquote{cache\_masks}}}
{\sphinxparam{\DUrole{n}{types\_to\_recompute}\DUrole{o}{=}\DUrole{default_value}{None}}}
{}
\pysigstopsignatures
\sphinxAtStartPar
Compute and cache masks, specify types to recompute via list of tokenTypes

\end{fulllineitems}

\index{compute\_mask() (nodes.nodeTensors.Tokens method)@\spxentry{compute\_mask()}\spxextra{nodes.nodeTensors.Tokens method}}

\begin{fulllineitems}
\phantomsection\label{\detokenize{nodes:nodes.nodeTensors.Tokens.compute_mask}}
\pysigstartsignatures
\pysiglinewithargsret
{\sphinxbfcode{\sphinxupquote{compute\_mask}}}
{\sphinxparam{\DUrole{n}{token\_type}\DUrole{p}{:}\DUrole{w}{ }\DUrole{n}{{\hyperref[\detokenize{nodes:nodes.nodeEnums.Type}]{\sphinxcrossref{Type}}}}}}
{}
\pysigstopsignatures
\sphinxAtStartPar
Compute the mask for a token type

\end{fulllineitems}

\index{del\_Nodes() (nodes.nodeTensors.Tokens method)@\spxentry{del\_Nodes()}\spxextra{nodes.nodeTensors.Tokens method}}

\begin{fulllineitems}
\phantomsection\label{\detokenize{nodes:nodes.nodeTensors.Tokens.del_Nodes}}
\pysigstartsignatures
\pysiglinewithargsret
{\sphinxbfcode{\sphinxupquote{del\_Nodes}}}
{\sphinxparam{\DUrole{n}{nodes}}}
{}
\pysigstopsignatures
\sphinxAtStartPar
Delete nodes from tensor

\end{fulllineitems}

\index{get\_combined\_mask() (nodes.nodeTensors.Tokens method)@\spxentry{get\_combined\_mask()}\spxextra{nodes.nodeTensors.Tokens method}}

\begin{fulllineitems}
\phantomsection\label{\detokenize{nodes:nodes.nodeTensors.Tokens.get_combined_mask}}
\pysigstartsignatures
\pysiglinewithargsret
{\sphinxbfcode{\sphinxupquote{get\_combined\_mask}}}
{\sphinxparam{\DUrole{n}{n\_types}\DUrole{p}{:}\DUrole{w}{ }\DUrole{n}{list\DUrole{p}{{[}}{\hyperref[\detokenize{nodes:nodes.nodeEnums.Type}]{\sphinxcrossref{Type}}}\DUrole{p}{{]}}}}}
{}
\pysigstopsignatures
\sphinxAtStartPar
Return combined mask of given types

\end{fulllineitems}

\index{get\_mask() (nodes.nodeTensors.Tokens method)@\spxentry{get\_mask()}\spxextra{nodes.nodeTensors.Tokens method}}

\begin{fulllineitems}
\phantomsection\label{\detokenize{nodes:nodes.nodeTensors.Tokens.get_mask}}
\pysigstartsignatures
\pysiglinewithargsret
{\sphinxbfcode{\sphinxupquote{get\_mask}}}
{\sphinxparam{\DUrole{n}{token\_type}\DUrole{p}{:}\DUrole{w}{ }\DUrole{n}{{\hyperref[\detokenize{nodes:nodes.nodeEnums.Type}]{\sphinxcrossref{Type}}}}}}
{}
\pysigstopsignatures
\sphinxAtStartPar
Return mask for given token type

\end{fulllineitems}

\index{initialise\_act() (nodes.nodeTensors.Tokens method)@\spxentry{initialise\_act()}\spxextra{nodes.nodeTensors.Tokens method}}

\begin{fulllineitems}
\phantomsection\label{\detokenize{nodes:nodes.nodeTensors.Tokens.initialise_act}}
\pysigstartsignatures
\pysiglinewithargsret
{\sphinxbfcode{\sphinxupquote{initialise\_act}}}
{\sphinxparam{\DUrole{n}{n\_type}\DUrole{p}{:}\DUrole{w}{ }\DUrole{n}{list\DUrole{p}{{[}}{\hyperref[\detokenize{nodes:nodes.nodeEnums.Type}]{\sphinxcrossref{Type}}}\DUrole{p}{{]}}}}}
{}
\pysigstopsignatures
\sphinxAtStartPar
Initialize act to 0.0,  and call initialise\_inputs
\begin{quote}\begin{description}
\sphinxlineitem{Parameters}
\sphinxAtStartPar
\sphinxstyleliteralstrong{\sphinxupquote{n\_type}} (\sphinxstyleliteralemphasis{\sphinxupquote{list}}\sphinxstyleliteralemphasis{\sphinxupquote{{[}}}{\hyperref[\detokenize{nodes:nodes.nodeEnums.Type}]{\sphinxcrossref{\sphinxstyleliteralemphasis{\sphinxupquote{Type}}}}}\sphinxstyleliteralemphasis{\sphinxupquote{{]}}}) \textendash{} The types of nodes to initialise.

\end{description}\end{quote}

\end{fulllineitems}

\index{initialise\_float() (nodes.nodeTensors.Tokens method)@\spxentry{initialise\_float()}\spxextra{nodes.nodeTensors.Tokens method}}

\begin{fulllineitems}
\phantomsection\label{\detokenize{nodes:nodes.nodeTensors.Tokens.initialise_float}}
\pysigstartsignatures
\pysiglinewithargsret
{\sphinxbfcode{\sphinxupquote{initialise\_float}}}
{\sphinxparam{\DUrole{n}{n\_type}\DUrole{p}{:}\DUrole{w}{ }\DUrole{n}{list\DUrole{p}{{[}}{\hyperref[\detokenize{nodes:nodes.nodeEnums.Type}]{\sphinxcrossref{Type}}}\DUrole{p}{{]}}}}\sphinxparamcomma \sphinxparam{\DUrole{n}{features}\DUrole{p}{:}\DUrole{w}{ }\DUrole{n}{list\DUrole{p}{{[}}{\hyperref[\detokenize{nodes:nodes.nodeEnums.TF}]{\sphinxcrossref{TF}}}\DUrole{p}{{]}}}}}
{}
\pysigstopsignatures
\sphinxAtStartPar
Initialise given features
\begin{quote}\begin{description}
\sphinxlineitem{Parameters}\begin{itemize}
\item {} 
\sphinxAtStartPar
\sphinxstyleliteralstrong{\sphinxupquote{n\_type}} (\sphinxstyleliteralemphasis{\sphinxupquote{list}}\sphinxstyleliteralemphasis{\sphinxupquote{{[}}}{\hyperref[\detokenize{nodes:nodes.nodeEnums.Type}]{\sphinxcrossref{\sphinxstyleliteralemphasis{\sphinxupquote{Type}}}}}\sphinxstyleliteralemphasis{\sphinxupquote{{]}}}) \textendash{} The types of nodes to initialise.

\item {} 
\sphinxAtStartPar
\sphinxstyleliteralstrong{\sphinxupquote{features}} (\sphinxstyleliteralemphasis{\sphinxupquote{list}}\sphinxstyleliteralemphasis{\sphinxupquote{{[}}}{\hyperref[\detokenize{nodes:nodes.nodeEnums.TF}]{\sphinxcrossref{\sphinxstyleliteralemphasis{\sphinxupquote{TF}}}}}\sphinxstyleliteralemphasis{\sphinxupquote{{]}}}) \textendash{} The features to initialise.

\end{itemize}

\end{description}\end{quote}

\end{fulllineitems}

\index{initialise\_input() (nodes.nodeTensors.Tokens method)@\spxentry{initialise\_input()}\spxextra{nodes.nodeTensors.Tokens method}}

\begin{fulllineitems}
\phantomsection\label{\detokenize{nodes:nodes.nodeTensors.Tokens.initialise_input}}
\pysigstartsignatures
\pysiglinewithargsret
{\sphinxbfcode{\sphinxupquote{initialise\_input}}}
{\sphinxparam{\DUrole{n}{n\_type}\DUrole{p}{:}\DUrole{w}{ }\DUrole{n}{list\DUrole{p}{{[}}{\hyperref[\detokenize{nodes:nodes.nodeEnums.Type}]{\sphinxcrossref{Type}}}\DUrole{p}{{]}}}}\sphinxparamcomma \sphinxparam{\DUrole{n}{refresh}\DUrole{p}{:}\DUrole{w}{ }\DUrole{n}{float}}}
{}
\pysigstopsignatures
\sphinxAtStartPar
Initialize inputs to 0, and td\_input to refresh
\begin{quote}\begin{description}
\sphinxlineitem{Parameters}\begin{itemize}
\item {} 
\sphinxAtStartPar
\sphinxstyleliteralstrong{\sphinxupquote{n\_type}} (\sphinxstyleliteralemphasis{\sphinxupquote{list}}\sphinxstyleliteralemphasis{\sphinxupquote{{[}}}{\hyperref[\detokenize{nodes:nodes.nodeEnums.Type}]{\sphinxcrossref{\sphinxstyleliteralemphasis{\sphinxupquote{Type}}}}}\sphinxstyleliteralemphasis{\sphinxupquote{{]}}}) \textendash{} The types of nodes to initialise.

\item {} 
\sphinxAtStartPar
\sphinxstyleliteralstrong{\sphinxupquote{refresh}} (\sphinxstyleliteralemphasis{\sphinxupquote{float}}) \textendash{} The value to set the td\_input to.

\end{itemize}

\end{description}\end{quote}

\end{fulllineitems}

\index{initialise\_state() (nodes.nodeTensors.Tokens method)@\spxentry{initialise\_state()}\spxextra{nodes.nodeTensors.Tokens method}}

\begin{fulllineitems}
\phantomsection\label{\detokenize{nodes:nodes.nodeTensors.Tokens.initialise_state}}
\pysigstartsignatures
\pysiglinewithargsret
{\sphinxbfcode{\sphinxupquote{initialise\_state}}}
{\sphinxparam{\DUrole{n}{n\_type}\DUrole{p}{:}\DUrole{w}{ }\DUrole{n}{list\DUrole{p}{{[}}{\hyperref[\detokenize{nodes:nodes.nodeEnums.Type}]{\sphinxcrossref{Type}}}\DUrole{p}{{]}}}}}
{}
\pysigstopsignatures
\sphinxAtStartPar
Set self.retrieved to false, and call initialise\_act
\begin{quote}\begin{description}
\sphinxlineitem{Parameters}
\sphinxAtStartPar
\sphinxstyleliteralstrong{\sphinxupquote{n\_type}} (\sphinxstyleliteralemphasis{\sphinxupquote{list}}\sphinxstyleliteralemphasis{\sphinxupquote{{[}}}{\hyperref[\detokenize{nodes:nodes.nodeEnums.Type}]{\sphinxcrossref{\sphinxstyleliteralemphasis{\sphinxupquote{Type}}}}}\sphinxstyleliteralemphasis{\sphinxupquote{{]}}}) \textendash{} The types of nodes to initialise.

\end{description}\end{quote}

\end{fulllineitems}

\index{p\_get\_mode() (nodes.nodeTensors.Tokens method)@\spxentry{p\_get\_mode()}\spxextra{nodes.nodeTensors.Tokens method}}

\begin{fulllineitems}
\phantomsection\label{\detokenize{nodes:nodes.nodeTensors.Tokens.p_get_mode}}
\pysigstartsignatures
\pysiglinewithargsret
{\sphinxbfcode{\sphinxupquote{p\_get\_mode}}}
{}
{}
\pysigstopsignatures
\sphinxAtStartPar
Set mode for all P units

\end{fulllineitems}

\index{p\_initialise\_mode() (nodes.nodeTensors.Tokens method)@\spxentry{p\_initialise\_mode()}\spxextra{nodes.nodeTensors.Tokens method}}

\begin{fulllineitems}
\phantomsection\label{\detokenize{nodes:nodes.nodeTensors.Tokens.p_initialise_mode}}
\pysigstartsignatures
\pysiglinewithargsret
{\sphinxbfcode{\sphinxupquote{p\_initialise\_mode}}}
{}
{}
\pysigstopsignatures
\sphinxAtStartPar
Initialize mode to neutral for all P units.

\end{fulllineitems}

\index{po\_get\_max\_semantic\_weight() (nodes.nodeTensors.Tokens method)@\spxentry{po\_get\_max\_semantic\_weight()}\spxextra{nodes.nodeTensors.Tokens method}}

\begin{fulllineitems}
\phantomsection\label{\detokenize{nodes:nodes.nodeTensors.Tokens.po_get_max_semantic_weight}}
\pysigstartsignatures
\pysiglinewithargsret
{\sphinxbfcode{\sphinxupquote{po\_get\_max\_semantic\_weight}}}
{}
{}
\pysigstopsignatures
\sphinxAtStartPar
Set max link weight feature for all PO nodes

\end{fulllineitems}

\index{po\_get\_weight\_length() (nodes.nodeTensors.Tokens method)@\spxentry{po\_get\_weight\_length()}\spxextra{nodes.nodeTensors.Tokens method}}

\begin{fulllineitems}
\phantomsection\label{\detokenize{nodes:nodes.nodeTensors.Tokens.po_get_weight_length}}
\pysigstartsignatures
\pysiglinewithargsret
{\sphinxbfcode{\sphinxupquote{po\_get\_weight\_length}}}
{}
{}
\pysigstopsignatures
\sphinxAtStartPar
Set sem count feature for all PO nodes

\end{fulllineitems}

\index{reset\_inhibitor() (nodes.nodeTensors.Tokens method)@\spxentry{reset\_inhibitor()}\spxextra{nodes.nodeTensors.Tokens method}}

\begin{fulllineitems}
\phantomsection\label{\detokenize{nodes:nodes.nodeTensors.Tokens.reset_inhibitor}}
\pysigstartsignatures
\pysiglinewithargsret
{\sphinxbfcode{\sphinxupquote{reset\_inhibitor}}}
{\sphinxparam{\DUrole{n}{n\_type}\DUrole{p}{:}\DUrole{w}{ }\DUrole{n}{list\DUrole{p}{{[}}{\hyperref[\detokenize{nodes:nodes.nodeEnums.Type}]{\sphinxcrossref{Type}}}\DUrole{p}{{]}}}}}
{}
\pysigstopsignatures
\sphinxAtStartPar
Reset the inhibitor input and act to 0.0 for given type
\begin{quote}\begin{description}
\sphinxlineitem{Parameters}
\sphinxAtStartPar
\sphinxstyleliteralstrong{\sphinxupquote{n\_type}} (\sphinxstyleliteralemphasis{\sphinxupquote{list}}\sphinxstyleliteralemphasis{\sphinxupquote{{[}}}{\hyperref[\detokenize{nodes:nodes.nodeEnums.Type}]{\sphinxcrossref{\sphinxstyleliteralemphasis{\sphinxupquote{Type}}}}}\sphinxstyleliteralemphasis{\sphinxupquote{{]}}}) \textendash{} The types of nodes to reset inhibitor inputs and acts.

\end{description}\end{quote}

\end{fulllineitems}

\index{update\_act() (nodes.nodeTensors.Tokens method)@\spxentry{update\_act()}\spxextra{nodes.nodeTensors.Tokens method}}

\begin{fulllineitems}
\phantomsection\label{\detokenize{nodes:nodes.nodeTensors.Tokens.update_act}}
\pysigstartsignatures
\pysiglinewithargsret
{\sphinxbfcode{\sphinxupquote{update\_act}}}
{\sphinxparam{\DUrole{n}{gamma}\DUrole{p}{:}\DUrole{w}{ }\DUrole{n}{float}}\sphinxparamcomma \sphinxparam{\DUrole{n}{delta}\DUrole{p}{:}\DUrole{w}{ }\DUrole{n}{float}}\sphinxparamcomma \sphinxparam{\DUrole{n}{HebbBias}\DUrole{p}{:}\DUrole{w}{ }\DUrole{n}{float}}}
{}
\pysigstopsignatures
\sphinxAtStartPar
Update act of nodes
\begin{quote}\begin{description}
\sphinxlineitem{Parameters}\begin{itemize}
\item {} 
\sphinxAtStartPar
\sphinxstyleliteralstrong{\sphinxupquote{gamma}} (\sphinxstyleliteralemphasis{\sphinxupquote{float}}) \textendash{} Effects the increase in act for each unit.

\item {} 
\sphinxAtStartPar
\sphinxstyleliteralstrong{\sphinxupquote{delta}} (\sphinxstyleliteralemphasis{\sphinxupquote{float}}) \textendash{} Effects the decrease in act for each unit.

\item {} 
\sphinxAtStartPar
\sphinxstyleliteralstrong{\sphinxupquote{HebbBias}} (\sphinxstyleliteralemphasis{\sphinxupquote{float}}) \textendash{} The bias for mapping input relative to TD/BU/LATERAL inputs.

\end{itemize}

\end{description}\end{quote}

\end{fulllineitems}

\index{update\_inhibitor\_act() (nodes.nodeTensors.Tokens method)@\spxentry{update\_inhibitor\_act()}\spxextra{nodes.nodeTensors.Tokens method}}

\begin{fulllineitems}
\phantomsection\label{\detokenize{nodes:nodes.nodeTensors.Tokens.update_inhibitor_act}}
\pysigstartsignatures
\pysiglinewithargsret
{\sphinxbfcode{\sphinxupquote{update\_inhibitor\_act}}}
{\sphinxparam{\DUrole{n}{n\_type}\DUrole{p}{:}\DUrole{w}{ }\DUrole{n}{list\DUrole{p}{{[}}{\hyperref[\detokenize{nodes:nodes.nodeEnums.Type}]{\sphinxcrossref{Type}}}\DUrole{p}{{]}}}}}
{}
\pysigstopsignatures
\sphinxAtStartPar
Update the inhibitor act for given type
\begin{quote}\begin{description}
\sphinxlineitem{Parameters}
\sphinxAtStartPar
\sphinxstyleliteralstrong{\sphinxupquote{n\_type}} (\sphinxstyleliteralemphasis{\sphinxupquote{list}}\sphinxstyleliteralemphasis{\sphinxupquote{{[}}}{\hyperref[\detokenize{nodes:nodes.nodeEnums.Type}]{\sphinxcrossref{\sphinxstyleliteralemphasis{\sphinxupquote{Type}}}}}\sphinxstyleliteralemphasis{\sphinxupquote{{]}}}) \textendash{} The types of nodes to update inhibitor acts.

\end{description}\end{quote}

\end{fulllineitems}

\index{update\_inhibitor\_input() (nodes.nodeTensors.Tokens method)@\spxentry{update\_inhibitor\_input()}\spxextra{nodes.nodeTensors.Tokens method}}

\begin{fulllineitems}
\phantomsection\label{\detokenize{nodes:nodes.nodeTensors.Tokens.update_inhibitor_input}}
\pysigstartsignatures
\pysiglinewithargsret
{\sphinxbfcode{\sphinxupquote{update\_inhibitor\_input}}}
{\sphinxparam{\DUrole{n}{n\_type}\DUrole{p}{:}\DUrole{w}{ }\DUrole{n}{list\DUrole{p}{{[}}{\hyperref[\detokenize{nodes:nodes.nodeEnums.Type}]{\sphinxcrossref{Type}}}\DUrole{p}{{]}}}}}
{}
\pysigstopsignatures
\sphinxAtStartPar
Update inputs to inhibitors by current activation for nodes of type n\_type
\begin{quote}\begin{description}
\sphinxlineitem{Parameters}
\sphinxAtStartPar
\sphinxstyleliteralstrong{\sphinxupquote{n\_type}} (\sphinxstyleliteralemphasis{\sphinxupquote{list}}\sphinxstyleliteralemphasis{\sphinxupquote{{[}}}{\hyperref[\detokenize{nodes:nodes.nodeEnums.Type}]{\sphinxcrossref{\sphinxstyleliteralemphasis{\sphinxupquote{Type}}}}}\sphinxstyleliteralemphasis{\sphinxupquote{{]}}}) \textendash{} The types of nodes to update inhibitor inputs.

\end{description}\end{quote}

\end{fulllineitems}

\index{zero\_lateral\_input() (nodes.nodeTensors.Tokens method)@\spxentry{zero\_lateral\_input()}\spxextra{nodes.nodeTensors.Tokens method}}

\begin{fulllineitems}
\phantomsection\label{\detokenize{nodes:nodes.nodeTensors.Tokens.zero_lateral_input}}
\pysigstartsignatures
\pysiglinewithargsret
{\sphinxbfcode{\sphinxupquote{zero\_lateral\_input}}}
{\sphinxparam{\DUrole{n}{n\_type}\DUrole{p}{:}\DUrole{w}{ }\DUrole{n}{list\DUrole{p}{{[}}{\hyperref[\detokenize{nodes:nodes.nodeEnums.Type}]{\sphinxcrossref{Type}}}\DUrole{p}{{]}}}}}
{}
\pysigstopsignatures
\sphinxAtStartPar
Set lateral\_input to 0;
to allow synchrony at different levels by 0\sphinxhyphen{}ing lateral inhibition at that level
(e.g., to bind via synchrony, 0 lateral inhibition in POs).
\begin{quote}\begin{description}
\sphinxlineitem{Parameters}
\sphinxAtStartPar
\sphinxstyleliteralstrong{\sphinxupquote{n\_type}} (\sphinxstyleliteralemphasis{\sphinxupquote{list}}\sphinxstyleliteralemphasis{\sphinxupquote{{[}}}{\hyperref[\detokenize{nodes:nodes.nodeEnums.Type}]{\sphinxcrossref{\sphinxstyleliteralemphasis{\sphinxupquote{Type}}}}}\sphinxstyleliteralemphasis{\sphinxupquote{{]}}}) \textendash{} The types of nodes to set lateral\_input to 0.

\end{description}\end{quote}

\end{fulllineitems}


\end{fulllineitems}



\section{nodes.nodes module}
\label{\detokenize{nodes:module-nodes.nodes}}\label{\detokenize{nodes:nodes-nodes-module}}\index{module@\spxentry{module}!nodes.nodes@\spxentry{nodes.nodes}}\index{nodes.nodes@\spxentry{nodes.nodes}!module@\spxentry{module}}\index{Nodes (class in nodes.nodes)@\spxentry{Nodes}\spxextra{class in nodes.nodes}}

\begin{fulllineitems}
\phantomsection\label{\detokenize{nodes:nodes.nodes.Nodes}}
\pysigstartsignatures
\pysiglinewithargsret
{\sphinxbfcode{\sphinxupquote{\DUrole{k}{class}\DUrole{w}{ }}}\sphinxcode{\sphinxupquote{nodes.nodes.}}\sphinxbfcode{\sphinxupquote{Nodes}}}
{\sphinxparam{\DUrole{n}{driver}\DUrole{p}{:}\DUrole{w}{ }\DUrole{n}{{\hyperref[\detokenize{nodes:nodes.nodeTensors.Driver}]{\sphinxcrossref{Driver}}}}}\sphinxparamcomma \sphinxparam{\DUrole{n}{recipient}\DUrole{p}{:}\DUrole{w}{ }\DUrole{n}{{\hyperref[\detokenize{nodes:nodes.nodeTensors.Recipient}]{\sphinxcrossref{Recipient}}}}}\sphinxparamcomma \sphinxparam{\DUrole{n}{LTM}\DUrole{p}{:}\DUrole{w}{ }\DUrole{n}{{\hyperref[\detokenize{nodes:nodes.nodeTensors.Tokens}]{\sphinxcrossref{Tokens}}}}}\sphinxparamcomma \sphinxparam{\DUrole{n}{new\_set}\DUrole{p}{:}\DUrole{w}{ }\DUrole{n}{{\hyperref[\detokenize{nodes:nodes.nodeTensors.Tokens}]{\sphinxcrossref{Tokens}}}}}\sphinxparamcomma \sphinxparam{\DUrole{n}{semantics}\DUrole{p}{:}\DUrole{w}{ }\DUrole{n}{{\hyperref[\detokenize{nodes:nodes.nodeTensors.Semantic}]{\sphinxcrossref{Semantic}}}}}\sphinxparamcomma \sphinxparam{\DUrole{n}{mappings}\DUrole{p}{:}\DUrole{w}{ }\DUrole{n}{{\hyperref[\detokenize{nodes:nodes.nodeMemObjects.Mappings}]{\sphinxcrossref{Mappings}}}}}\sphinxparamcomma \sphinxparam{\DUrole{n}{DORA\_mode}\DUrole{p}{:}\DUrole{w}{ }\DUrole{n}{bool}}}
{}
\pysigstopsignatures
\sphinxAtStartPar
Bases: \sphinxcode{\sphinxupquote{object}}

\sphinxAtStartPar
A class for holding token tensors for each set, and accessing node operations.
\index{checkDriverPOs() (nodes.nodes.Nodes method)@\spxentry{checkDriverPOs()}\spxextra{nodes.nodes.Nodes method}}

\begin{fulllineitems}
\phantomsection\label{\detokenize{nodes:nodes.nodes.Nodes.checkDriverPOs}}
\pysigstartsignatures
\pysiglinewithargsret
{\sphinxbfcode{\sphinxupquote{checkDriverPOs}}}
{}
{}
\pysigstopsignatures
\sphinxAtStartPar
Check local inhibitor activation.

\end{fulllineitems}

\index{checkDriverRBs() (nodes.nodes.Nodes method)@\spxentry{checkDriverRBs()}\spxextra{nodes.nodes.Nodes method}}

\begin{fulllineitems}
\phantomsection\label{\detokenize{nodes:nodes.nodes.Nodes.checkDriverRBs}}
\pysigstartsignatures
\pysiglinewithargsret
{\sphinxbfcode{\sphinxupquote{checkDriverRBs}}}
{}
{}
\pysigstopsignatures
\sphinxAtStartPar
Check global inhibitor activation.

\end{fulllineitems}

\index{fire\_global\_inhibitor() (nodes.nodes.Nodes method)@\spxentry{fire\_global\_inhibitor()}\spxextra{nodes.nodes.Nodes method}}

\begin{fulllineitems}
\phantomsection\label{\detokenize{nodes:nodes.nodes.Nodes.fire_global_inhibitor}}
\pysigstartsignatures
\pysiglinewithargsret
{\sphinxbfcode{\sphinxupquote{fire\_global\_inhibitor}}}
{}
{}
\pysigstopsignatures
\sphinxAtStartPar
Fire the global inhibitor.

\end{fulllineitems}

\index{fire\_local\_inhibitor() (nodes.nodes.Nodes method)@\spxentry{fire\_local\_inhibitor()}\spxextra{nodes.nodes.Nodes method}}

\begin{fulllineitems}
\phantomsection\label{\detokenize{nodes:nodes.nodes.Nodes.fire_local_inhibitor}}
\pysigstartsignatures
\pysiglinewithargsret
{\sphinxbfcode{\sphinxupquote{fire\_local\_inhibitor}}}
{}
{}
\pysigstopsignatures
\sphinxAtStartPar
Fire the local inhibitor.

\end{fulllineitems}

\index{update\_acts\_am() (nodes.nodes.Nodes method)@\spxentry{update\_acts\_am()}\spxextra{nodes.nodes.Nodes method}}

\begin{fulllineitems}
\phantomsection\label{\detokenize{nodes:nodes.nodes.Nodes.update_acts_am}}
\pysigstartsignatures
\pysiglinewithargsret
{\sphinxbfcode{\sphinxupquote{update\_acts\_am}}}
{\sphinxparam{\DUrole{n}{gamma}}\sphinxparamcomma \sphinxparam{\DUrole{n}{delta}}\sphinxparamcomma \sphinxparam{\DUrole{n}{hebb\_bias}}}
{}
\pysigstopsignatures
\sphinxAtStartPar
Update the acts in the active memory.

\end{fulllineitems}

\index{update\_acts\_driver() (nodes.nodes.Nodes method)@\spxentry{update\_acts\_driver()}\spxextra{nodes.nodes.Nodes method}}

\begin{fulllineitems}
\phantomsection\label{\detokenize{nodes:nodes.nodes.Nodes.update_acts_driver}}
\pysigstartsignatures
\pysiglinewithargsret
{\sphinxbfcode{\sphinxupquote{update\_acts\_driver}}}
{\sphinxparam{\DUrole{n}{gamma}}\sphinxparamcomma \sphinxparam{\DUrole{n}{delta}}\sphinxparamcomma \sphinxparam{\DUrole{n}{hebb\_bias}}}
{}
\pysigstopsignatures
\sphinxAtStartPar
Update the acts in the driver.

\end{fulllineitems}

\index{update\_acts\_recipient() (nodes.nodes.Nodes method)@\spxentry{update\_acts\_recipient()}\spxextra{nodes.nodes.Nodes method}}

\begin{fulllineitems}
\phantomsection\label{\detokenize{nodes:nodes.nodes.Nodes.update_acts_recipient}}
\pysigstartsignatures
\pysiglinewithargsret
{\sphinxbfcode{\sphinxupquote{update\_acts\_recipient}}}
{\sphinxparam{\DUrole{n}{gamma}}\sphinxparamcomma \sphinxparam{\DUrole{n}{delta}}\sphinxparamcomma \sphinxparam{\DUrole{n}{hebb\_bias}}}
{}
\pysigstopsignatures
\sphinxAtStartPar
Update the acts in the recipient.

\end{fulllineitems}

\index{update\_inputs\_am() (nodes.nodes.Nodes method)@\spxentry{update\_inputs\_am()}\spxextra{nodes.nodes.Nodes method}}

\begin{fulllineitems}
\phantomsection\label{\detokenize{nodes:nodes.nodes.Nodes.update_inputs_am}}
\pysigstartsignatures
\pysiglinewithargsret
{\sphinxbfcode{\sphinxupquote{update\_inputs\_am}}}
{\sphinxparam{\DUrole{n}{as\_DORA}}\sphinxparamcomma \sphinxparam{\DUrole{n}{phase\_set}}\sphinxparamcomma \sphinxparam{\DUrole{n}{lateral\_input\_level}}\sphinxparamcomma \sphinxparam{\DUrole{n}{ignore\_object\_semantics}\DUrole{o}{=}\DUrole{default_value}{False}}}
{}
\pysigstopsignatures
\sphinxAtStartPar
Update the inputs in the active memory.
\begin{quote}\begin{description}
\sphinxlineitem{Parameters}\begin{itemize}
\item {} 
\sphinxAtStartPar
\sphinxstyleliteralstrong{\sphinxupquote{as\_DORA}} (\sphinxstyleliteralemphasis{\sphinxupquote{bool}}) \textendash{} Whether to use DORA mode.

\item {} 
\sphinxAtStartPar
\sphinxstyleliteralstrong{\sphinxupquote{phase\_set}} (\sphinxstyleliteralemphasis{\sphinxupquote{Int}}) \textendash{} The current phase set.

\item {} 
\sphinxAtStartPar
\sphinxstyleliteralstrong{\sphinxupquote{lateral\_input\_level}} (\sphinxstyleliteralemphasis{\sphinxupquote{float}}) \textendash{} The lateral input level.

\item {} 
\sphinxAtStartPar
\sphinxstyleliteralstrong{\sphinxupquote{ignore\_object\_semantics}} (\sphinxstyleliteralemphasis{\sphinxupquote{bool}}\sphinxstyleliteralemphasis{\sphinxupquote{, }}\sphinxstyleliteralemphasis{\sphinxupquote{optional}}) \textendash{} Whether to ignore object semantics input. Defaults to False.

\end{itemize}

\end{description}\end{quote}

\end{fulllineitems}

\index{update\_inputs\_driver() (nodes.nodes.Nodes method)@\spxentry{update\_inputs\_driver()}\spxextra{nodes.nodes.Nodes method}}

\begin{fulllineitems}
\phantomsection\label{\detokenize{nodes:nodes.nodes.Nodes.update_inputs_driver}}
\pysigstartsignatures
\pysiglinewithargsret
{\sphinxbfcode{\sphinxupquote{update\_inputs\_driver}}}
{\sphinxparam{\DUrole{n}{as\_DORA}}}
{}
\pysigstopsignatures
\sphinxAtStartPar
Update the inputs in the driver.
\begin{quote}\begin{description}
\sphinxlineitem{Parameters}
\sphinxAtStartPar
\sphinxstyleliteralstrong{\sphinxupquote{as\_DORA}} (\sphinxstyleliteralemphasis{\sphinxupquote{bool}}) \textendash{} Whether to use DORA mode.

\end{description}\end{quote}

\end{fulllineitems}

\index{update\_inputs\_recpient() (nodes.nodes.Nodes method)@\spxentry{update\_inputs\_recpient()}\spxextra{nodes.nodes.Nodes method}}

\begin{fulllineitems}
\phantomsection\label{\detokenize{nodes:nodes.nodes.Nodes.update_inputs_recpient}}
\pysigstartsignatures
\pysiglinewithargsret
{\sphinxbfcode{\sphinxupquote{update\_inputs\_recpient}}}
{\sphinxparam{\DUrole{n}{as\_DORA}}\sphinxparamcomma \sphinxparam{\DUrole{n}{phase\_set}}\sphinxparamcomma \sphinxparam{\DUrole{n}{lateral\_input\_level}}\sphinxparamcomma \sphinxparam{\DUrole{n}{ignore\_object\_semantics}\DUrole{o}{=}\DUrole{default_value}{False}}}
{}
\pysigstopsignatures
\sphinxAtStartPar
Update the inputs in the recipient.
\begin{quote}\begin{description}
\sphinxlineitem{Parameters}\begin{itemize}
\item {} 
\sphinxAtStartPar
\sphinxstyleliteralstrong{\sphinxupquote{as\_DORA}} (\sphinxstyleliteralemphasis{\sphinxupquote{bool}}) \textendash{} Whether to use DORA mode.

\item {} 
\sphinxAtStartPar
\sphinxstyleliteralstrong{\sphinxupquote{phase\_set}} (\sphinxstyleliteralemphasis{\sphinxupquote{int}}) \textendash{} The current phase set.

\item {} 
\sphinxAtStartPar
\sphinxstyleliteralstrong{\sphinxupquote{lateral\_input\_level}} (\sphinxstyleliteralemphasis{\sphinxupquote{float}}) \textendash{} The lateral input level.

\item {} 
\sphinxAtStartPar
\sphinxstyleliteralstrong{\sphinxupquote{ignore\_object\_semantics}} (\sphinxstyleliteralemphasis{\sphinxupquote{bool}}\sphinxstyleliteralemphasis{\sphinxupquote{, }}\sphinxstyleliteralemphasis{\sphinxupquote{optional}}) \textendash{} Whether to ignore object semantics input. Defaults to False.

\end{itemize}

\end{description}\end{quote}

\end{fulllineitems}


\end{fulllineitems}



\section{nodes.tensorOps module}
\label{\detokenize{nodes:module-nodes.tensorOps}}\label{\detokenize{nodes:nodes-tensorops-module}}\index{module@\spxentry{module}!nodes.tensorOps@\spxentry{nodes.tensorOps}}\index{nodes.tensorOps@\spxentry{nodes.tensorOps}!module@\spxentry{module}}\index{diag\_zeros() (in module nodes.tensorOps)@\spxentry{diag\_zeros()}\spxextra{in module nodes.tensorOps}}

\begin{fulllineitems}
\phantomsection\label{\detokenize{nodes:nodes.tensorOps.diag_zeros}}
\pysigstartsignatures
\pysiglinewithargsret
{\sphinxcode{\sphinxupquote{nodes.tensorOps.}}\sphinxbfcode{\sphinxupquote{diag\_zeros}}}
{\sphinxparam{\DUrole{n}{M}}}
{}
\pysigstopsignatures
\sphinxAtStartPar
Return MxM matrix of all ones except the diagonal from T{[}0, 0{]} to T{[}M, M{]}
:param M: The size of the matrix
:type M: int
\begin{quote}\begin{description}
\sphinxlineitem{Returns}
\sphinxAtStartPar
A MxM matrix of all ones except the diagonal from T{[}0, 0{]} to T{[}M, M{]}

\sphinxlineitem{Return type}
\sphinxAtStartPar
torch.Tensor

\end{description}\end{quote}

\end{fulllineitems}

\index{refine\_mask() (in module nodes.tensorOps)@\spxentry{refine\_mask()}\spxextra{in module nodes.tensorOps}}

\begin{fulllineitems}
\phantomsection\label{\detokenize{nodes:nodes.tensorOps.refine_mask}}
\pysigstartsignatures
\pysiglinewithargsret
{\sphinxcode{\sphinxupquote{nodes.tensorOps.}}\sphinxbfcode{\sphinxupquote{refine\_mask}}}
{\sphinxparam{\DUrole{n}{tensor}}\sphinxparamcomma \sphinxparam{\DUrole{n}{mask}}\sphinxparamcomma \sphinxparam{\DUrole{n}{index}}\sphinxparamcomma \sphinxparam{\DUrole{n}{value}}\sphinxparamcomma \sphinxparam{\DUrole{n}{in\_place}\DUrole{o}{=}\DUrole{default_value}{False}}}
{}
\pysigstopsignatures
\sphinxAtStartPar
Returns a mask, that is the union of mask and the submask where tensor{[}mask, index{]} == value
:param tensor: The input tensor
:type tensor: torch.Tensor
:param mask: The input mask
:type mask: torch.Tensor
:param index: The index of the value to check
:type index: int
:param value: The value to check for
:type value: int
:param in\_place: Whether to modify the input mask in place
:type in\_place: bool
\begin{quote}\begin{description}
\sphinxlineitem{Returns}
\sphinxAtStartPar
Mask(size of input mask) with union of input mask and submask where tensor{[}mask, index{]} == value

\sphinxlineitem{Return type}
\sphinxAtStartPar
torch.Tensor

\end{description}\end{quote}

\end{fulllineitems}

\index{sub\_union() (in module nodes.tensorOps)@\spxentry{sub\_union()}\spxextra{in module nodes.tensorOps}}

\begin{fulllineitems}
\phantomsection\label{\detokenize{nodes:nodes.tensorOps.sub_union}}
\pysigstartsignatures
\pysiglinewithargsret
{\sphinxcode{\sphinxupquote{nodes.tensorOps.}}\sphinxbfcode{\sphinxupquote{sub\_union}}}
{\sphinxparam{\DUrole{n}{mask}}\sphinxparamcomma \sphinxparam{\DUrole{n}{submask}}\sphinxparamcomma \sphinxparam{\DUrole{n}{in\_place}\DUrole{o}{=}\DUrole{default_value}{False}}}
{}
\pysigstopsignatures
\sphinxAtStartPar
Returns a mask, that is the union of the input mask and its submask
:param mask: The input mask
:type mask: torch.Tensor
:param submask: The submask
:type submask: torch.Tensor
:param in\_place: Whether to modify the mask in place
:type in\_place: bool
\begin{quote}\begin{description}
\sphinxlineitem{Returns}
\sphinxAtStartPar
Mask(size of input mask) with union of input mask and submask

\sphinxlineitem{Return type}
\sphinxAtStartPar
torch.Tensor

\end{description}\end{quote}

\end{fulllineitems}

\index{undirected() (in module nodes.tensorOps)@\spxentry{undirected()}\spxextra{in module nodes.tensorOps}}

\begin{fulllineitems}
\phantomsection\label{\detokenize{nodes:nodes.tensorOps.undirected}}
\pysigstartsignatures
\pysiglinewithargsret
{\sphinxcode{\sphinxupquote{nodes.tensorOps.}}\sphinxbfcode{\sphinxupquote{undirected}}}
{\sphinxparam{\DUrole{n}{T}}}
{}
\pysigstopsignatures
\sphinxAtStartPar
Returns the undirected matrix made by OR of both directions of a given matrix T
:param T: The input matrix
:type T: torch.Tensor
\begin{quote}\begin{description}
\sphinxlineitem{Returns}
\sphinxAtStartPar
The undirected matrix made by OR of both directions of T

\sphinxlineitem{Return type}
\sphinxAtStartPar
torch.Tensor

\end{description}\end{quote}

\end{fulllineitems}



\section{Module contents}
\label{\detokenize{nodes:module-nodes}}\label{\detokenize{nodes:module-contents}}\index{module@\spxentry{module}!nodes@\spxentry{nodes}}\index{nodes@\spxentry{nodes}!module@\spxentry{module}}

\renewcommand{\indexname}{Python Module Index}
\begin{sphinxtheindex}
\let\bigletter\sphinxstyleindexlettergroup
\bigletter{n}
\item\relax\sphinxstyleindexentry{nodes}\sphinxstyleindexpageref{nodes:\detokenize{module-nodes}}
\item\relax\sphinxstyleindexentry{nodes.nodeBuilder}\sphinxstyleindexpageref{nodes:\detokenize{module-nodes.nodeBuilder}}
\item\relax\sphinxstyleindexentry{nodes.nodeEnums}\sphinxstyleindexpageref{nodes:\detokenize{module-nodes.nodeEnums}}
\item\relax\sphinxstyleindexentry{nodes.nodeMemObjects}\sphinxstyleindexpageref{nodes:\detokenize{module-nodes.nodeMemObjects}}
\item\relax\sphinxstyleindexentry{nodes.nodePrinter}\sphinxstyleindexpageref{nodes:\detokenize{module-nodes.nodePrinter}}
\item\relax\sphinxstyleindexentry{nodes.nodes}\sphinxstyleindexpageref{nodes:\detokenize{module-nodes.nodes}}
\item\relax\sphinxstyleindexentry{nodes.nodeTensors}\sphinxstyleindexpageref{nodes:\detokenize{module-nodes.nodeTensors}}
\item\relax\sphinxstyleindexentry{nodes.nodeTests}\sphinxstyleindexpageref{nodes.nodeTests:\detokenize{module-nodes.nodeTests}}
\item\relax\sphinxstyleindexentry{nodes.nodeTests.test\_1}\sphinxstyleindexpageref{nodes.nodeTests:\detokenize{module-nodes.nodeTests.test_1}}
\item\relax\sphinxstyleindexentry{nodes.tensorOps}\sphinxstyleindexpageref{nodes:\detokenize{module-nodes.tensorOps}}
\end{sphinxtheindex}

\renewcommand{\indexname}{Index}
\printindex
\end{document}